\section{The Mean Value Theorems}

\begin{exercise}
  Recall from Exercise $4.4 .9$ that a function $f: A \rightarrow \mathbf{R}$ is Lipschitz on $A$ if there exists an $M>0$ such that
  $$
  \left|\frac{f(x)-f(y)}{x-y}\right| \leq M
  $$
  for all $x \neq y$ in $A$
  \enum{
  \item Show that if $f$ is differentiable on a closed interval $[a, b]$ and if $f^{\prime}$ is continuous on $[a, b]$, then $f$ is Lipschitz on $[a, b]$.
  \item Review the definition of a contractive function in Exercise 4.3.11. If we add the assumption that $\left|f^{\prime}(x)\right|<1$ on $[a, b]$, does it follow that $f$ is contractive on this set?
  }
\end{exercise}
\begin{solution}
  \enum{
  \item Since $f'$ is continuous on the compact set $[a,b]$ we can set $M$ such that $|f'(x)| \le M$ over $[a,b]$, then pick $x,y \in [a,b]$ with $x<y$. Apply MVT on $[x,y]$ to get a $c \in (x,y)$ with
    $$
    \frac{f(x)-f(y)}{x-y} = f'(c)
    $$
    Which implies
    $$
    \left|\frac{f(x)-f(y)}{x-y}\right| = |f'(c)| \le M
    $$
    Since $x,y$ were arbitrary this shows $f$ is Lipschitz.
  \item For $f$ to be contractive we need some $c \in (0, 1)$ with $|f(x)-f(y)| \le c|x-y|$.
    Set
    $$c = \sup \{|f'(x)| : x \in [a,b]\}$$
    Clearly $|f'(x)| \le c$, now apply the Extreme Value Theorem to find $d \in [a,b]$ with $|f'(d)| = c$, which implies $c < 1$ since $|f'(d)| < 1$.

    Now apply the MVT as in (a) to bound $|f(x) - f(y)| \le c|x-y|$, which shows $f$ is contractive.
  }
\end{solution}

\begin{exercise}
  Let $f$ be differentiable on an interval $A$. If $f^{\prime}(x) \neq 0$ on $A$, show that $f$ is one-to-one on $A$. Provide an example to show that the converse statement need not be true.
\end{exercise}
\begin{solution}
  Let $x,y$ be in $A$ with $x < y$, to show $f(x) \ne f(y)$ apply the Mean Value Theorem on $[x,y]$ to get $c \in (x,y)$ with
  $$
  f'(c) = \frac{f(x) - f(y)}{x-y}
  $$
  Now since $f'(c) \ne 0$ we must have $f(x) - f(y) \ne 0$, and thus $f(x) \ne f(y)$.

  To see the converse is false consider how $f(x) = x^3$ is 1-1 but has $f'(0) = 0$.
\end{solution}

\begin{exercise}
  Let $h$ be a differentiable function defined on the interval $[0,3]$, and assume that $h(0)=1, h(1)=2$, and $h(3)=2$.
  \enum{
  \item Argue that there exists a point $d \in[0,3]$ where $h(d)=d$.
  \item Argue that at some point $c$ we have $h^{\prime}(c)=1 / 3$.
  \item Argue that $h^{\prime}(x)=1 / 4$ at some point in the domain.
  }
\end{exercise}
\begin{solution}
  \TODO
\end{solution}

\begin{exercise}
  Let $f$ be differentiable on an interval $A$ containing zero, and assume $\left(x_{n}\right)$ is a sequence in $A$ with $\left(x_{n}\right) \rightarrow 0$ and $x_{n} \neq 0$.
  \enum{
  \item If $f\left(x_{n}\right)=0$ for all $n \in N$, show $f(0)=0$ and $f^{\prime}(0)=0$.
  \item Add the assumption that $f$ is twice-differentiable at zero and show that $f^{\prime \prime}(0)=0$ as well.
  }
\end{exercise}
\begin{solution}
  \TODO
\end{solution}

\begin{exercise}
  \enum{
  \item Supply the details for the proof of Cauchy's Generalized Mean Value Theorem (Theorem 5.3.5).
  \item Give a graphical interpretation of the Generalized Mean Value Theorem analogous to the one given for the Mean Value Theorem at the beginning of Section 5.3. (Consider $f$ and $g$ as parametric equations for a curve.)
  }
\end{exercise}
\begin{solution}
  \TODO
\end{solution}

\begin{exercise}
  \enum{
  \item Let $g:[0, a] \rightarrow \mathbf{R}$ be differentiable, $g(0)=0$, and $\left|g^{\prime}(x)\right| \leq M$ for all $x \in[0, a] .$ Show $|g(x)| \leq M x$ for all $x \in[0, a] .$
  \item Let $h:[0, a] \rightarrow \mathbf{R}$ be twice differentiable, $h^{\prime}(0)=h(0)=0$ and $\left|h^{\prime \prime}(x)\right| \leq$ $M$ for all $x \in[0, a] .$ Show $|h(x)| \leq M x^{2} / 2$ for all $x \in[0, a] .$
  \item Conjecture and prove an analogous result for a function that is differentiable three times on $[0, a]$.
  }
\end{exercise}
\begin{solution}
  \TODO
\end{solution}

\begin{exercise}
  A fixed point of a function $f$ is a value $x$ where $f(x)=x$. Show that if $f$ is differentiable on an interval with $f^{\prime}(x) \neq 1$, then $f$ can have at most one fixed point.
\end{exercise}
\begin{solution}
  \TODO
\end{solution}

\begin{exercise}
  Assume $f$ is continuous on an interval containing zero and differentiable for all $x \neq 0$. If $\lim _{x \rightarrow 0} f^{\prime}(x)=L$, show $f^{\prime}(0)$ exists and equals $L$.
\end{exercise}
\begin{solution}
  \TODO
\end{solution}

\begin{exercise}
  Assume $f$ and $g$ are as described in Theorem 5.3.6, but now add the assumption that $f$ and $g$ are differentiable at $a$, and $f^{\prime}$ and $g^{\prime}$ are continuous at $a$ with $g^{\prime}(a) \neq 0$. Find a short proof for the $0 / 0$ case of L'Hospital's Rule under this stronger hypothesis.
\end{exercise}
\begin{solution}
  \TODO
\end{solution}

\begin{exercise}
  Let $f(x)=x \sin \left(1 / x^{4}\right) e^{-1 / x^{2}}$ and $g(x)=e^{-1 / x^{2}}$. Using the familiar properties of these functions, compute the limit as $x$ approaches zero of $f(x), g(x), f(x) / g(x)$, and $f^{\prime}(x) / g^{\prime}(x)$. Explain why the results are surprising but not in conflict with the content of Theorem 5.3.6. ${ }^{1}$
\end{exercise}
\begin{solution}
  \TODO
\end{solution}

\begin{exercise}
  \enum{
  \item Use the Generalized Mean Value Theorem to furnish a proof of the $0 / 0$ case of L'Hospital's Rule (Theorem 5.3.6).
  \item If we keep the first part of the hypothesis of Theorem 5.3.6 the same but we assume that
    $$
    \lim _{x \rightarrow a} \frac{f^{\prime}(x)}{g^{\prime}(x)}=\infty
    $$
    does it necessarily follow that
    $$
    \lim _{x \rightarrow a} \frac{f(x)}{g(x)}=\infty ?
    $$
  }
\end{exercise}
\begin{solution}
  \TODO
\end{solution}

\begin{exercise}
  If $f$ is twice differentiable on an open interval containing $a$ and $f^{\prime \prime}$ is continuous at $a$, show
  $$
  \lim _{h \rightarrow 0} \frac{f(a+h)-2 f(a)+f(a-h)}{h^{2}}=f^{\prime \prime}(a) .
  $$
  (Compare this to Exercise 5.2.6(b).)
\end{exercise}
\begin{solution}
  \TODO
\end{solution}

