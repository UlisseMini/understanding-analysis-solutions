\section{Baire's Theorem}

\begin{exercise}
  Argue that a set $A$ is a $G_{\delta}$ set if and only if its complement is an $F_{\sigma}$ set.
\end{exercise}

\begin{solution}
  If $A$ is a $G_\delta$ set, then $A^c$ is a $F_\sigma$ set by De Morgan's laws.
  Likewise if $A$ is an $F_\sigma$ set then $A^c$ must be a $G_\delta$ set.
\end{solution}

\begin{exercise}
  Replace each with the word finite or countable, depending on which is more appropriate.
  \enum{
  \item The \blankk union of $F_{\sigma}$ sets is an $F_{\sigma}$ set.
  \item The \blankk intersection of $F_{\sigma}$ sets is an $F_{\sigma}$ set.
  \item The \blankk union of $G_{\delta}$ sets is a $G_{\delta}$ set.
  \item The \blankk intersection of $G_{\delta}$ sets is a $G_{\delta}$ set.
  }
\end{exercise}

\begin{solution}
  \enum {
  \item Countable, since two countable union can be written as a single countable union over the diagonal (see Exercise 1.2.4). Another way of seeing this is that we can form a bijection between $\mathbf N$ and $\mathbf N^2$, therefore a double infinite union can be written as a single infinite union.
  \item Finite
  \item Finite
  \item Countable, by the same logic as in (a) we can write two countable intersections as a single countable intersection.
  }
\end{solution}

\begin{exercise}
  \enum{
  \item Show that a closed interval $[a, b]$ is a $G_{\delta}$ set.
  \item Show that the half-open interval $(a, b]$ is both a $G_{\delta}$ and an $F_{\sigma}$ set.
  \item Show that $\mathbf{Q}$ is an $F_{\sigma}$ set, and the set of irrationals I forms a $G_{\delta}$ set.
  }
\end{exercise}

\begin{solution}
  This exercise has already appeared as Exercise 3.2.15.
\end{solution}

\begin{exercise}
  Let $\left\{G_{1}, G_{2}, G_{3}, \ldots\right\}$ be a countable collection of dense, open sets, we will prove that the intersection $\bigcap_{n=1}^{\infty} G_{n}$ is not empty.

  Starting with $n=1$, inductively construct a nested sequence of closed intervals $I_{1} \supseteq I_{2} \supseteq I_{3} \supseteq \cdots$ satisfying $I_{n} \subseteq G_{n}$. Give special attention to the issue of the endpoints of each $I_{n}$. Show how this leads to a proof of the theorem.
\end{exercise}

\begin{solution}
  Because $G_1$ is open there exists an open interval $(a_1,b_1) \subseteq G_1$, letting $[c_1, d_1]$ be a closed interval contained in $(a_1, b_1)$ gives $I_1 \subseteq G_1$ as desired.

  Now suppose $I_{n} \subseteq G_{n}$. because $G_{n+1}$ is dense and $(c_{n}, d_{n}) \cap G_{n+1}$ is open there exists an interval $(a_{n+1}, b_{n+1}) \subseteq G_n \cap (c_{n}, d_{n})$. Letting $[c_{n+1}, d_{n+1}] \subseteq (a_{n+1}, b_{n+1})$ gives us our new closed interval.

  This gives us our collection of sets with $I_{n+1} \subseteq I_n$, $I_n \subseteq G_n$ and $I_n \ne \emptyset$ allowing us to apply the Nested Interval Property to conclude
  $$
  \bigcap_{n=1}^\infty I_n \ne \emptyset
  $$
  and thus $\bigcap_{n=1}^\infty G_n \ne \emptyset$ since each $I_n \subseteq G_n$.
\end{solution}

\begin{exercise}
  Show that it is impossible to write
  $$
  \mathbf{R}=\bigcup_{n=1}^{\infty} F_{n}
  $$
  where for each $n \in \mathbf{N}, F_{n}$ is a closed set containing no nonempty open intervals.
\end{exercise}

\begin{solution}
  This is just the complement of Exercise 3.5.4, If we had $\mathbf R = \bigcup_{n=1}^\infty F_n$ then we would also have
  $$\emptyset = \bigcap_{n=1}^\infty G_n$$
  for $G_n = F_n^c$. $G_n$ is open as it is the complement of a closed set, and since $F_n$ contains no nonempty open intervals $G_n$ is dense. This contradicts $\bigcap_{n=1}^\infty G_n \ne \emptyset$  from 3.5.4.

  To be totally rigorous we still have to justify $F_n^c$ being dense. Let $a,b \in \mathbf{R}$ with $a < b$, since $(a,b) \not \subseteq F_n$ there exists a $c \in (a,b)$ with $c \in F_n^c$ and thus $F_n^c$ is dense.
\end{solution}

\begin{exercise}
  Show how the previous exercise implies that the set $\mathbf I$ of irrationals cannot be an $F_{\sigma}$ set, and $\mathbf{Q}$ cannot be a $G_{\delta}$ set.
\end{exercise}

\begin{solution}
  Recall from 3.5.3 that $\mathbf Q$ is an $F_\sigma$ set,
  suppose for contradiction that $\mathbf I$ were also an $F_\sigma$ set. Then we could write
  $$
  \mathbf{Q} = \bigcup_{n=1}^\infty F_n \quad\text{and}\quad \mathbf{I} = \bigcup_{n=1}^\infty F_n
  $$
  Each $F_n$ and $F_n'$ must contain no nonempty open intervals, since otherwise $F_n$ would contain irrationals and vise versa.
  Combine the countable unions by setting $\tilde F_{2n} = F_n$ and $\tilde F_{2n-1} = F_n'$ to get
  $$
  \mathbf{R} = \mathbf{Q} \cup \mathbf{I} = \bigcup_{n=1}^\infty \tilde F_n
  $$


  But in 3.5.5 we showed
  $$\mathbf{R} \ne \bigcup_{n=1}^\infty F_n$$
  which gives our desired contradiction, hence $\mathbf{I}$ is not an $F_\sigma$ set and $\mathbf{Q}$ is not a $G_{\delta}$ set (take complements).
\end{solution}

\begin{exercise}
  Using Exercise 3.5.6 and versions of the statements in Exercise 3.5.2, construct a set that is neither in $F_{\sigma}$ nor in $G_{\delta}$.
\end{exercise}

\begin{solution}
  \TODO
\end{solution}

\begin{exercise}
  Show that a set $E$ is nowhere-dense in $\mathbf{R}$ if and only if the complement of $\closure{E}$ is dense in $\mathbf{R}$.
\end{exercise}

\begin{solution}
  First suppose $E$ is nowhere-dense, then $\closure{E}$ contains no nonempty open intervals meaning for every $a,b \in \mathbf R$ we have $(a,b) \not \subseteq \closure{E}$ meaning we can find a $c \in (a,b)$ with $c \notin \closure{E}$. But this is just saying $c \in \closure{E}^c$ which implies $\closure{E}^c$ is dense since for every $a,b \in \mathbf R$ we can find a $c \in \closure{E}^c$ with $a < c < b$.

  Now suppose $\closure{E}^c$ is dense in $\mathbf R$, then then every interval $(a,b)$ contains a point $c \in \closure{E}^c$, implying that $(a,b) \not \subseteq \closure{E}$ since $c \notin \closure{E}$ and $c \in (a,b)$. therefore $\closure{E}$ contains no nonempty open intervals and so $E$ is nowhere-dense by defintion 3.5.3.
\end{solution}

\begin{exercise}
  Decide whether the following sets are dense in $\mathbf{R}$, nowhere-dense in $\mathbf{R}$, or somewhere in between.
  \enum{
  \item $A=\mathbf{Q} \cap[0,5]$.
  \item $B=\{1 / n: n \in \mathbf{N}\}$.
  \item the set of irrationals.
  \item the Cantor set.
  }
\end{exercise}

\begin{solution}
  \enum{
  \item between, since $A$ is dense in $[0,5]$ but not in all of $\mathbf R$.
  \item nowhere-dense since $\closure{B} = B \cup \{0\}$ contains no nonempty open intervals
  \item dense since $\closure{\mathbf I} = \mathbf R$
  \item between since $\closure{C} = [0,1]$
  }
\end{solution}

\begin{exercise}[Baire's Theorem]
  Prove set of real numbers $\mathbf{R}$ cannot be written as the countable union of nowhere-dense sets.

  To start, assume that $E_{1}, E_{2}, E_{3}, \ldots$ are each nowhere-dense and satisfy $\mathbf{R}=\bigcup_{n=1}^{\infty} E_{n}$ then find a contradiction to the results in this section.
\end{exercise}

\begin{solution}
  By the definition of $E_n$ being nowhere-dense, the closure $\overline{E_n}$ contains no nonempty open intervals meaning we can apply Exercise 3.5.5 to conclude that
  $$
  \bigcup_{n=1}^\infty \closure{E_n} \ne \mathbf{R}
  $$
  Since each $E_n \subseteq \closure{E_n}$ we have
  $$
  \bigcup_{n=1}^\infty E_n \ne \mathbf{R}
  $$
  as desired.
\end{solution}
