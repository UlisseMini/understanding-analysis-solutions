\section{Perfect Sets and Connected Sets}

\begin{exercise}
  If $P$ is a perfect set and $K$ is compact, is the intersection $P \cap K$ always compact? Always perfect?
\end{exercise}

\begin{solution}
  Recall a perfect set is a closed set with no isolated points. Thus the intersection of a closed set $P$ and a closed bounded set $K$ gives a closed bounded (and thus compact) set $P \cap K$.

  Now take $P = \mathbf{R}$, we get $P \cap K = K$ which is not nessesarily perfect.
\end{solution}

\begin{exercise}
  Does there exist a perfect set consisting of only rational numbers?
\end{exercise}

\begin{solution}
  No, since any nonempty set $P \subseteq \mathbf Q$ is countable but, nonempty perfect sets are uncountable by Theorem 3.4.3
\end{solution}

\begin{exercise}
  Review the portion of the proof given in Example 3.4.2 and follow these steps to complete the argument.
  \enum{
  \item Because $x \in C_{1}$, argue that there exists an $x_{1} \in C \cap C_{1}$ with $x_{1} \neq x$ satisfying $\left|x-x_{1}\right| \leq 1 / 3$.
  \item Finish the proof by showing that for each $n \in \mathbf{N}$, there exists $x_{n} \in C \cap C_{n}$, different from $x$, satisfying $\left|x-x_{n}\right| \leq 1 / 3^{n}$.
  }
\end{exercise}

\begin{solution}
  \enum{
  \item \TODO
  \item \TODO
  }
\end{solution}

\begin{exercise}
  Repeat the Cantor construction from Section $3.1$ starting with the interval $[0,1]$. This time, however, remove the open middle \emph{fourth} from each component.
  \enum{
  \item Is the resulting set compact? Perfect?
  \item Using the algorithms from Section 3.1, compute the length and dimension of this Cantor-like set.
  }
\end{exercise}

\begin{solution}
  \enum{
  \item \TODO
  \item \TODO
  }
\end{solution}

\begin{exercise}
  Let $A$ and $B$ be nonempty subsets of $\mathbf{R}$. Show that if there exist disjoint open sets $U$ and $V$ with $A \subseteq U$ and $B \subseteq V$, then $A$ and $B$ are separated.
\end{exercise}

\begin{solution}
  Disjoint open sets are separated, therefor so are their subsets.
\end{solution}

\begin{exercise}
  Prove that $A$ set $E \subseteq \mathbf{R}$ is connected if and only if, for all nonempty disjoint sets $A$ and $B$ satisfying $E=A \cup B$, there always exists a convergent sequence $\left(x_{n}\right) \rightarrow x$ with $\left(x_{n}\right)$ contained in one of $A$ or $B$, and $x$ an element of the other. (Theorem 3.4.6)

\end{exercise}

\begin{solution}
  Both are obvious if you think about the definitions, here's some formal(ish) garbage though

  Suppose $\closure{A} \cup B$ is nonempty and let $x$ be an element in both, $x \in B$ implies $x \notin A$ therefor $x \in L$ (the set of limit points of $A$) meaning there must exist a sequence $(x_n) \to x$ contained in $A$.

  Now suppose there exists an $(x_n) \to x$ in $A$ with limit in $B$, then clearly $\closure{A} \cap B \subseteq \{x\}$ is nonempty.
\end{solution}

\begin{exercise}
  A set $E$ is totally disconnected if, given any two distinct points $x, y \in E$, there exist separated sets $A$ and $B$ with $x \in A, y \in B$, and $E=A \cup B$.
  \enum{
  \item Show that $\mathbf{Q}$ is totally disconnected.
  \item Is the set of irrational numbers totally disconnected?
  }
\end{exercise}

\begin{solution}
  \enum{
  \item Let $x,y \in \mathbf Q$, and let $z \in (x,y)$ with $z \in \mathbf I$. The sets $A = (-\infty, z) \cap \mathbf Q$ and $B = (z, \infty) \cap \mathbf Q$ are separated and have $A \cup B = \mathbf Q$.
  \item Now let $x,y \in \mathbf I$, and let $z \in (x,y)$ with $z \in \mathbf Q$. The sets $A = (-\infty, z) \cap \mathbf I$ and $B = (z, \infty) \cap \mathbf I$ are separated and have $A \cup B = \mathbf I$.
  }
\end{solution}

\begin{exercise}
  Follow these steps to show that the Cantor set is totally disconnected in the sense described in Exercise 3.4.7.
  Let $C=\bigcap_{n=0}^{\infty} C_{n}$, as defined in Section 3.1.
  \enum{
  \item Given $x, y \in C$, with $x<y$, set $\epsilon=y-x$. For each $n=0,1,2, \ldots$, the set $C_{n}$ consists of a finite number of closed intervals. Explain why there must exist an $N$ large enough so that it is impossible for $x$ and $y$ both to belong to the same closed interval of $C_{N}$.
  \item Show that $C$ is totally disconnected.
  }
\end{exercise}

\begin{solution}
  \enum{
  \item Since the length of every interval goes to zero, we set $N$ large enough that the length of every interval is less then $\epsilon$, meaning $x$ and $y$ cannot be in the same interval.
  \item Obvious
  }
\end{solution}

\begin{exercise}
  Let $\left\{r_{1}, r_{2}, r_{3}, \ldots\right\}$ be an enumeration of the rational numbers, and for each $n \in \mathbf{N}$ set $\epsilon_{n}=1 / 2^{n}$. Define $O=\bigcup_{n=1}^{\infty} V_{\epsilon_{n}}\left(r_{n}\right)$, and let $F=O^{c}$.
  \enum{
  \item Argue that $F$ is a closed, nonempty set consisting only of irrational numbers.
  \item Does $F$ contain any nonempty open intervals? Is $F$ totally disconnected? (See Exercise 3.4.7 for the definition.)
  \item Is it possible to know whether $F$ is perfect? If not, can we modify this construction to produce a nonempty perfect set of irrational numbers?
  }
\end{exercise}

\begin{solution}
  \enum{
  \item \TODO
  \item \TODO
  \item \TODO
  }
\end{solution}

