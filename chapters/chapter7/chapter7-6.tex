\section{Lebesgue's Criterion for Riemann Integrability}
\begin{exercise} Recall that Thomae's function
$$
t(x)= \begin{cases}1 & \text { if } x=0 \\ 1 / n & \text { if } x=m / n \in \mathbf{Q} \backslash\{0\} \text { is in lowest terms with } n>0 \\ 0 & \text { if } x \notin \mathbf{Q}\end{cases}
$$
has a countable set of discontinuities occurring at precisely every rational number. Let's prove that Thomae's function is integrable on \([0,1]\) with \(\int^1_0 t = 0\).
\enum{
\item First argue that $L(t, P)=0$ for any partition $P$ of $[0,1]$.
\item Let $\epsilon>0$, and consider the set of points $D_{\epsilon / 2}=\{x \in[0,1]: t(x) \geq \epsilon / 2\}$. How big is $D_{\epsilon / 2}$ ?
\item To complete the argument, explain how to construct a partition $P_{\epsilon}$ of $[0,1]$ so that $U\left(t, P_{\epsilon}\right)<\epsilon$.
}
\end{exercise}
\begin{solution}
See Exercise 7.3.2
\end{solution}

\begin{exercise}
We first met the Cantor set \(C\) in Section 3.1. We have since learned that \(C\) is a compact, uncountable subset of the interval \([0, 1]\).

Define \[h(x) = \begin{cases}
   1 & \text{if } x \in C \\
   0 & \text{if }x \notin C
\end{cases}\]

\enum{
    \item Show \(h\) has discontinuities at each point of \(C\) and is continuous at every point of the complement of \(C\). Thus, \(h\) is not continuous on an uncountably infinite set.
    \item Now prove that \(h\) is integrable on \([0,1]\).
}
\end{exercise}
\begin{solution}
    See Exercise 7.3.9 (d)
\end{solution}

\begin{exercise}
Show that any countable set has measure zero.
\end{exercise}
\begin{solution}
Enumerate the elements of any countable set \(A\) as \(\{a_1, a_2, \cdots\}\). Given \(\epsilon > 0\), define the collection of open intervals \(O_n\) with \(O_i = (a_i - \frac{\epsilon}{2^{i + 1}}, a_i + \frac{\epsilon}{2^{i+1}})\).
\end{solution}

\begin{exercise}
Prove that the Cantor set has measure zero.
\end{exercise}

\begin{solution}
Exercise 7.3.9 (c) showed that the Cantor set has content zero, and any set with content zero clearly has measure zero.
\end{solution}

\begin{exercise}
Show that if two sets \(A\) and \(B\) each have measure zero, then \(A \cup B\) has measure zero as well. In addition, discuss the proof of the stronger statement that the countable union of sets of measure zero also has measure zero.
\end{exercise}

\begin{solution}
Let \(\epsilon > 0\), and define \(O_{A,n}\) and \(O_{B,n}\) so that
\[A \subseteq \bigcup^\infty_{n=1} O_{A,n} \text{, }
B \subseteq \bigcup^\infty_{n=1} O_{B,n} \text{, }
\sum^\infty_{n=1}|O_{A,n}| \leq \frac{\epsilon}{2} \text{, }
\sum^\infty_{n=1}|O_{B,n}| \leq \frac{\epsilon}{2}
\]
Then let \(O_n\) be the union of the sets in \(O_{A,n}\) and \(O_{B,n}\); \(O_n\) satisifies the conditions necessary to show \(A \cup B\) has measure zero.

This argument is essentially identical for a countable union of sets of measure zero. Some key points:
\begin{itemize}
    \item The countable union of countable sets is also countable.
    \item Assuming the sets are enumerated as \(A_m\), have each \(\sum |O_{A_m, n}| < \frac{\epsilon}{2^m}\), to ensure the total sum by the end is \(\leq \epsilon\)
    \item Since \(|O_{A_m, n}| \geq 0\), the infinte sums involved all converge absolutely, so we can use the results in Section 2.8 to safely sum the lengths of intervals over an enumeration of the final collection of sets \(O_n\).
\end{itemize}
\end{solution}

\begin{exercise}
If \(\alpha < \alpha'\), show that \(D^{\alpha'} \subseteq D^\alpha\).
\end{exercise}

\begin{solution}
See Exercise 4.6.9.
\end{solution}

\begin{exercise}
\enum{
\item Let \(\alpha > 0\) be given. Show that if \(f\) is continuous at \(x \in [a,b]\), then it is \(\alpha\)-continuous at \(x\) as well. Explain how it follows that \(D^\alpha \subseteq D\).
\item Show that if \(f\) is not continuous at \(x\), then \(f\) is not \(\alpha\)-continuous for some \(\alpha > 0\). Now, explain why this guarentees that
\[D = \bigcup^\infty_{n=1} D^{\alpha_n} \text{\quad where } \alpha_n = 1/n\]
}
\end{exercise}
\begin{solution}
\enum{
    \item See Exercise 4.6.10
    \item See Exercise 4.6.11
}
\end{solution}

\begin{exercise}
Prove that for a fixed \(\alpha > 0\), the set \(D^\alpha\) is closed.
\end{exercise}

\begin{solution}
    See Exercise 4.6.8
\end{solution}

\begin{exercise}
Show that there exists a \emph{finite} collection of disjoint open intervals \(\{G_1, G_2, \dots, G_N\}\) whose union contains \(D^\alpha\) and that satisfies
\[\sum^N_{n=1} \abs{G_n} < \frac{\epsilon}{4M}\]
\end{exercise}

\begin{solution}
Given \(D\) has measure zero, and since \(D^\alpha \subseteq D\), we also know that \(D^\alpha\) has measure zero. We can thus construct a countable open cover \(\{H_1, \dots\}\) such that \(\sum^\infty_{n=1} |G_n| < \frac{\epsilon}{4M}\).

Since \(D^\alpha\) is closed, we can find a finite subcover \(\{I_1, \dots, I_P\}\). Finally, since this is a finite set, we can merge any overlapping intervals (which can only decrease the total length of the intervals), leaving us with the desired finite collection of disjoint open intervals.
\end{solution}

\begin{exercise}
Let \(K\) be what remains of the interval \([a,b]\) after the open intervals \(G_n\) are all removed; that is, \(K = [a,b]\backslash \bigcup^N_{n=1} G_n\). Argue that \(f\) is uniformly \(\alpha\)-continuous on \(K\).
\end{exercise}
\begin{solution}
    Note that since each \(G_n\) is open, and the union of any collection of open sets is open, \(\bigcup^N_{n=1} G_n\) is open, its complement is closed, and \(K = [a,b] \cap (\bigcup^N_{n=1} G_n)^c\) is closed. \(K\) is also bounded so \(K\) is compact.

    By how we defined \(G_n\), \(f\) is \(\alpha\)-continuous pointwise on \(K\). Taking it as fact that \(\alpha\)-continuity on a compact set implies uniform \(\alpha\)-continuity, since \(K\) is compact, we have that \(f\) is uniformly \(\alpha\)-continuous on \(K\).
\end{solution}
