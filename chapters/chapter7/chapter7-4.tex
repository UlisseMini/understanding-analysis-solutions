\section{Properties of the Integral}

\begin{exercise} Let $f$ be a bounded function on a set $A$, and set
$$
\begin{gathered}
M=\sup \{f(x): x \in A\}, \quad m=\inf \{f(x): x \in A\}, \\
M^{\prime}=\sup \{|f(x)|: x \in A\}, \quad \text { and } \quad m^{\prime}=\inf \{|f(x)|: x \in A\} .
\end{gathered}
$$
\enum{
\item Show that $M-m \geq M^{\prime}-m^{\prime}$.
\item Show that if $f$ is integrable on the interval $[a, b]$, then $|f|$ is also integrable on this interval.
\item Provide the details for the argument that in this case we have $\left|\int_{a}^{b} f\right| \leq \int_{a}^{b}|f|$.
}
\end{exercise}
\begin{solution}
\enum{
\item If \(f(x) \geq 0\) then \(M' = M, m' = m\), and the result is trivial. Similarly if \(f(x) \leq 0\) then \(M' = -m,\ m' = -M\). Finally if \(f(x)\) has both positive and negative values then \(M \geq 0,\ m \leq 0,\ M' = \max\{M, -m\},\ m' \geq 0,\ M - m = M + (-m) \geq M' \geq M' - m'\)
\item Given part (a), if we have a partition \(P\) so that \(U(f, P) - L(f,P) < \epsilon\), then we also have \(U(|f|, P) - L(|f|, P) \leq U(f, P) - L(f,P) \epsilon\).
\item \(f \leq |f|\), so \(\int^b_a f \leq \int^b_a |f|\). Similarly \(\int^b_a f \geq \int^b_a -|f| = -\int^b_a |f|\), which together means \(\left|\int_a^b f\right| \leq \int_a^b|f|\)
}
\end{solution}

\begin{exercise}
\enum{
\item Let $g(x)=x^{3}$, and classify each of the following as positive, negative, or zero.
$$
\text { (i) } \int_{0}^{-1} g+\int_{0}^{1} g \quad \text { (ii) } \int_{1}^{0} g+\int_{0}^{1} g \quad \text { (iii) } \int_{1}^{-2} g+\int_{0}^{1} g \text {. }
$$
\item Show that if $b \leq a \leq c$ and $f$ is integrable on the interval $[b, c]$, then it is still the case that $\int_{a}^{b} f=\int_{a}^{c} f+\int_{c}^{b} f$.
}
\end{exercise}
\begin{solution}
    \TODO
\end{solution}

\begin{exercise} Decide which of the following conjectures is true and supply a short proof. For those that are not true, give a counterexample.
\enum{
\item If $|f|$ is integrable on $[a, b]$, then $f$ is also integrable on this set.
\item Assume $g$ is integrable and $g(x) \geq 0$ on $[a, b]$. If $g(x)>0$ for an infinite number of points $x \in[a, b]$, then $\int_{a}^{b} g>0$.
\item If $g$ is continuous on $[a, b]$ and $g(x) \geq 0$ with $g(y_0)>0$ for at least one point $y_{0} \in[a, b]$, then $\int_{a}^{b} g>0$.
}
\end{exercise}
\begin{solution}
\enum{
\item Letting \(d\) be Dirichlet's function, let \(f(x) = d(x) - 1/2\) (not integrable), with \(|f| = 1/2\) (integrable)
\item \(h(x)\) in Exercise 7.3.9 is a counterexample
\item Let \(\epsilon = g(y_0) /2 \). Since \(g\) is continuous there must be some \(\delta> 0\) so that over \(V_\delta(y_0)\),\(g > y_0 / 2\). Then let
\[h(x) = \begin{cases}
    g(y_0) / 2 & x \in V_\delta(y_0) \\
    0 & \text{otherwise}
\end{cases}\]
then since \(g \geq h\), \(\int^b_a g \geq \int^b_a h = \frac{\delta g(y_0)}{2} > 0\).
}
\end{solution}

\begin{exercise} Show that if $f(x)>0$ for all $x \in[a, b]$ and $f$ is integrable, then $\int_{a}^{b} f>0$.
\end{exercise}
\begin{solution}
I claim that there exists some non-empty interval \(I = [c,d] \subseteq [a,b]\) and some \(\epsilon > 0\) such that for all non-empty subintervals \(J \subseteq I\), there exists \(x \in J\) with \(f(x) \geq \epsilon\). Note that once we prove this claim, we can readily show \(\int^b_a f \geq \int^b_a g = \epsilon (d-c)\), where
\[g(x) = \begin{cases}
    \epsilon & x \in I \\
    0 & \text{otherwise}
\end{cases}\]
We'll prove this claim by contradiction. Suppose that for all non-empty intervals \(I \subseteq [a,b]\) and for all \(\epsilon > 0\), there exists a non-empty subinterval of \(J \subseteq I\) where \(f(x) < \epsilon \ \forall x \in J \).

Now let \(I_0 = [a,b]\), and for \(n > 0\), let \(\epsilon_n = 1/n\) and \(I_n \subseteq I_{n-1}\) satisfying \(f(x) < \epsilon_n \ \forall x \in I_n\). Now by the Nested Interval Property, \(\exists s \in \bigcap^\infty_{n=1} I_n \). Now, \(s\) satisfies \(f(s) < 1/n\) for all \(n\), implying \(f(s) \leq 0\) --- a contradiction since \(f(x) > 0\) for all \(x \in [a,b]\).
\end{solution}

\begin{exercise} Let $f$ and $g$ be integrable functions on $[a, b]$.
\enum{
\item Show that if $P$ is any partition of $[a, b]$, then
$$
U(f+g, P) \leq U(f, P)+U(g, P) .
$$
Provide a specific example where the inequality is strict. What does the corresponding inequality for lower sums look like?
\item Review the proof of Theorem 7.4.2 (ii), and provide an argument for part (i) of this theorem.
}
\end{exercise}
\begin{solution}
    \TODO
\end{solution}

\begin{exercise} Although not part of Theorem 7.4.2, it is true that the product of integrable functions is integrable. Provide the details for each step in the following proof of this fact:
\enum{
\item If $f$ satisfies $|f(x)| \leq M$ on $[a, b]$, show
$$
\left|(f(x))^{2}-(f(y))^{2}\right| \leq 2 M|f(x)-f(y)|
$$
\item Prove that if $f$ is integrable on $[a, b]$, then so is $f^{2}$.
\item Now show that if $f$ and $g$ are integrable, then $f g$ is integrable. (Consider $(f+g)^{2}$.)
}
\end{exercise}
\begin{solution}
\enum{
\item \[\abs{f(x)^2 - f(y)^2} = \abs{f(x) + f(y)}\abs{f(x) - f(y)} \leq 2M \abs{f(x) - f(y)}\]

\item Consider some subinterval \([c,d] \subseteq [a,b]\), with:
\[m_k = \inf\{f(x) : x \in [c,d]\}\] \[M_k = \sup\{f(x) : x \in [c,d]\}\]
\[m'_k = \inf\{f(x)^2 : x \in [c,d]\}\] \[M'_k = \sup\{f(x)^2 : x \in [c,d]\}\]
We have that \(M'_k - m'_k \leq 2M (M_k - m_k) \); this implies that for any partition \(P\), \[U(f^2, P) - L(f^2, P) \leq 2M \left(U(f,P) - L(f,P)\right)\]
which since \(M\) is constant, implies \(f^2\) is integrable.

\item If \(f\) and \(g\) are integrable, then so are \(f+g\), \((f+g)^2 = f^2 + 2fg + g^2\), \(2fg\), and \(fg\).
}
\end{solution}

\begin{exercise} Review the discussion immediately preceding Theorem 7.4.4.
\enum{
\item Produce an example of a sequence $f_{n} \rightarrow 0$ pointwise on $[0,1]$ where $\lim _{n \rightarrow \infty} \int_{0}^{1} f_{n}$ does not exist.
\item Produce an example of a sequence $g_{n}$ with $\int_{0}^{1} g_{n} \rightarrow 0$ but $g_{n}(x)$ does not converge to zero for any $x \in[0,1]$. To make it more interesting, let's insist that $g_{n}(x) \geq 0$ for all $x$ and $n$.
}
\end{exercise}
\begin{solution}
\enum{
\item \[f_n(x) = \begin{cases}
   n^2 & x \in (0, 1/n) \\
   0 & \text{otherwise}
\end{cases}\]
\item Define the set \(S_{i,j} = [(i-1)/j, i/j]\) for \(j > i > 0\), and the sequence of sets \(R_i\) enumerating through all \(S_{i,j}\); specifically \(R_1 = S_{1,1},\ R_2 = S_{1,2},\ R_3 = S_{2,2},\ R_4 = S_{1,3}\), and so on. Let
\[g_n(x) = \begin{cases}
    1 & x \in R_n \\
    0 & \text{otherwise}
\end{cases}\]
Each \(g_n\) looks like a ``pulse'', and as \(n\) increases and we increase the \(j\) index of \(S_{i,j}\), the pulse gets increasingly narrower; in particular, if \(g_n\) is based on \(S_{i,j}\) then \(\int^1_0 g_n = 1/j\). This ensures \(\lim_{n \to \infty} \int^1_0 g_n = 0\). However, as we go through the \(i\) index, we slide the pulse over all values in \([0,1]\). A bit more formally, for all \(x \in [0,1]\) and for all \(j > 0\), \(x \in S_{i,j}\) for some \(i\). This ensures that for all \(x \in [0,1]\), \(g_n(x) = 1\)  infinitely many times, preventing \(g_n(x) \to 0\).
}
\end{solution}

\begin{exercise} For each $n \in \mathbf{N}$, let
$$
h_{n}(x)=\left\{\begin{array}{ll}
1 / 2^{n} & \text { if } 1 / 2^{n}<x \leq 1 \\
0 & \text { if } 0 \leq x \leq 1 / 2^{n}
\end{array},\right.
$$
and set $H(x)=\sum_{n=1}^{\infty} h_{n}(x)$. Show $H$ is integrable and compute $\int_{0}^{1} H$.

\end{exercise}
\begin{solution}
\(\int_0^1 h_i = (1/2)^n - (1/4)^n\). \(H_n(x) = \sum^n_{i=1}h_i(x)\) is integrable, and \((H_n) \to H\) uniformly, so an application of the geometric series formula and the Integrable Limit Theorem gives us \(\int^1_0 H = 2/3\).
\end{solution}

\begin{exercise}
Let $g_{n}$ and $g$ be uniformly bounded on $[0,1]$, meaning that there exists a single $M>0$ satisfying $|g(x)| \leq M$ and $\left|g_{n}(x)\right| \leq M$ for all $n \in \mathbf{N}$ and $x \in[0,1]$. Assume $g_{n} \rightarrow g$ pointwise on $[0,1]$ and uniformly on any set of the form $[0, \alpha]$, where $0<\alpha<1$.
If all the functions are integrable, show that $\lim _{n \rightarrow \infty} \int_{0}^{1} g_{n}=\int_{0}^{1} g$.
\end{exercise}

\begin{solution}
    \TODO
\end{solution}

\begin{exercise} Assume $g$ is integrable on $[0,1]$ and continuous at 0 . Show
$$
\lim _{n \rightarrow \infty} \int_{0}^{1} g\left(x^{n}\right) d x=g(0)
$$
\end{exercise}
\begin{solution}
    \TODO
\end{solution}

\begin{exercise}
Review the original definition of integrability in Section 7.2, and in particular the definition of the upper integral $U(f)$. One reasonable suggestion might be to bypass the complications introduced in Definition 7.2.7 and simply define the integral to be the value of $U(f)$. Then every bounded function is integrable! Although tempting, proceeding in this way has some significant drawbacks. Show by example that several of the properties in Theorem 7.4.2 no longer hold if we replace our current definition of integrability with the proposal that $\int_{a}^{b} f=U(f)$ for every bounded function $f$.
\end{exercise}
\begin{solution}
    \TODO
\end{solution}
