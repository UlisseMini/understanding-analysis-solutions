\section{The Fundamental Theorem of Calculus}

\begin{exercise}
\enum{
\item Let $f(x)=|x|$ and define $F(x)=\int_{-1}^{x} f$. Find a piecewise algebraic formula for $F(x)$ for all $x$. Where is $F$ continuous? Where is $F$ differentiable? Where does $F^{\prime}(x)=f(x)$ ?
\item Repeat part (a) for the function
$$
f(x)= \begin{cases}1 & \text { if } x<0 \\ 2 & \text { if } x \geq 0\end{cases}
$$
}
\end{exercise}
\begin{solution}
    \TODO
\end{solution}

\begin{exercise}
Decide whether each statement is true or false, providing a short justification for each conclusion.
\enum{
\item If $g=h^{\prime}$ for some $h$ on $[a, b]$, then $g$ is continuous on $[a, b]$.
\item If $g$ is continuous on $[a, b]$, then $g=h^{\prime}$ for some $h$ on $[a, b]$.
\item If $H(x)=\int_{a}^{x} h$ is differentiable at $c \in[a, b]$, then $h$ is continuous at $c$.
}
\end{exercise}
\begin{solution}
\enum{
\item False, e.g. \(h(x) = x^2 \sin (1/x)\) as explored in section 5.1.
\item True; if \(g\) is continuous then it is also integrable, and by the Fundamental Theorem of Calculus we have that \(G(x) = \int^x_a g\) is differentiable over \([a,b]\) with \(G' = g\).
\item False; consider Thomae's function \(t(x)\). From Exercise 7.3.2 we have that \(t\) is integrable with \(\int^1_0 t = 0\); it's also easy to show that \(\int^x_0 t = 0\). Then \(H(x) = 0\) is differentiable over \([0,1]\), but \(t(x)\) is not continuous over \([0,1]\).
}
\end{solution}

\begin{exercise} The hypothesis in Theorem 7.5.1 (i) that $F^{\prime}(x)=f(x)$ for all $x \in[a, b]$ is slightly stronger than it needs to be. Carefully read the proof and state exactly what needs to be assumed with regard to the relationship between $f$ and $F$ for the proof to be valid.
\end{exercise}
\begin{solution}
We use the Mean Value Theorem to ensure that for all subintervals of a particular partition \([x_{k-1}, x_k]\), we have \(F(x_k) - F(x_{k-1}) = F'(t_k) (x_k - x_{k-1})\). But the MVT does not require that \(F\) is differentiable at the endpoints, so we can relax the requirement slightly and have \(F'(x) = f(x)\) over the open interval \((a,b)\).
\end{solution}

\begin{exercise} Show that if $f:[a, b] \rightarrow \mathbf{R}$ is continuous and $\int_{a}^{x} f=0$ for all $x \in[a, b]$, then $f(x)=0$ everywhere on $[a, b]$. Provide an example to show that this conclusion does not follow if $f$ is not continuous.
\end{exercise}
\begin{solution}
Since \(f\) is continuous, by Theorem 7.5.1 (ii), letting \(F(x) = \int^x_a f = 0\), we have \(f(x) = F'(x) = 0\). If \(f\) is not continuous, then this does not hold (e.g. Thomae's function over \([0,1]\))
\end{solution}

\begin{exercise}
The Fundamental Theorem of Calculus can be used to supply a shorter argument for Theorem 6.3.1 under the additional assumption that the sequence of derivatives is continuous.

Assume $f_{n} \rightarrow f$ pointwise and $f_{n}^{\prime} \rightarrow g$ uniformly on $[a, b]$. Assuming each $f_{n}^{\prime}$ is continuous, we can apply Theorem 7.5.1 (i) to get
$$
\int_{a}^{x} f_{n}^{\prime}=f_{n}(x)-f_{n}(a)
$$
for all $x \in[a, b]$. Show that $g(x)=f^{\prime}(x)$.
\end{exercise}
\begin{solution}
Since \(f'_n \to g\) uniformly, \(g\) is continuous. The Integrable Limit Theorem tells us \(\int^x_a f'_n \to \int^x_a g = f(x) - f(a)\), and Theorem 7.5.1 (ii) tells us \(G(x) = f(x) - f(a)\) satisfies \[G'(x) = f'(x) = g(x)\]
(since \(f(a)\) is constant).
\end{solution}

\begin{exercise}[Integration-by-parts]
\enum{
\item Assume $h(x)$ and $k(x)$ have continuous derivatives on $[a, b]$ and derive the familiar integration-by-parts formula
$$
\int_{a}^{b} h(t) k^{\prime}(t) d t=h(b) k(b)-h(a) k(a)-\int_{a}^{b} h^{\prime}(t) k(t) d t
$$
\item Explain how the result in Exercise 7.4.6 can be used to slightly weaken the hypothesis in part (a).
}
\end{exercise}
\begin{solution}
    \TODO
\end{solution}

\begin{exercise} Use part (ii) of Theorem 7.5.1 to construct another proof of part (i) of Theorem 7.5.1 under the stronger hypothesis that $f$ is continuous. (To get started, set \(G(x)=\int_{a}^{x} f\).)
\end{exercise}
\begin{solution}
    \TODO
\end{solution}

\begin{exercise}[Natural Logarithm and Euler's Constant]
    Let
$$
L(x)=\int_{1}^{x} \frac{1}{t} d t
$$
where we consider only $x>0$.
\enum{
\item What is $L(1)$? Explain why $L$ is differentiable and find $L^{\prime}(x)$.
\item Show that $L(x y)=L(x)+L(y)$. (Think of $y$ as a constant and differentiate $g(x)=L(x y)$)
\item Show $L(x / y)=L(x)-L(y)$.
\item Let
$$
\gamma_{n}=\left(1+\frac{1}{2}+\frac{1}{3}+\cdots+\frac{1}{n}\right)-L(n)
$$
Prove that $\left(\gamma_{n}\right)$ converges. The constant $\gamma=\lim \gamma_{n}$ is called Euler's constant.
\item Show how consideration of the sequence $\gamma_{2 n}-\gamma_{n}$ leads to the interesting identity
$$
L(2)=1-\frac{1}{2}+\frac{1}{3}-\frac{1}{4}+\frac{1}{5}-\frac{1}{6}+\cdots .
$$
}
\end{exercise}
\begin{solution}
\enum{
\item \(L(1) = 0\). Since \(1/t\) is continuous, by Theorem 7.5.1 (ii) \(L'(x) = 1/t\).
\item Differenting \(L(xy)\) with respect to \(x\) leaves us with \(1/x\), so by Theorem 7.5.1 (i) we have
\[\int^x_1 \frac{1}{t}dt = L(x) - L(1) = L(x) = L(xy) - L(y) \implies L(xy) = L(x) + L(y)\]
\item Differenting \(L(1/x)\) with respect to \(x\) leaves us with \(-1/x\), so
\[\int^x_1 \frac{-1}{t}dt = L(1/x) = -L(x) = -\int^x_1 \frac{1}{t} dt \]
Then \(L(x/y) = L(x) + L(1/y) = L(x) - L(y)\) as desired.
\item For conciseness let \(h_n = \sum^n_{i=1} \frac{1}{n}\) and \(f(t) = \frac{1}{t}\). Consider
\(\int^n_1 f\).
By definition, this is \(L(n)\). Consider the partition \(P = \{x : 1 \leq x \leq n,\ x \in \mathbf{N} \}\). Since \(1/t\) is decreasing, we have
\[U(f,P) = \sum^{n-1}_{i=1} f = h_n - \frac{1}{n} \text{ and } L(f,P) = \sum^n_{i=2} f = h_n - 1\]
We therefore have \(h_n - L(n) \geq h_n - U(f, P) = \frac{1}{n}\) and \(h_n - L(n) \leq h_n - L(f,P) = 1\). This indicates \((\gamma_n)\) is bounded.  Now note that

\[\gamma_{n+1} - \gamma_n = \frac{1}{n+1} - \int^{n+1}_n \frac{1}{x}dx \leq 0\]
where \(\frac{1}{n+1} = L(f, \{n,n+1\})\). This indicates that the sequence \((\gamma_n)\) is also monotone, so by the Monotone Convergence Theorem \((\gamma_n)\) converges.

\item \[\gamma_{2n} - \gamma_n = h_{2n} - h_n - L(2n) + L(n) = h_{2n} - h_n - L(2)\]
If we pair every element of \(h_n\) with every other element of \(h_{2n}\) we get
\[\gamma_{2n} - \gamma_n + L(2) = \sum^{2n}_{i=1} \frac{(-1)^{i+1}}{i} \]
Since \((\gamma_n)\) converges, the left side converges to \(L(2)\), giving us the desired identity.
}
\end{solution}

\begin{exercise}
Given a function $f$ on $[a, b]$, define the total variation of $f$ to be
$$
V f=\sup \left\{\sum_{k=1}^{n}\left|f\left(x_{k}\right)-f\left(x_{k-1}\right)\right|\right\},
$$
where the supremum is taken over all partitions $P$ of $[a, b]$.
\enum{
\item If $f$ is continuously differentiable ($f^{\prime}$ exists as a continuous function), use the Fundamental Theorem of Calculus to show $V f \leq \int_{a}^{b}\left|f^{\prime}\right|$.
\item Use the Mean Value Theorem to establish the reverse inequality and conclude that $V f=\int_{a}^{b}\left|f^{\prime}\right|$.
}
\end{exercise}
\begin{solution}
\enum{
\item For any subinterval \([c,d] \subseteq [a,b]\),
\[\int^d_c \abs{f'(x)} \geq \abs{\int^d_c f'(x)} = \abs{f(d) - f(c)}\]
Applying this to each subinterval of any partition \(P\) leaves us with
\[\sum_{k=1}^n \abs{f(x_k)-f(x_{k-1})} \leq \int^b_a \abs{f'}\]
and hence \(Vf \leq \int^b_a |f'|\).

\item Let \(\epsilon > 0\). Since \(|f'|\) is uniformly continuous over the closed interval \([a,b]\), we can create a partition \(P\) with \(n\) elements so that each subinterval has length less than \(\delta\), where \(|x-y| < \delta \implies \abs{ \abs{f'(x)} - \abs{f'(y)} } < \frac{\epsilon}{b-a}\).

Within a representative subinterval \([x_{k-1}, x_k]\) of \(P\), by the Mean Value Theorem \(\exists \hat{x}_k \in [x_{k-1}, x_k]\) satisfying
\[f'(\hat{x}_k) = \frac{f(x_k) - f(x_{k-1})}{x_k - x_{k-1}} \]
Letting \(M_k\) be the supremum of \(|f'|\) over \([x_{k-1}, x_k]\), note that
\[
    \begin{aligned}
        U(\abs{f'}, P) - \sum^n_{k=1} \abs{f(x_k) - f(x_{k-1})}
&= \sum^n_{k=1} \left(M_k - \abs{\frac{f(x_k) - f(x_{k-1})}{x_k - x_{k-1}}}\right) (x_k - x_{k-1}) \\
&= \sum^n_{k=1} \left(M_k - \abs{f'(\hat{x}_k)} \right) (x_k - x_{k-1}) \\
&\leq \frac{\epsilon}{b-a} \sum^n_{k-1} x_k-x_{k-1} = \epsilon
    \end{aligned}
\]
Since \(U(|f'|, P) \geq \int^b_a |f'|\), we have \(V f \geq \int^b_a |f'| - \epsilon\), or simply \(V f \geq \int^b_a |f'|\).

It's worth noting that we can't use argument with \(L(|f'|, P)\) for the \(\leq\) inequality because of the \(\sup\) in \(V f\).

}
\end{solution}

\begin{exercise}[Change-of-variable Formula] Let $g:[a, b] \rightarrow \mathbf{R}$ be differentiable and assume $g^{\prime}$ is continuous. Let $f:[c, d] \rightarrow \mathbf{R}$ be continuous, and assume that the range of $g$ is contained in $[c, d]$ so that the composition $f \circ g$ is properly defined.
\enum{
\item Why are we sure $f$ is the derivative of some function? How about $(f \circ g) g^{\prime}$ ?
\item Prove the change-of-variable formula
$$
\int_{a}^{b} f(g(x)) g^{\prime}(x) d x=\int_{g(a)}^{g(b)} f(t) d t .
$$
}
\end{exercise}
\begin{solution}
    \TODO
\end{solution}

\begin{exercise} Assume $f$ is integrable on $[a, b]$ and has a "jump discontinuity" at $c \in(a, b)$. This means that both one-sided limits exist as $x$ approaches $c$ from the left and from the right, but that
$$
\lim _{x \rightarrow c^{-}} f(x) \neq \lim _{x \rightarrow c^{+}} f(x) .
$$
(This phenomenon is discussed in more detail in Section 4.6.)
\enum{
\item Show that, in this case, $F(x)=\int_{a}^{x} f$ is not differentiable at $x=c$.
\item The discussion in Section $5.5$ mentions the existence of a continuous monotone function that fails to be differentiable on a dense subset of $\mathbf{R}$. Combine the results of part (a) with Exercise 6.4.10 to show how to construct such a function.
}
\end{exercise}
\begin{solution}
    \TODO
\end{solution}
