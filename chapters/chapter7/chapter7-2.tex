\section{The Definition of the Riemann Integral}

\begin{exercise}
Let $f$ be a bounded function on $[a, b]$, and let $P$ be an arbitrary partition of $[a, b]$. First, explain why $U(f) \geq L(f, P)$. Now, prove Lemma 7.2.6.
\end{exercise}
\begin{solution}
If \(U(f) < L(f,P)\) then since \(U(f) = \inf\{U(f,P) : P \in \mathcal{P}\}\) there must also be some \(P_1\) with \(U(f, P_1) < L(f,P)\) which contradicts Lemma 7.2.4.

Similarly, if \(U(f) < L(f)\) then there must be some \(P\) where \(U(f) < L(f,P)\) we've just shown to be impossible.
\end{solution}

\begin{exercise}
Consider $f(x)=1 / x$ over the interval $[1,4]$. Let $P$ be the partition consisting of the points $\{1,3 / 2,2,4\}$.
\enum{
\item Compute $L(f, P), U(f, P)$, and $U(f, P)-L(f, P)$.
\item What happens to the value of $U(f, P)-L(f, P)$ when we add the point 3 to the partition?
\item Find a partition $P^{\prime}$ of $[1,4]$ for which $U\left(f, P^{\prime}\right)-L\left(f, P^{\prime}\right)<2 / 5$.
}
\end{exercise}
\begin{solution}
    \TODO
\end{solution}

\begin{exercise}[Sequential Criterion for Integrability]
\enum{
\item Prove that a bounded function $f$ is integrable on $[a, b]$ if and only if there exists a sequence of partitions $\left(P_{n}\right)_{n=1}^{\infty}$ satisfying
$$
\lim _{n \rightarrow \infty}\left[U\left(f, P_{n}\right)-L\left(f, P_{n}\right)\right]=0,
$$
and in this case $\int_{a}^{b} f=\lim _{n \rightarrow \infty} U\left(f, P_{n}\right)=\lim _{n \rightarrow \infty} L\left(f, P_{n}\right)$.
\item For each $n$, let $P_{n}$ be the partition of $[0,1]$ into $n$ equal subintervals. Find formulas for $U\left(f, P_{n}\right)$ and $L\left(f, P_{n}\right)$ if $f(x)=x$. The formula $1+2+3+$ $\cdots+n=n(n+1) / 2$ will be useful.
\item Use the sequential criterion for integrability from (a) to show directly that $f(x)=x$ is integrable on $[0,1]$ and compute $\int_{0}^{1} f$.
}
\end{exercise}
\begin{solution}
\enum{
    \item \((\implies)\) If \(f\) is integrable, then we can choose \(P_{Un}\) to satisfy \(U(f, P_{Un}) - U(f) < 1/n\), \(P_{Ln}\) to satisfy \(L(f) - L(f, P_{Ln}) < 1/n\), and \(P_n\) to be the common refimenemt of \(P_{Un}\) and \(P_{Ln}\); it's easy to show in this case that \(lim_{n \to \infty} U(f, P_n) = U(f) = \int^b_a f = L(f) = \lim_{n \to \infty} L(f, P_n) \).

    \((\impliedby)\) Consider \(\lim_{n\to\infty} U(f,P_n) - L(f) \leq U(f,P_n) - L(f,P_n)\). By the Squeeze Theorem, \(U(f,P_n) - L(f)\)  approaches 0, and therefore \(\lim_{n\to\infty} U(f,P_n) = L(f)\). A similar argument shows that \(\lim_{n\to\infty} L(f,P_n) = U(f)\). Applying the Algebraic Limit Theorem gets us that \(L(f) - U(f) = 0\), or that \(L(f) = U(f) = \int^b_a f\) by definition, with this being equal to \(lim_{n \to \infty} U(f, P_n) = \lim_{n \to \infty} L(f, P_n) \).

    \item Let \([a_{ni}, b_{ni}]\) be the \(i\)'th partition of \(P_n\) (indexing from 0), with \(a_{ni} = i/n\) and \(b_{ni} = (i+1)/n\). It is easy to see that
    \[U(f,P_n) = \sum^{n-1}_{i=0} \frac{b_{ni}}{n} = \sum^{n-1}_{i=0} \frac{i+1}{n^2} = \frac{1}{2} + \frac{1}{2n}\]
    and
    \[L(f,P_n) = \sum^{n-1}_{i=0} \frac{a_{ni}}{n} = \sum^{n-1}_{i=0} \frac{i}{n^2} = \frac{1}{2} - \frac{1}{2n}\]

    \item \(U(f,P_n)\) and \(L(f,P_n)\) both approach \(1/2\) as \(n \to \infty\) and therefore \(\int_0^1 f = 1/2\).
}
\end{solution}

\begin{exercise}
Let $g$ be bounded on $[a, b]$ and assume there exists a partition $P$ with $L(g, P)=U(g, P)$. Describe $g$. Is it integrable? If so, what is the value of $\int_{a}^{b} g$ ?
\end{exercise}
\begin{solution}
    \TODO
\end{solution}

\begin{exercise}
Assume that, for each $n, f_{n}$ is an integrable function on $[a, b]$. If $\left(f_{n}\right) \rightarrow f$ uniformly on $[a, b]$, prove that $f$ is also integrable on this set. (We will see that this conclusion does not necessarily follow if the convergence is pointwise.)
\end{exercise}
\begin{solution}
Since both \(U(f, P_n) \) and \(U(f)\) are defined in terms of supremums (and likewise for \(L\) and infimums), it's useful to have the following lemma: For all \(n \in \mathbf{N}\), let \(A_n\) be a set of real numbers with supremum \(s_n\) and infimum \(i_n\), and let \(B\) be a set with supremum \(s\) and infimum \(i\). If \( \forall \epsilon > 0, \exists N\) so that \(n > N\) implies that \(\forall a \in A_n, \exists b \in B \) such that \(|a-b| < \epsilon\), then \(\lim_{n \to \infty} s_n = s\) and \(\lim_{n \to \infty} i_n = i\).

Proof: Let \(\epsilon > 0\), and choose \(N\) large enough so that for \(n > N\),\(\forall a \in A_n, \exists b \in B \) with \(|a - b| < \epsilon / 2\). Note this implies \(a < b + \epsilon / 2\). Since \(s_n\) is the supremum of \(A_n\) we have
that for some \(a \in A_n\),
\[s_n < a + \epsilon/2  < b + \epsilon < s + \epsilon\]
 A similar argument in reverse shows that \(s < s_n + \epsilon\), or \(|s_n - s| < \epsilon\), as desired. The proof is identical for \(i_n \to i\).

Back to the main proof --- consider a particular interval \([c,d]\) which is part of a partition \(P_m\). We can consider
    \[u_{n,c,d} = (d-c) \cdot \sup\{f_n(x) : x \in [c,d]\}\]
    to be the contribution of the interval \([c,d]\) to \(U(f_n, P_m)\), and similarly define \[u_{c,d} = (d-c) \cdot \sup\{f(x) : x \in [c,d]\}\].
    Because \((f_n) \to f\) uniformly, we can apply our lemma to claim that as \(n \to \infty\), \(u_{n,c,d} \to u_{c,d}\). Since the interval \([c,d]\) is arbitrary, we can apply this to each interval in \(P_m\) to get \(U(f_n,P_m) \to U(f, P_m)\).

    This, plus using our lemma again, implies that \(U(f_n) \to U(f)\). A similar argument shows \(L(f_n) \to L(f)\). But since \(U(f_n) = L(f_n)\), we must have \(U(f) = L(f)\) and therefore \(f\) is integrable on \([a,b]\).
\end{solution}

\begin{exercise}
A tagged partition $\left(P,\left\{c_{k}\right\}\right)$ is one where in addition to a partition $P$ we choose a sampling point $c_{k}$ in each of the subintervals $\left[x_{k-1}, x_{k}\right]$. The corresponding Riemann sum,
$$
R(f, P)=\sum_{k=1}^{n} f\left(c_{k}\right) \Delta x_{k},
$$
is discussed in Section 7.1, where the following definition is alluded to.

{\bf Riemann's Original Definition of the Integral}: A bounded function $f$ is integrable on $[a, b]$ with $\int_{a}^{b} f=A$ if for all $\epsilon>0$ there exists a $\delta>0$ such that for any tagged partition $\left(P,\left\{c_{k}\right\}\right)$ satisfying $\Delta x_{k}<\delta$ for all $k$, it follows that
$$
|R(f, P)-A|<\epsilon .
$$
Show that if $f$ satisfies Riemann's definition above, then $f$ is integrable in the sense of Definition 7.2.7. (The full equivalence of these two characterizations of integrability is proved in Section 8.1.)
\end{exercise}
\begin{solution}
We'll use the Sequential Criterion for Integrability and construct a sequence of \((P_n)\). Let \(\epsilon = 1/n\). By Riemann's definition we can easily form a partition \(P_n\), and tag it to form \(P_u\) with the sampling point close to the supremum, so that
\[|R(f,P_u) - A| < \epsilon/4 \text{ and } |U(f,P_n) - R(f,P_u)| < \epsilon/4\]
 Similarly we can retag the partition to form \(P_l\) with the sampling point close to the infimum, to get
\[|R(f, P_l) - A| < \epsilon/4\text{ and }|L(f,P_n) - R(f, P_l)|<\epsilon/4\]
Applying the Triangle Inequality we get \(U(f,P_n) - L(f,P_n) < \epsilon\) as desired.
\end{solution}

\begin{exercise}
Let $f:[a, b] \rightarrow \mathbf{R}$ be increasing on the set $[a, b]$ (i.e., $f(x) \leq$ $f(y)$ whenever $x<y)$. Show that $f$ is integrable on $[a, b]$.
\end{exercise}
\begin{solution}
Since \(f\) is increasing, let \(P_n\) be comprised of the \(n+1\) evenly spaced points \(\{x_0, \cdots, x_n\}\), with \(\Delta x_k = x_k - x_{k-1} = \Delta_n\) (where \(\Delta_n\) is a constant for a fixed \(n\)). Then
\[U(f,P) = \sum^n_{i=1} f(x_i) (x_i - x_{i-1}) = \Delta_n \sum^n_{i=1}f(x_i)\]
and
\[L(f,P) = \sum^{n-1}_{i=0} f(x_i) (x_{i+1} - x_i) = \Delta_n \sum^{n-1}_{i=0} f(x_i)\]
Therefore \(U(f,P) - L(f,P) = \Delta_n(f(b) - f(a))\). Since \(\Delta_n\) can be made arbitrarily small while \(f(b) - f(a)\) is independent of \(n\), \(U(f) - L(f) = 0\) and therefore \(f\) is integrable.
\end{solution}
