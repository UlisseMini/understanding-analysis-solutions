\section{The Generalized Riemann Integral}

\begin{exercise}
\enum{
\item Explain why both the Riemann sum \(R(f,P)\) and \(\int^b_a f\) fall between \(L(f,P)\) and \(U(f,P)\).
\item Explain why \(U(f,P') - L(f,P') < \epsilon / 3\).
}
\end{exercise}
\begin{solution}
\enum{
\item \(L(f,P) \leq R(f,P) \leq U(f,P)\) is clear from their definitions, as noted earlier in the section's discussion. The definition of \(\int^b_a f\) as the supremum of \(L(f,P)\) over all partitions \(P\) shows \(\int^b_a f \geq L(f,P)\), and similar reasoning gives \(\int^b_a f \leq U(f,P)\).
\item \(P\) is a refinement of \(P_\epsilon\), so from Lemma 7.2.3,
\[U(f, P') - L(f,P') = U(f,P_\epsilon) - L(f, P_\epsilon) < \frac{\epsilon}{3}\]
}
\end{solution}

\begin{exercise}
Explain why \(U(f,P) - U(f,P') \geq 0\).
\end{exercise}
\begin{solution}
\(P'\) is a refinement of \(P\), so by Lemma 7.2.3, \(U(f,P') \leq U(f,P)\).
\end{solution}

\begin{exercise}
\enum{
    \item In terms of \(n\), what is the largest number of terms of the form \(M_k (x_k - x_{k-1})\) that could appear in one of \(U(f,P)\) or \(U(f,P')\) but not the other?
    \item Finish the proof in this direction by arguing that
    \[U(f,P) - U(f,P') < \epsilon/3\]
}
\end{exercise}
\begin{solution}
\enum{
\item In order to transform \(P\) into \(P'\), we add the \(n-1\) points from \(P_\epsilon\) which are not the endpoints \(a\) or \(b\). Each point added can increase the number of non-cancelled terms by at most three (by preventing an interval from \(P\) being cancelled, and by creating two new intervals in \(P'\)). Therefore the maximum number of terms is \(3n - 3\).
\item A triangle inequality gives that \(U(f,P) - U(f,P') \leq \sum M_k (x_k - x_{k-1})\), where \(k\) goes over all of the subintervals in both \(P\) and \(P'\) which weren't cancelled. Since the length of each subinterval in \(P\) and \(P'\) is no more than \(\delta\),
\[\sum M_k (x_k - x_{k-1}) \leq \sum M \delta = (3n-3) \frac{\epsilon}{9n} < \frac{\epsilon}{3} \]
}
\end{solution}
