\section{A Construction of R from Q}
\begin{exercise}
\enum{
\item Fix \(r \in \mathbf{Q}\). Show that the set \(C_r = \{t \in \mathbf{Q} : t < r\}\) is a cut.
The temptation to think of all cuts as being of this form should be avoided. Which of the following subsets of \(\mathbf{Q}\) are cuts?
\item \(S = \{t \in \mathbf{Q}: t \leq 2\}\)
\item \(T = \{t \in \mathbf{Q} : t^2 < 2 \text{ or } t < 0\}\)
\item \(U = \{t \in \mathbf{Q} : t^2 \leq 2 \text{ or } t < 0\}\)
}
\end{exercise}
\begin{solution}
\enum{
\item \(C_r\) contains \(r - 1\) and does not contain \(r\), so (c1) is satisfied. If \(p \in C_r\) and \(q < p\), then \(q < p < r\) and thus \(q \in C_r\), so (c2) is satisfied. Also, \(p < \frac{p + r}{2} < r\) so \(\frac{p+r}{2} \in C_r\) and (c3) is satisfied.
\item Not a cut, \(S\) has the maximum \(p = 2\). There are no elements in \(S\) that can be greater than \(2\).
\item Is a cut. \(0 \in T\) and \(2 \notin T\) so (c1) is satisfied. Let \(r \in T\) and \(q < r\). If \(q < 0\) then \(q \in T\) trivially. Otherwise, \(r > q \geq 0\) implies \(2 > r^2 > q^2\) and therefore \(q \in T\), showing (c2) is satisfied. Finally, to show (c3), let \(r \in T\) with \(r \geq 1\). (If \(r < 1\) then we can trivially identify \(r < 1 \in T\) to confirm (c3).)  Let \(a = 2- r^2\), and note \(1 \geq a > 0\). Consider the rational number
\[q = \left(r + \frac{a}{4r}\right)^2 = r^2 + \frac{a}{2} + \frac{a^2}{4r} > r\]
It is easy to show \(\frac{a^2}{4r} < \frac{a}{2}\), implying \(q < r^2 + a = 2\) and thus \(q \in 2\), and thus \(r\) is not a maximum and (c3) is true.
\item Is a cut. The only difference from part (c) is that we cannot immediately claim by definition that \(a > 0\); instead the definition of \(U\) only implies \(a \geq 0\). However, Section 1.1. provides a proof that \(a \neq 0\); therefore we can maintain \(a > 0\) and reuse the rest of the logic.
}
\end{solution}

\begin{exercise}
Let \(A\) be a cut. Show that if \(r \in A\) and \(s \notin A\), then \(r < s\).
\end{exercise}
\begin{solution}
If \(r \geq s\) then by (c2) \(s\) would be in \(A\), a contradiction.
\end{solution}

\begin{exercise}
Using the usual definitions of addition and multiplication, determine which of these properties are possessed by \(\mathbf{N}\), \(\mathbf{Z}\), and \(\mathbf{Q}\), respectively.
\end{exercise}
\begin{solution}
\(\mathbf{N}\) possesses properties (f1), (f2), and (f5). The natural numbers do not contain 0 or any additive inverses. \(\mathbf{Z}\) additionally satisfies (f3) but still does not satisfy (f4) for multiplicative inverses (e.g. \(2^{-1}\) is not in \(\mathbf{Z}\)). \(\mathbf{Q}\) possesses all properties; in particular if \(q \in \mathbf{Q}\) then so are \(-q\) and \(1/q\).
\end{solution}

\begin{exercise}
Show that this defines an ordering on \(\mathbf{R}\) by verifying properties (o1), (o2), and (o3) from Definition 8.6.5.
\end{exercise}
\begin{solution}
To prove property (o1), assume \(A \nsubseteq B\). By definition, this means there is some  \(a \in A\) with \(a \notin B\). From Exercise 8.6.2 this means \(\forall b \in B\), \(b < a\). Then by property (c2) \(b \in A\); hence \(B \subseteq A\).

(o2) is a direct result of the definition of set equality, and (o3) is true because of transivity of the set inclusion relationship.
\end{solution}

\begin{exercise}
\enum{
\item Show that (c1) and (c3) also hold for \(A+B\). Conclude that \(A+B\) is a cut.
\item Check that addition in \(\mathbf{R}\) is commutative (f1) and associative (f2).
\item Show that property (o4) holds.
\item Show that the cut
\[O = \{p \in \mathbf{Q} : p < 0\}\]
successfully plays the role of the additive identity (f3). (Showing \(A+O = A\) amounts to proving that these two sets are the same. The standard way to prove such a thing is to show two inclusions: \(A + O \subseteq A\) and \(A \subseteq A + O\).)
}
\end{exercise}
\begin{solution}
\enum{
\item For (c1), we can find \(a \in A\), \(a' \notin A\), \(b \in B\), and \(b' \notin B\). Then \(a + b \in A + B\) so \(A + B \neq \emptyset\). Also, since \(a < a'\) and \(b < b'\), \(a + b \neq a' + b'\) and therefore \(a' + b' \notin A + B\).

For (c3), let arbitrary \(a + b \in A + B\) with \(a \in A\) and \(b \in B\), and let \(a < a' \in A\) and \(b < b' \in B\). Then \(a + b < a' + b' \in A + B\).
\item Let arbitrary \(a + b \in A + B\); then \(b + a \in B + A\); hence (f1) holds for addition. Let arbitrary \((a + b) + c \in (A+B)+C \), then \(a + (b+c) \in A + (B+C)\); hence (f2) holds for addition.
\item Let \(Y \subseteq Z\), and let \(x + y \in X + Y\) with \(x \in X\) and \(y \in Y\). Then \(y \in Z\) so \(x + y \in X + Z\), implying \(X + Y \subseteq X + Z\).
\item Let \(a + o \in A + O\) where \(a \in A\) and \(o \in O\). Then \(a + o < a\) implying \(a + o \in A\) so \(A + O \subseteq A\). Now let arbitrary \(a \in A\), and find \(a < a' \in A\). Define \(s = a - a' < 0\), so \(s \in O\). Then \(a = a' + s \in A + O\), so \(A \subseteq A + O\).
}
\end{solution}

\begin{solution}
\enum{
\item Prove that \(-A\) defines a cut.
\item What goes wrong if we set \(-A = \{r \in \mathbf{Q} : -r \notin A\}\)?
\item If \(a \in A\) and \(r \in -A\), show \(a + r \in O\). This shoes \(A + (-A) \subseteq O\). Now, finish the proof of property (f4) for addition in Definition 8.6.4.
}
\end{solution}
\begin{solution}
\enum{
\item Let \(t \neq A\) and \(r < -t\). Then \(-r > t\) so \(r \in -A\) and \(-A \neq \emptyset\). Let \(a \in A\). Then \(\forall t \neq A\), \(t > a\); therefore \(-a \notin -A\) and \(A \neq \mathbf{Q}\), proving (c1).
To prove (c2), let arbitrary \(r \in -A\) and let \(q < r\). Then \(-q > -r > t\) for some \(t \notin A\); hence \(q \in -A\).
To prove (c3), let \(r \in -A\), and let \(t \notin A\) satisfy \(t < -r\). Let \(q = (t - r) / 2\), and note that \(t < q < -r\), so \(-q \in -A\). Moreover \(-q > r\), so (c3) is proved.

\item This would fail on the cut \(A = \{t \in \mathbf{Q} : t < -2\}\), since then \(-A = \{t \in \mathbf{Q} : t \leq 2\}\) which from Exercise 8.6.1 (b) is not a cut.
\item Since \(-r \in -A\), then we have some \(t \notin A\) where \(-r > t > a\). Then \(0 > a + r\) implying \(a + r \in O\). Now suppose \(o \in O\); we would like to find \(a \in A\) and \(r \in -A\) satisfying \(o \leq a + r\); this would imply \(O \subseteq A + (-A)\) and thus \(O = A + (-A)\). Since \(o \in \mathbf{Q}\) and \(o < 0\), let \(o = -p/q\) where \(p, q \in \mathbf{N}\). We can prove the following lemma: for any cut \(A\) and \(n \in \mathbf{N}\), we can find \(z \in \mathbf{Z}\) where
\[\frac{z}{n} \in A \text{\quad and \quad} \frac{z + 1}{n} \notin A\]
To do this, start with \(a \in A\) and \(a' \notin A\), and find \(N, M \in \mathbf{Z}\) satisfying
\[\frac{N}{n} < a \text{\quad and \quad} \frac{M}{n} > a'\]
Clearly \(\frac{N}{n} \in A\) and \(\frac{M}{n} \notin A\). If we count upwards from \(i = N\), at some point \(\frac{i}{n}\) is going to transition from being in \(A\) to not being in \(A\), so that by the time we reach \(i = M\) we have \(\frac{i}{n} \notin A\). This transition point implies the existence of \(z\).

Now, let \(a = \frac{n}{2q} \in A\) with \(\frac{n+1}{2q} \notin A\). Then \(r = -\frac{n+2}{2q} \notin A\), and
\[a + r = \frac{-2}{2q} = -\frac{1}{q} \geq \frac{-p}{q} = o\]
completing the proof.
}
\end{solution}
