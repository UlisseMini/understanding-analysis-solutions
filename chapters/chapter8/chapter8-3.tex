\section{Euler's Sum}
\begin{exercise}
Supply the details to show that when \(x = \pi/2\) the product formula in (2) is equivalent to
\[\frac{\pi}{2} = \lim_{n \to \infty} \left(\frac{2 \cdot 2}{1 \cdot 3}\right) \left(\frac{4 \cdot 4}{3 \cdot 5}\right) \left(\frac{6 \cdot 6}{5 \cdot 5}\right) \cdots \left(\frac{2n \cdot 2n}{(2n-1)(2n+1)}\right),\]
where the infinite product in (2) is interpreted to be a limit of partial products.
\end{exercise}
\begin{solution}
Plugging \(x = \pi/2\) into (2),
\[
1 = \frac{\pi}{2} \prod^\infty_{i=1} \left(1 - \frac{1}{2i}\right)\left(1 + \frac{1}{2i}\right)\]
\[
\frac{2}{\pi} = \prod^\infty_{i=1} \frac{(2i-1)(2i + 1)}{4i^2}
\]
Taking the reciprocal of both sides leads us with the desired equality.
\end{solution}

\begin{exercise}
Assume $h(x)$ and $k(x)$ have continuous derivatives on $[a, b]$ and derive the integration-by-parts formula
$$
\int_{a}^{b} h(t) k^{\prime}(t) dt=h(b) k(b)-h(a) k(a)-\int_{a}^{b} h^{\prime}(t) k(t) dt
$$
\end{exercise}
\begin{solution}
See Exercise 7.5.6 (a).
\end{solution}

\begin{exercise}
\enum{
\item Using the simple identity \(sin^n(x) = \sin^{n-1}(x) \sin(x)\) and the previous exercise, derive the recurrence relation
\[b_n = \frac{n-1}{n}b_{n-2}\]
for all \(n \geq 2\).
\item Use this relation to generate the first three even terms and the first three odd terms of the sequence \((b_n)\).
\item Write a general expression for \(b_{2n}\) and \(b_{2n+1}\).
}
\end{exercise}
\begin{solution}
\enum{
\item Apply integration-by-parts with \(h = \sin^{n-1}(x)\) and \(k = -\cos(x)\):
\[ \begin{aligned}
    \int^\frac{\pi}{2}_0 \sin^{n-1}(x) \sin(x) &= \left(\sin^{n-1}\frac{\pi}{2}\right)\left(-\cos \frac{\pi}{2}\right) - \left(\sin^{n-1} 0\right)\left(\cos 0\right) + \int^\frac{\pi}{2}_0 (n-1)\sin^{n-2}(x)\cos^2(x) \\
    b_n &=(n-1)\int^\frac{\pi}{2}_0 \sin^{n-2}(x) \left(1 - \sin^2(x)\right) = (n-1) b_{n-2} - (n-1)b_n \\
    n b_n &= (n-1) b_{n-2} \\
    b_n &= \frac{n-1}{n}b_{n-2}
\end{aligned}\]
\item Evens: \(b_2 = \frac{1}{4} \pi\), \(b_4 = \frac{3}{16}\pi\), \(b_6 = \frac{5}{32} \pi\). Odds: \(b_1 = 1\), \(b_3 = \frac{2}{3}\), \(b_5 = \frac{8}{15}\)
\item For \(n \geq 1\),
\[b_{2n} = \frac{\pi}{2}\prod^n_{i=1} \frac{2i - 1}{2i} \text{ and } b_{2n+1} = \prod^n_{i=1} \frac{2i}{2i + 1}\]
}
\end{solution}
