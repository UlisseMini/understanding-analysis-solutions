\section{Euler's Sum}
\begin{exercise}
Supply the details to show that when \(x = \pi/2\) the product formula in (2) is equivalent to
\[\frac{\pi}{2} = \lim_{n \to \infty} \left(\frac{2 \cdot 2}{1 \cdot 3}\right) \left(\frac{4 \cdot 4}{3 \cdot 5}\right) \left(\frac{6 \cdot 6}{5 \cdot 5}\right) \cdots \left(\frac{2n \cdot 2n}{(2n-1)(2n+1)}\right),\]
where the infinite product in (2) is interpreted to be a limit of partial products.
\end{exercise}
\begin{solution}
Plugging \(x = \pi/2\) into (2),
\[
1 = \frac{\pi}{2} \prod^\infty_{i=1} \left(1 - \frac{1}{2i}\right)\left(1 + \frac{1}{2i}\right)\]
\[
\frac{2}{\pi} = \prod^\infty_{i=1} \frac{(2i-1)(2i + 1)}{4i^2}
\]
Taking the reciprocal of both sides leads us with the desired equality.
\end{solution}

\begin{exercise}
Assume $h(x)$ and $k(x)$ have continuous derivatives on $[a, b]$ and derive the integration-by-parts formula
$$
\int_{a}^{b} h(t) k^{\prime}(t) dt=h(b) k(b)-h(a) k(a)-\int_{a}^{b} h^{\prime}(t) k(t) dt
$$
\end{exercise}
\begin{solution}
See Exercise 7.5.6 (a).
\end{solution}

\begin{exercise}
\enum{
\item Using the simple identity \(sin^n(x) = \sin^{n-1}(x) \sin(x)\) and the previous exercise, derive the recurrence relation
\[b_n = \frac{n-1}{n}b_{n-2}\]
for all \(n \geq 2\).
\item Use this relation to generate the first three even terms and the first three odd terms of the sequence \((b_n)\).
\item Write a general expression for \(b_{2n}\) and \(b_{2n+1}\).
}
\end{exercise}
\begin{solution}
\enum{
\item Apply integration-by-parts with \(h = \sin^{n-1}(x)\) and \(k = -\cos(x)\):
\[ \begin{aligned}
    \int^\frac{\pi}{2}_0 \sin^{n-1}(x) \sin(x) &= \left(\sin^{n-1}\frac{\pi}{2}\right)\left(-\cos \frac{\pi}{2}\right) - \left(\sin^{n-1} 0\right)\left(\cos 0\right) \\
        &+ \int^\frac{\pi}{2}_0 (n-1)\sin^{n-2}(x)\cos^2(x) \\
    b_n &=(n-1)\int^\frac{\pi}{2}_0 \sin^{n-2}(x) \left(1 - \sin^2(x)\right) = (n-1) b_{n-2} - (n-1)b_n \\
    n b_n &= (n-1) b_{n-2} \\
    b_n &= \frac{n-1}{n}b_{n-2}
\end{aligned}\]
\item Evens: \(b_2 = \frac{1}{4} \pi\), \(b_4 = \frac{3}{16}\pi\), \(b_6 = \frac{5}{32} \pi\). Odds: \(b_1 = 1\), \(b_3 = \frac{2}{3}\), \(b_5 = \frac{8}{15}\)
\item For \(n \geq 1\),
\[b_{2n} = \frac{\pi}{2}\prod^n_{i=1} \frac{2i - 1}{2i} \text{ and } b_{2n+1} = \prod^n_{i=1} \frac{2i}{2i + 1}\]
}
\end{solution}

\begin{exercise}
Show
\[\lim_{n \to \infty} \frac{b_{2n}}{b_{2n+1}} = 1,\]
and use this fact to finish Wallis's product formula in (3).
\end{exercise}
\begin{solution}
    Equivalently, we wish to evaluate
\[\lim_{n \to \infty} \frac{b_n}{b_{n+1}} = \lim_{n \to \infty} \frac{\int^{\pi/2}_0 \sin^{n+1} (t) dt}{\int^{\pi/2}_0 \sin^n (t) dt}\]
Note that since \(b_n > b_{n+1}\), \(\frac{b_n}{b_{n+1}} \geq 1\).
We prove the following lemma: for \(\frac{\pi}{2} > b > a > 0\),
\[\lim_{n \to \infty} \frac{\int^{\pi/2}_b \sin^n x}{\int^a_0 \sin^n x} = \infty\]

To see this, let \(y_n\) be the value in the limit. Note that \(\int^{\pi/2}_b \sin^{n+1}x \geq \sin b \int^{\pi/2}_b \sin^n x\) and similarly \(\int_0^a \sin^{n+1}x \leq \sin a \int_0^a \sin^n x\), and note
\[\frac{y_{n+1}}{y_n} = \left(\frac{\int^{\pi/2}_b \sin^{n+1} x}{\int^{\pi/2}_b \sin^{n} x}\right) \left(\frac{\int_0^a \sin^{n} x}{\int_0^a \sin^{n+1} x}\right) \geq \frac{\sin b}{\sin a}> 1\]
In other words, \((y_n)\) grows at least exponentially, and therefore must not be bounded.

We now show that for arbitrary \(\epsilon > 0\), we can find \(N\) large enough so that for \(n \geq N\),
\[\int^\frac{\pi}{2}_0 \sin^{n+1} x \geq (1-\epsilon) \int_0^\frac{\pi}{2} \sin^n x\]
Note that, for any \(a \in (0, \pi / 2)\),
\[\int^\frac{\pi}{2}_0 \sin^{n+1} x = \int_0^a \sin^{n+1} x + \int_a^\frac{\pi}{2} \sin^{n+1} x \geq \sin a \int_a^\frac{\pi}{2} \sin^n x\]
and for any \(b \in (a, \pi/2)\),
\[\int_0^\frac{\pi}{2} \sin^n x = \int_0^a \sin^n x + \int_a^b \sin^n x + \int_b^\frac{\pi}{2}\sin^n x\]
Find \(a \in (0, \pi/2)\) so that \(\sin a > \sqrt{1-\epsilon}\), and fix \(b = \left(a + \frac{\pi}{2}\right)/2\). From our earlier lemma, we can find \(N\) large enough that \(n > N\) ensures
\[\int_0^a \sin^n x < \epsilon_1 \int_b^\frac{\pi}{2}\sin^n x\]
where we'll choose \(\epsilon_1\) to satisfy \(\frac{1}{1 + \epsilon_1} = \sqrt{1 - \epsilon}\). This ensures that
\[\int^\frac{\pi}{2}_a \sin^n x > \sqrt{1-\epsilon} \int_0^\frac{\pi}{2} \sin^n x\]
and
\[\int^\frac{\pi}{2}_0 \sin^{n+1} x \geq (1-\epsilon) \int_0^\frac{\pi}{2} \sin^n x.\]
We can convert this to indicate that for large enough \(n\), \(b_n / b_{n+1} \leq 1 + \epsilon\), which together with our earlier observation that \(b_n > b_{n+1}\) lets us conclude
\[\lim_{n \to \infty} \frac{b_n}{b_{n+1}} = 1\]
\end{solution}
