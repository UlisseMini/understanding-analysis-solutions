\section{Inventing the Factorial Function}
\begin{exercise}
For \(n \in N\), let
\[N\# = n + (n-1) + (n-1) + \cdots + 2 + 1\]
\enum{
\item Without looking ahead, decide if there is a natural way to define \(0\#\). How about \((-2)\#\)? Conjecture a reasonable value for \(\frac{7}{2}\#\).
\item Now prove \(n\# = \frac{1}{2}n (n+1)\) for all \(n \in \mathbf{N}\), and revisit part (a).
}
\end{exercise}
\begin{solution}
\enum{
\item Noting that \(N\# = (N-1)\# + N\) and \(1\# = 1\), we could have \(0\# = 0\), \(-1\# = 0\), and \(-2\# = 1\). \(7/2\#\) could just be defined to be the result from linearly interpolating between \(3\# = 6\) and \(4\# = 10\) to get \(8\).
\item This is obviously true for \(n = 1\), and
\[(n+1)\# = n+1+n\# = n+1 + \frac{1}{2}n(n+1) = (n+1)\left(1 + \frac{n}{2}\right) = (n+1)\left(\frac{n+2}{2}\right)\]
which proves the formula by induction.
}
\end{solution}
\begin{exercise}
Verify that the series converges absolutely for all \(x \in \mathbf{R}\), that \(E(x)\) is differentiable on \(\mathbf{R}\), and \(E'(x) = E(x)\).
\end{exercise}
\begin{solution}
Note that
\[\sum^\infty_{n=0} \abs{\frac{x^n}{n!}} = E(\abs{x})\]
so we only need to show the series converges for \(x \geq 0\).

Fix \(x \geq 0\) and let \(N > x\) for \(N \in \mathbf{N}\). We have
\[E(x) = \sum^{2N}_{n=0} \frac{x^n}{n!} + \sum_{n=2N+1}^\infty \frac{x^n}{n!} \leq K + \sum_{n=2N+1}^\infty \frac{N^n}{N!(N+1)^n} = K + \frac{1}{N!}\sum_{n=2N+1}^\infty \left(\frac{N}{N+1}\right)^n\]
for some finite constant \(K\). The infinite series left over is a geometric series which converges.

Term-by-term differentiation is safe to apply on power series which converge (Theorem 6.5.6), and it's clear that \(E'(x) = E(x)\) when applying termwise differentiation.
\end{solution}

\begin{exercise}
\enum{
\item Use the results of Exercise 2.8.7 and the binomial formula to show that \(E(x+y) = E(x)E(y)\) for all \(x,y\in \mathbf{R}\).
\item Show that \(E(0) = 1\) , \(E(-x) = 1/E(x)\), and \(E(x) > 0\) for all \(x \in \mathbf{R}\).
}
\end{exercise}
\begin{solution}
\enum{
\item
\[\begin{aligned}
    E(x+y) &= \sum^\infty_{n=0} \frac{(x+y)^n}{n!} = \sum^\infty_{n=0} \frac{1}{n!} \sum^n_{i=0} \frac{n!}{i!(n-i)!} x^i y^{n-i} = \sum^\infty_{n=0} \sum^n_{i=0} \left(\frac{x^i}{i!}\right) \left(\frac{y^{n-i}}{(n-i)!}\right) \\
    &= \sum^\infty_{i=0} \sum^\infty_{j=0} \left(\frac{x^i}{i!}\right) \left(\frac{y^j}{j!}\right) = \left(\sum^\infty_{i=0} \frac{x^i}{i!}\right)\left(\sum^\infty_{i=0} \frac{y^i}{i!}\right) = E(x) E(y)
\end{aligned}
    \]
\item For \(E(0)\), all terms for \(n \geq 1\) become 0, so \(E(0) = 1\). We showed somewhat informally in Exercise 6.6.5(c) that \(E(-x) E(x) = 1\) by collecting common terms. However, Exercise 2.8.7 lets us conclude that \(E(-x)E(x)\) does in fact equal \(\sum^\infty_{n=0} d_n\) where \(d_n\) is defined as in Exercise 6.6.5(c). Finally, it's clear that \(E(x) > 0\)  for \(x \geq 0\) simply because all terms are positive; then \(E(-x) = 1/E(x) > 0\) and so \(E(x) > 0\) for \(x < 0\) as well.
}
\end{solution}

\begin{exercise}
Define \(e = E(1)\). Show \(E(n) = e^n\) and \(E(m/n) = \left(\sqrt[n]{e}\right)^m\) for all \(m,n \in \mathbf{Z}\).
\end{exercise}
\begin{solution}
We have for \(n \geq 1\) that
\[E(n) = E\left(\sum^n_{i=1} 1\right) = \prod^n_{i=1} E(1) = e^n \]
We also have for \(n=0\), \(E(0) = 1 = e^0\), by the standard definition of \(a^n\) for any \(a \in \mathbf{R}\). Finally for \(n > 0\), \(e^{-n} = 1/e^n = 1/E(n) = E(-n)\), so \(e^n = E(n)\) for all \(n \in \mathbf{Z}\).

By definition \(\sqrt[n]{e}\) is the unique positive number which satisfies \((\sqrt[n]{e})^n = e\). \(E(1/n)\) satisfies this equality, since
\[E(1) = E\left(\sum^n_{i=1} \frac{1}{n}\right)= \prod^n_{i=1} E(1/n) \]
so \(\sqrt[n]{e} = E(1/n)\). Finally
\[E(m/n) = E\left(\sum^m_{i=1} \frac{1}{n}\right) = prod^m_{i=1}E(1/n) = \left(\sqrt[n]{e}\right)^m\]
\end{solution}

\begin{exercise}
Show \(\lim_{x \to \infty} x^n e^{-x} = 0\) for all \(n = 0,1, 2, \dots\).
\end{exercise}
\begin{solution}
Note that for a fixed \(n\), \(\forall K\in \mathbf{R}\),we can find \(N > 0\) so that whenever \(x \geq N\), \(e^x x^{-n} > K\). We do this by noting
\[\frac{E(x)}{x^n} > \frac{x^{n+1}}{(n+1)! x^n} = \frac{x}{(n+1)!}\]
and setting \(N = K(n+1)! \).

Now, let \(\epsilon > 0\), and find \(M\) so that \(x \geq M\) implies \(e^x x^{-n} > 1/\epsilon\). Then
\[x^n e^{-x} = \frac{1}{x^{-n}e^x} < \epsilon\]
as desired.
\end{solution}

