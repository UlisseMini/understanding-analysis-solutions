\section{Metric Spaces and the Baire Category Theorem}
\begin{exercise}
Decide which of the following are metrics on \(X = \mathbf{R}^2\). For each, we let \(x = (x_1,x_2)\) and \(y = y_1, y_2\) be points in the plane.
\enum{
\item \(d(x,y) \sqrt{(x_1 - y_1)^2 + (x_2 - y_2)^2}\)
\item \(d(x,y) = \max\{|x_1 - y_1|, |x_2 - y_2|\}\)
\item \(d(x,y) = |x_1x_2 + y_1y_2|\)
}
\end{exercise}
\begin{solution}
\enum{
\item This is just the Euclidean distance between \(x\) and \(y\). The first two properties are obvious, while the third can be demonstrated with a little geometry.
\item It's fairly easy to see that the first two properties are met. To demonstrate the triangle inequality, note
\[d(x, z) + d(z, y) \geq \abs{x_1 - z_1} + \abs{z_1 - y_1} \geq \abs{x_1 - y_1}\]
and
\[d(x, z) + d(z, y) \geq \abs{x_2 - z_2} + \abs{z_2 - y_2} \geq \abs{x_2 - y_2}\]
therefore \(d(x,z) + d(z,y)\) is greater than or equal to both possible values of \(d(x,y)\).
\item Property (i) is not met; take \(x = (0,1)\) and \(y = (1,0)\).
}
\end{solution}

\begin{exercise}
Let \(C[0,1]\) be the collection of continuous functions on the closed interval \([0,1]\). Decide which of the following are metrics on \(C[0,1]\).
\enum{
\item \(d(f,g) = \sup{\abs{f(x) - g(x)} : x \in [0,1]}\)
\item \(d(f,g) = \abs{f(1) - g(1)}\)
\item \(d(f,g) = \int^1_0 \abs{f-g}\)
}
\end{exercise}
\begin{solution}
\enum{
\item The first two properties are trivial. For the triangle inequality:
\[\sup |f-g| \leq \sup (\abs{f-h} + \abs{h-g}) \leq \sup |f-h| + \sup |h-g|\]
\item The first property fails, e.g. \(f(x) = 1\), \(g(x) = x\)
\item Clearly \(\int^1_0 |f-g| \geq \int^1_0 0 = 0\), and if \(f = g\) then \(\int^1_0 |f-g| = 0\). If \(\int^1_0 |f-g| = 0\), let \(F(x) = \int^x_0 \abs{f - g}\). By the Fundamental Theorem of Calculus, and noting that \(\abs{f-g}\) is continuous,
\[F'(x) = |f-g| = 0\]
implying that \(f = g\). This indicates the first property is met. The second property is trivially true. The third property follows from the triangle inequality on absolute values.
}
\end{solution}

\begin{exercise}
Verify that the discrete metric is actually a metric.
\end{exercise}
\begin{solution}
Property (i) and (ii) are trivial. For property (iii), there are two cases - either \(x = y\) or \(x \neq y\). If \(x = y\) then certainly \(\rho(x,z) + \rho(z,y) \geq 0\) by property (i). If \(x \neq y\) then at least one of \(z \neq y\) and \(z \neq x\) is true, so
\[\rho(x,z) + \rho(z, y) \geq 1 = \rho(x,y)\]
\end{solution}

\begin{exercise}
Show that a convergent sequence is Cauchy.
\end{exercise}
\begin{solution}
Given \((x_n) \to x\), choose \(N\) large enough that for \(n \geq N\), \(d(x_n, x) < \epsilon/2\). Then for \(m,n \geq N\), \(d(x_m - x_n) \leq d(x_m - x) + d(x, x_n) < \epsilon\)
\end{solution}

\begin{exercise}
\enum{
\item Consider \(\mathbf{R}^2\) with the discrete metric \(\rho(x,y)\) examined in Exercise 8.2.3. What do Cauchy sequences look like in this space? Is \(\mathbf{R}^2\) complete with respect to this metric?
\item Show that \(C[0,1]\) is complete with respect to the metric in Exercise 8.2.2 (a).
\item Define \(C^1[0,1]\) to be the collection of differentiable functions on \([0,1]\) whose derivatives are also continuous. Is \(C^1[0,1]\) complete with respect to the metric defined in Exercise 8.2.2 (a)?
}
\end{exercise}
\begin{solution}
\enum{
\item In order to reach any \(\epsilon < 1\), after a certain point in any Cauchy sequence, all elements need to be identical, with the sequence converging to this identical value. Therefore \(\mathbf{R}^2\) (and any set) is complete with respect to the discrete metric.
\item By Theorem 6.2.5 (Cauchy Criterion for Uniform Convergence), any \((f_n)\) which is a Cauchy sequence, uniformly converges to some function \(f\). Since uniform convergence preserves continuity, \(f\) is also continuous and thus in \(C[0,1]\).
\item Recall that Theorem 6.2.5 is an if-and-only-if statement, so any uniformly convergent sequence of functions is a Cauchy sequence. With that in mind, Exercise 6.3.2 is an example of a sequence of functions which converges uniformly to a function which is not differentiable. Thus \(C^1[0,1]\) is not complete with respect to this metric.
}
\end{solution}
