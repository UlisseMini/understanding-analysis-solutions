\section{The Weierstrauss Approximation Theorem}

\begin{exercise}
Assuming WAT, show that if \(f\) is continuous on \([a, b]\), then there exists a sequence \((p_n)\) of polynomials such that \(p_n \to f\) uniformly on \([a, b]\).
\end{exercise}
\begin{solution}
Repeatedly apply WAT with \(\epsilon = 1/n\).
\end{solution}

\begin{exercise}
Prove Theorem 6.7.3.
\end{exercise}
\begin{solution}
Recall Theorem 4.4.7, which states that a continuous functions over a compact set is uniformly continuous over that set. Given \(\epsilon > 0\), apply uniform continuity on \(f\) with \(\epsilon/2\) to obtain some \(\delta > 0\), and partition \([a,b]\) into uniform segments, with each segment length lower than \(\delta\). Define \(\phi(x)\) at the endpoints of each segment to be equal to \(f(x)\), and to linearly interpolate between segment endpoints.

For any \(x \in (a, b)\), let \(q\) be the largest segment endpoint less than \(x\), and \(r\) be the following segment endpoint. (If \(x = a\) or \(x = b\) then these aren't necessarilly defined, but then \(\phi(x) = f(x)\) so there's nothing to worry about.) Since \(\abs{x-q} < \delta\) we have that \(\abs{f(x) - \phi(q)} < \epsilon/2\). We similarly also have \(\abs{\phi(q) - \phi(r)} < \epsilon/2\). Also, note that \(\phi(x)\) must lie between \(\phi(q)\) and \(\phi(r)\), so \(\abs{\phi(q) - \phi(x)} \leq \abs{\phi(q) - \phi(r)} < \epsilon/2\). Applying the triangle inequality leaves us with \(\abs{f(x) - \phi(x)}< \epsilon\) as desired.
\end{solution}

\begin{exercise}
\enum{
    \item Find the second degree polynomial $p(x)=q_{0}+q_{1} x+q_{2} x^{2}$ that interpolates the three points $(-1,1),(0,0)$, and $(1,1)$ on the graph of $g(x)=|x|$. Sketch $g(x)$ and $p(x)$ over $[-1,1]$ on the same set of axes.
    \item Find the fourth degree polynomial that interpolates $g(x)=|x|$ at the points $x=-1,-1 / 2,0,1 / 2$, and 1 . Add a sketch of this polynomial to the graph from (a).
}
\end{exercise}
\begin{solution}
\enum{
    \item \(p(x) = x^2\)
    \item \(p(x) = \frac{7}{3}x^2 - \frac{4}{3}x^4\)
}
\end{solution}

\begin{exercise}
    Show that $f(x)=\sqrt{1-x}$ has Taylor series coefficients $a_{n}$ where $a_{0}=1$ and
$$
a_{n}=\frac{-1 \cdot 3 \cdot 5 \cdots(2 n-3)}{2 \cdot 4 \cdot 6 \cdots 2 n}
$$
for $n \geq 1$.
\end{exercise}
\begin{solution}
For \(n \geq 1\):
\[f^{(n)} = -\frac{\prod^{N-1}_{i=1} (2i - 1)}{2^n} (1-x)^{-\frac{2n-1}{2}}\]
\[a_n = \frac{f^{(n)}(0)}{n!}
= -\frac{ \prod^{N-1}_{i=1} (2i - 1) }{ \left(\prod^{N}_{i=1}2\right) \left(\prod^N_{i=1}i\right) }
=\frac{-1 \cdot 3 \cdot 5 \cdots(2 n-3)}{2 \cdot 4 \cdot 6 \cdots 2 n}
\]

\end{solution}

\begin{exercise}
    \enum{
    \item Follow the advice in Exercise 6.6.9 to prove the Cauchy form of the remainder:
$$
E_{N}(x)=\frac{f^{(N+1)}(c)}{N !}(x-c)^{N} x
$$
for some $c$ between 0 and $x$.
\item Use this result to prove equation (1) is valid for all $x \in(-1,1)$.

    }
\end{exercise}
\begin{solution}
\enum{
    \item See solution to Exercise 6.6.9
    \item \[E_N(x) = \frac{-x (1-x)^{-\frac{1}{2}} }{2}  \left(\prod^N_{i=1} \frac{2i-1}{2i} \right)  \left(\frac{x-c}{1-c}\right)^N\]
    For \(0 < c < x < 1\), the first term is constant, the second term is less than 1, and the last term converges to 0.
}
\end{solution}

\begin{exercise}
    \enum{
\item Let
$$
c_{n}=\frac{1 \cdot 3 \cdot 5 \cdots(2 n-1)}{2 \cdot 4 \cdot 6 \cdots 2 n}
$$
for $n \geq 1$. Show $c_{n}<\frac{2}{\sqrt{2 n+1}}$.
\item Use (a) to show that $\sum_{n=0}^{\infty} a_{n}$ converges (absolutely, in fact) where $a_{n}$ is the sequence of Taylor coefficients generated in Exercise 6.7.4.
\item Carefully explain how this verifies that equation (1) holds for all $x \in$ $[-1,1]$
    }
\end{exercise}
\begin{solution}
\enum{
    \item We can show this by induction, if a bit inelegantly. The base case is trivial matter of computation. For the inductive case, we want to show \(c_{n+1} = c_n \frac{2n+1}{2n+2} \leq \frac{2}{\sqrt{2n+1}} \sqrt{\frac{2n+1}{2n+3}}\). If we work through the algebra of the claim \(\frac{2n+1}{2n+1} < \sqrt{\frac{2n+1}{2n+3}}\), we find that this is equivalent to
    \[8n^3 + 20n^2 + 14n + 3 < 8n^3 + 20n^2 + 16n + 4\]
which is clearly true for \(n \geq 1\), and the inductive step is done.
    \item \(\abs{a_n} = c_n / (2n-1) < \frac{2}{(2n-1)\sqrt{2n+1}} \leq 2 \cdot (2n-1)^{2/3}\) which implies absolute convergence by comparison against an appropriate geometric series.
    \item (b) implies that the Taylor series of \(\sqrt{1-x}\) converges absolutely at 1. With Theorem 6.5.2, the Taylor series converges uniformly over \([-1,1]\) and is therefore continuous. We also have that the Taylor series converges to \(\sqrt{1-x}\) for \(x \in (-1, 1)\), which is also continuous. Therefore taking limits as both functions approach \(\pm 1\) gets us that they are equal over \([-1,1]\).
}
\end{solution}

\begin{exercise}
    \enum{
    \item Use the fact that $|a|=\sqrt{a^{2}}$ to prove that, given $\epsilon>0$, there exists a polynomial $q(x)$ satisfying
$$
|| x|-q(x)|<\epsilon
$$
for all $x \in[-1,1]$.
\item Generalize this conclusion to an arbitrary interval $[a, b]$.
    }
\end{exercise}
\begin{solution}
\enum{
    \item Let the polynomial \(p(x)\) be the partial sum of the Taylor series of \(\sqrt{1-x}\) which satisfies \(\abs{p(x) - \sqrt{1-x}} < \epsilon\), and let \(q(x) = p(1-x^2)\). We then have
        \[\abs{q(x) - \sqrt{1-(1-x^2)}} = \abs{|x| - q(x)} < \epsilon\] as desired.
    \item Let \(c = \max\{|a|,|b|\}\), and let \(p(x)\) satisfy \(\abs{|x| - p(x)} < \epsilon / c \). Then
    \[\abs{\abs{\frac{x}{c}} - p\left(\frac{x}{c}\right)} < \frac{\epsilon}{c}\]
    \[\abs{|x| - c \cdot p\left(\frac{x}{c}\right)} < \epsilon\]
     for \(x \in [-c,c] \supseteq [a,b]\), so we can use the polynomial \(c \cdot p(\frac{x}{c})\).
}
\end{solution}

\begin{exercise}
    \enum{
    \item Fix $a \in[-1,1]$ and sketch
$$
h_{a}(x)=\frac{1}{2}(|x-a|+(x-a))
$$
over $[-1,1]$. Note that $h_{a}$ is polygonal and satisfies $h_{a}(x)=0$ for all $x \in[-1, a]$.
\item Explain why we know $h_{a}(x)$ can be uniformly approximated with a polynomial on $[-1,1]$.
\item Let $\phi$ be a polygonal function that is linear on each subinterval of the partition
$$
-1=a_{0}<a_{1}<a_{2}<\cdots<a_{n}=1 .
$$
Show there exist constants $b_{0}, b_{1}, \ldots, b_{n-1}$ so that
$$
\phi(x)=\phi(-1)+b_{0} h_{a_{0}}(x)+b_{1} h_{a_{1}}(x)+\cdots+b_{n-1} h_{a_{n-1}}(x)
$$
for all $x \in[-1,1]$.
\item Complete the proof of WAT for the interval $[-1,1]$, and then generalize to an arbitrary interval $[a, b]$.
    }
\end{exercise}
\begin{solution}
\enum{
    \item Left as an application for your favourite graphing calculator
    \item \(|x-a|\) can be uniformly approximated, and multiplication by a constant and addition of polynomials preserves the ability to be uniform approximated.
    \item \(b_0 = \frac{\phi(a_1) - \phi(a_0)}{a_1 - a_0}\), and for \(n \geq 1 \), \(b_n = \frac{\phi(a_{n+1}) - \phi(a_n)}{a_{n+1} - a_n} - b_{n-1}\)
    \item Fix \(\epsilon > 0\). For a function \(f\) continuous over \([-1,1]\), apporixmate it uniformly within \(\epsilon/2\) with a polygonal function \(\phi(x)\), and approximate \(\phi(x)\) uniformly within \(\epsilon/2\) with a polynomial. The triangle inequality ensures that this polynomial uniformly approximates \(f\) within \(\epsilon\). To generalize over \([a,b]\) the same technique in Exercise 6.6.7 of scaling \(x\) and \(f\) can be used.
}
\end{solution}

\begin{exercise}
    \enum{
    \item Find a counterexample which shows that WAT is not true if we replace the closed interval $[a, b]$ with the open interval $(a, b)$.
    \item What happens if we replace $[a, b]$ with the closed set $[a, \infty)$. Does the theorem still hold?
    }
\end{exercise}
\begin{solution}
    \enum{
    \item \(1/x\) over \((0,1)\), since \(1/x\) is unbounded while any approximating polynomial must be bounded over \((0,1)\).
    \item \(e^x\), since exponentials grow faster than polynomials, and therefore the difference between \(e^x\) and any polynomial is unbounded over \([0, \infty)\).
    }
\end{solution}

\begin{exercise}
    Is there a countable subset of polynomials $\mathcal{C}$ with the property that every continuous function on $[a, b]$ can be uniformly approximated by polynomials from $\mathcal{C}$ ?
\end{exercise}
\begin{solution}
Yes - we will approach this by adapting the logic used to prove WAT.

We start by choosing some sequence of polynomials which converge uniformly to \(|x|\). The set of all polynomials that appear in this sequence is countable; denote this set by \(\mathcal{A}\). For a fixed \(a \in [-1,1]\), we can turn this into a sequence of polynomials approaching \(h_a\) as described in Exercise 6.7.8; denote this countable set of polynomials \(\mathcal{A}_a\).

The rationals in \([-1,1]\) are countable, and therefore the union of all \(\mathcal{A_a}\) for rational \(a\) is also countable; denote this set as \(\mathcal{B}\). For a similar reason, the set of all polynomials of the form \(a \cdot p(x / b)\) where \(a,b\) are rational numbers and \(p \in \mathcal{B}\) is also countable; denote this set \(\mathcal{D}\).

Because rationals are dense in \(\mathbf{R}\), it's easy to adapt the proof of Theorem 6.7.3 (Exercise 6.7.2) to only work with polygonal functions whose segment endpoints are rational (i.e.\ both the endpoint and the value of the function at the endpoint are rational). Let \(\mathcal{P}\) denote the set of polygonal functions over \([-1,1]\) with rational segment endpoints only, and \(\mathcal{P}_n\) be the elements in \(\mathcal{P}\) with \(n\) segments. Any element in \(\mathcal{P}_n\) can be uniformly approximated by the sum of \(n\) elements from \(\mathcal{D}\) plus some rational constant \(c\); call the set of polynomials that can be generated this way \(\mathcal{D}_n\). Since \(n\) is finite, \(\mathcal{D}_n\) is also countable, and so is \(\bigcup^\infty_{i=1} \mathcal{D}_i\), meaning we have a countable set of polynomials which can uniformly approximate any function from \(\mathcal{P}\).

This implies any continuous function can be uniformly apprxoimated over \([-1,1]\) by a countable set of polynomials. The scaling trick used to extend the original proof of WAT can be used here (except limited to scaling factors of rational numbers) to complete the proof.
\end{solution}

\begin{exercise}
    Assume that $f$ has a continuous derivative on $[a, b]$. Show that there exists a polynomial $p(x)$ such that
$$
|f(x)-p(x)|<\epsilon \quad \text { and } \quad\left|f^{\prime}(x)-p^{\prime}(x)\right|<\epsilon
$$
for all $x \in[a, b]$.
\end{exercise}
