\section{The Weierstrauss Approximation Theorem}

\begin{exercise}
Assuming WAT, show that if \(f\) is continuous on \([a, b]\), then there exists a sequence \((p_n)\) of polynomials such that \(p_n \to f\) uniformly on \([a, b]\).
\end{exercise}
\begin{solution}
Repeatedly apply WAT with \(\epsilon = 1/n\).
\end{solution}

\begin{exercise}
Prove Theorem 6.7.3.
\end{exercise}
\begin{solution}
Recall Theorem 4.4.7, which states that a continuous functions over a compact set is uniformly continuous over that set. Given \(\epsilon > 0\), apply uniform continuity on \(f\) with \(\epsilon/2\) to obtain some \(\delta > 0\), and partition \([a,b]\) into uniform segments, with each segment length lower than \(\delta\). Define \(\phi(x)\) at the endpoints of each segment to be equal to \(f(x)\), and to linearly interpolate between segment endpoints.

For any \(x \in (a, b)\), let \(q\) be the largest segment endpoint less than \(x\), and \(r\) be the following segment endpoint. (If \(x = a\) or \(x = b\) then these aren't necessarilly defined, but then \(phi(x) = f(x)\) so there's nothing to worry about.) Since \(\abs{x-q} < \delta\) we have that \(\abs{f(x) - \phi(q)} < \epsilon/2\). We similarly also have \(\abs{\phi(q) - \phi(r)} < \epsilon/2\). Also, note that \(\phi(x)\) must lie between \(\phi(q)\) and \(\phi(r)\), so \(\abs{\phi(q) - \phi(x)} \leq \abs{\phi(q) - \phi(r)} < \epsilon/2\). Applying the triangle inequality leaves us with \(\abs{f(x) - \phi(x)}< \epsilon\) as desired.
\end{solution}
