\section{The Weierstrauss Approximation Theorem}

\begin{exercise}
Assuming WAT, show that if \(f\) is continuous on \([a, b]\), then there exists a sequence \((p_n)\) of polynomials such that \(p_n \to f\) uniformly on \([a, b]\).
\end{exercise}
\begin{solution}
Repeatedly apply WAT with \(\epsilon = 1/n\).
\end{solution}

\begin{exercise}
Prove Theorem 6.7.3.
\end{exercise}
\begin{solution}
Recall Theorem 4.4.7, which states that a continuous functions over a compact set is uniformly continuous over that set. Given \(\epsilon > 0\), apply uniform continuity on \(f\) with \(\epsilon/2\) to obtain some \(\delta > 0\), and partition \([a,b]\) into uniform segments, with each segment length lower than \(\delta\). Define \(\phi(x)\) at the endpoints of each segment to be equal to \(f(x)\), and to linearly interpolate between segment endpoints.

For any \(x \in (a, b)\), let \(q\) be the largest segment endpoint less than \(x\), and \(r\) be the following segment endpoint. (If \(x = a\) or \(x = b\) then these aren't necessarilly defined, but then \(\phi(x) = f(x)\) so there's nothing to worry about.) Since \(\abs{x-q} < \delta\) we have that \(\abs{f(x) - \phi(q)} < \epsilon/2\). We similarly also have \(\abs{\phi(q) - \phi(r)} < \epsilon/2\). Also, note that \(\phi(x)\) must lie between \(\phi(q)\) and \(\phi(r)\), so \(\abs{\phi(q) - \phi(x)} \leq \abs{\phi(q) - \phi(r)} < \epsilon/2\). Applying the triangle inequality leaves us with \(\abs{f(x) - \phi(x)}< \epsilon\) as desired.
\end{solution}

\begin{exercise}
\enum{
    \item Find the second degree polynomial $p(x)=q_{0}+q_{1} x+q_{2} x^{2}$ that interpolates the three points $(-1,1),(0,0)$, and $(1,1)$ on the graph of $g(x)=|x|$. Sketch $g(x)$ and $p(x)$ over $[-1,1]$ on the same set of axes.
    \item Find the fourth degree polynomial that interpolates $g(x)=|x|$ at the points $x=-1,-1 / 2,0,1 / 2$, and 1 . Add a sketch of this polynomial to the graph from (a).
}
\end{exercise}
\begin{solution}
\enum{
    \item \(p(x) = x^2\)
    \item \(p(x) = \frac{7}{3}x^2 - \frac{4}{3}x^4\)
}
\end{solution}

\begin{exercise}
    Show that $f(x)=\sqrt{1-x}$ has Taylor series coefficients $a_{n}$ where $a_{0}=1$ and
$$
a_{n}=\frac{-1 \cdot 3 \cdot 5 \cdots(2 n-3)}{2 \cdot 4 \cdot 6 \cdots 2 n}
$$
for $n \geq 1$.
\end{exercise}
\begin{solution}
    \TODO
\end{solution}

\begin{exercise}
    \item Follow the advice in Exercise 6.6.9 to prove the Cauchy form of the remainder:
$$
E_{N}(x)=\frac{f^{(N+1)}(c)}{N !}(x-c)^{N} x
$$
for some $c$ between 0 and $x$.
\item Use this result to prove equation (1) is valid for all $x \in(-1,1)$.
\end{exercise}
\begin{solution}
    \TODO
\end{solution}

\begin{exercise}
    \enum{
\item Let
$$
c_{n}=\frac{1 \cdot 3 \cdot 5 \cdots(2 n-1)}{2 \cdot 4 \cdot 6 \cdots 2 n}
$$
for $n \geq 1$. Show $c_{n}<\frac{2}{\sqrt{2 n+1}}$.
\item Use (a) to show that $\sum_{n=0}^{\infty} a_{n}$ converges (absolutely, in fact) where $a_{n}$ is the sequence of Taylor coefficients generated in Exercise 6.7.4.
\item Carefully explain how this verifies that equation (1) holds for all $x \in$ $[-1,1]$
    }
\end{exercise}
\begin{solution}
    \TODO
\end{solution}

\begin{exercise}
    \enum{
    \item Use the fact that $|a|=\sqrt{a^{2}}$ to prove that, given $\epsilon>0$, there exists a polynomial $q(x)$ satisfying
$$
|| x|-q(x)|<\epsilon
$$
for all $x \in[-1,1]$.
\item Generalize this conclusion to an arbitrary interval $[a, b]$.
    }
\end{exercise}
\begin{solution}
    \TODO
\end{solution}

\begin{exercise}
    \enum{
    \item Fix $a \in[-1,1]$ and sketch
$$
h_{a}(x)=\frac{1}{2}(|x-a|+(x-a))
$$
over $[-1,1]$. Note that $h_{a}$ is polygonal and satisfies $h_{a}(x)=0$ for all $x \in[-1, a]$.
\item Explain why we know $h_{a}(x)$ can be uniformly approximated with a polynomial on $[-1,1]$.
\item Let $\phi$ be a polygonal function that is linear on each subinterval of the partition
$$
-1=a_{0}<a_{1}<a_{2}<\cdots<a_{n}=1 .
$$
Show there exist constants $b_{0}, b_{1}, \ldots, b_{n-1}$ so that
$$
\phi(x)=\phi(-1)+b_{0} h_{a_{0}}(x)+b_{1} h_{a_{1}}(x)+\cdots+b_{n-1} h_{a_{n-1}}(x)
$$
for all $x \in[-1,1]$. (d) Complete the proof of WAT for the interval $[-1,1]$, and then generalize to an arbitrary interval $[a, b]$.
    }
\end{exercise}
\begin{solution}
    \TODO
\end{solution}

\begin{exercise}
    \enum{
    \item Find a counterexample which shows that WAT is not true if we replace the closed interval $[a, b]$ with the open interval $(a, b)$.
    \item What happens if we replace $[a, b]$ with the closed set $[a, \infty)$. Does the theorem still hold?
    }
\end{exercise}
\begin{solution}
    \TODO
\end{solution}

\begin{exercise}
    Is there a countable subset of polynomials $\mathcal{C}$ with the property that every continuous function on $[a, b]$ can be uniformly approximated by polynomials from $\mathcal{C}$ ?
\end{exercise}
\begin{solution}
    \TODO
\end{solution}

\begin{exercise}
    Assume that $f$ has a continuous derivative on $[a, b]$. Show that there exists a polynomial $p(x)$ such that
$$
|f(x)-p(x)|<\epsilon \quad \text { and } \quad\left|f^{\prime}(x)-p^{\prime}(x)\right|<\epsilon
$$
for all $x \in[a, b]$.
\end{exercise}
