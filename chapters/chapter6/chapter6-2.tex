\section{Uniform Convergence of a Sequence of Functions}

\begin{exercise}
  Let
  $$
  f_{n}(x)=\frac{n x}{1+n x^{2}} .
  $$
  \enum {
  \item Find the pointwise limit of $\left(f_{n}\right)$ for all $x \in(0, \infty)$.
  \item Is the convergence uniform on $(0, \infty)$?
  \item Is the convergence uniform on $(0,1)$?
  \item Is the convergence uniform on $(1, \infty)$?
  }
\end{exercise}
\begin{solution}
  \enum{
  \item $\lim f_n(x) = \lim \frac{nx}{1+nx^2} = \lim \frac{x}{1/n + x^2} = 1/x$
  \item Examine the difference $|f_n(x) - f(x)|$
    $$
    \left|\frac{nx}{1+nx^2} - \frac{1}{x}\right|
    = \left|\frac{nx^2 - (1+nx^2)}{x(1+nx^2)}\right|
    = \frac{1}{x(1+nx^2)}
    $$
    Consider $x_n = 1/n$, then
    $$
    \left|f_n(x_n) - f(x_n)\right| = \frac{1}{(1/n)(1+n(1/n^2)} = \frac{1}{n/2} = \frac{n}{2}
    $$
    Which shows that no matter how big $n$ is, we can find $x = 1/n$ such that $|f_n(x) - f(x)| \ge 1/2$ meaning $\epsilon$ cannot be made smaller then $1/2$. So $f$ isn't uniformly continuous.
  \item No, same logic as (b)
  \item Yes, because $x \ge 1$ implies
    $$
    |f_n(x) - f(x)| = \frac{1}{x(1 + nx^2)} \le \frac{1}{n}
    $$
    Meaning forall $\epsilon > 0$, setting $N > 1/\epsilon$ implies every $n \ge N$ has $|f_n(x) - f(x)| \le 1/N < \epsilon$ for every $x \in (1,\infty)$.
  }
\end{solution}
\begin{exercise}

  \enum {
  \item Define a sequence of functions on $\mathbf{R}$ by
    $$
    f_{n}(x)= \begin{cases}1 & \text { if } x=1, \frac{1}{2}, \frac{1}{3}, \ldots, \frac{1}{n} \\ 0 & \text { otherwise }\end{cases}
    $$
    and let $f$ be the pointwise limit of $f_{n}$.
    Is each $f_{n}$ continuous at zero? Does $f_{n} \rightarrow f$ uniformly on $\mathbf{R} ?$ Is $f$ continuous at zero?
  \item Repeat this exercise using the sequence of functions
    $$
    g_{n}(x)= \begin{cases}x & \text { if } x=1, \frac{1}{2}, \frac{1}{3}, \ldots, \frac{1}{n} \\ 0 & \text { otherwise. }\end{cases}
    $$
  \item Repeat the exercise once more with the sequence
    $$
    h_{n}(x)= \begin{cases}1 & \text { if } x=\frac{1}{n} \\ x & \text { if } x=1, \frac{1}{2}, \frac{1}{3}, \ldots, \frac{1}{n-1} \\ 0 & \text { otherwise. }\end{cases}
    $$
    In each case, explain how the results are consistent with the content of the Continuous Limit Theorem (Theorem 6.2.6).
  }
\end{exercise}
\begin{solution}
  \enum{
  \item Each $f_n$ is continuous at zero, but $f$ is not continuous at zero meaning (by Theorem 6.2.6) that $f_n$ does not converge to $f$ uniformly.
  \item Each $g_n$ is continuous at zero, and the pointwise limit $g$ is also continuous at zero. Since we aren't contradicting 6.2.6 the convergence may or may not be uniform.

    The definitions show $|g(x)-g_n(x)| < 1/n$ for all $x$ (max is at $x=1/(n+1)$).
    Setting $N > 1/\epsilon$ gives (for all $n \ge N$ and for all $x \in \mathbf{R}$)
    $$
    |g(x)-g_n(x)| < \epsilon
    $$
    As desired, thus $(g_n) \to g$ uniformly.
  \item Each $h_n$ is continuous at zero, and so is the pointwise limit $h$. 6.2.6 doesn't apply so we'll have to check if the convergence is uniform.
    Notice that if $x_n = 1/n$ then
    $$
    |h(x_n) - h_n(x_n)| = 1 - 1/n
    $$
    For all $n$, meaning no matter how big $n$ is, we can't make $|h-h_n|<1/2$ for all $x$ implying $h_n$ \emph{does not} converge to $h$ uniformly.
  }
\end{solution}
\begin{exercise}
  For each $n \in \mathbf{N}$ and $x \in[0, \infty)$, let
  $$
  g_{n}(x)=\frac{x}{1+x^{n}} \quad \text { and } \quad h_{n}(x)= \begin{cases}1 & \text { if } x \geq 1 / n \\ n x & \text { if } 0 \leq x<1 / n\end{cases}
  $$
  Answer the following questions for the sequences $\left(g_{n}\right)$ and $\left(h_{n}\right)$;
  \enum {
  \item Find the pointwise limit on $[0, \infty)$.
  \item Explain how we know that the convergence cannot be uniform on $[0, \infty)$.
  \item Choose a smaller set over which the convergence is uniform and supply an argument to show that this is indeed the case.
  }
\end{exercise}
\begin{solution}
  \enum{
  \item
    $$
    \lim g_n(x) = \begin{cases}
      x &\text{if } x \in [0, 1) \\
      0 &\text{if } x \in [1, \infty) \\
    \end{cases}
    \quad\text{and}\quad
    \lim h_n(x) = \begin{cases}
      1 &\text{if } x > 0 \\
      0 &\text{if } x = 0
    \end{cases}
    $$
  \item They can't converge uniformly since it would contradict Theorem 6.2.6 as both $g_n$ and $h_n$ are continuous but the limit functions are not.
  \item
    Over $[1,\infty)$ we have $h_n(x) = h(x) = 1$ for all $n$, thus $|h_n(x) - h(x)| = 0$ for all $x \in [1,\infty)$ so $h_n$ converges uniformly.

    Now for $g_n$. Let $t \in [0,1)$, I claim $g_n(x) \to x$ uniformly over $[0, t)$ since
    $$
    \left|\frac{x}{1 + x^n} - x\right|
    = \left|\frac{x - x(1+x^n)}{1 + x^n}\right|
    = \left|\frac{x^{n+1}}{1 + x^n}\right|
    < \left|t^{n+1}\right|
    < \epsilon \quad\forall x
    $$
    After setting $n > \log_t \epsilon$.
  }
\end{solution}

\begin{exercise}
  Review Exercise 5.2.8 which includes the definition for a uniformly differentiable function. Use the results discussed in Section 2 to show that if $f$ is uniformly differentiable, then $f^{\prime}$ is continuous.

\end{exercise}
\begin{solution}
  The definition of $f$ being uniformly differentiable tells us: for every $\epsilon > 0$ there exists a $\delta > 0$ such that
  $$
  \left|\frac{f(x)-f(y)}{x-y} - f'(y)\right| < \epsilon \quad \text{whenever $0<|x-y|<\delta$}
  $$
  We can use this to show continuity of $f'$ via a triangle inequality and exploiting the symmetry in $x,y$.
  $$
  |f'(x) - f'(y)|
  < \left|f'(x) - \frac{f(x)-f(y)}{x-y}\right|
  + \left|\frac{f(x)-f(y)}{x-y} - f'(y)\right|
  < \epsilon
  $$
  After picking $\delta$ so that every $|x-y|<\delta$ has
  $$
  \left|\frac{f(x)-f(y)}{x-y} - f'(y)\right| < \epsilon/2
  $$
\end{solution}
\begin{exercise}
  Using the Cauchy Criterion for convergent sequences of real numbers (Theorem 2.6.4), supply a proof for Theorem 6.2.5 (Cauchy Criterion for Uniform Convergence). (First, define a candidate for $f(x)$, and then argue that $f_{n} \rightarrow f$ uniformly.)

\end{exercise}
\begin{solution}
  First suppose $(f_n)$ converges uniformly to $f$ and set $N$ large enough that $n \ge N$ has $|f_n(x) - f(x)| < \epsilon/2$ for all $x$, then use the triangle inequality (where $m \ge N$ as well)
  $$
  |f_n(x) - f_m(x)| \le |f_n(x) - f(x)| + |f(x) - f_m(x)| < \epsilon/2 + \epsilon/2 = \epsilon.
  $$
  Second suppose we can find an $N$ so that every $n,m \ge N$ has $|f_n(x) - f_m(x)| < \epsilon$ for all $x$.
  Fix $x$ and apply Theorem 2.6.4 to conclude the sequence $(f_n(x))$ converges to some limit $L$, and define $f(x) = L$. Doing this for all $x$ gives us the pointwise limit $f$. Now we show $(f_n) \to f$ uniformly using the fact that $|f_n(x) - f_m(x)| < \epsilon$ \emph{for all} $x$.
  Let $n \ge N$, notice that \emph{for all} $m$
  $$
  |f_n(x) - f(x)| \le |f_n(x) - f_m(x)| + |f_m(x) - f(x)|
  $$
  Let $\epsilon_m = |f_m(x) - f(x)|$. for all $m \ge N$ we have $|f_n(x) - f_m(x)| < \epsilon$ and
  $$
  |f_n(x) - f(x)| \le \epsilon + \epsilon_m
  $$
  Since $\epsilon_m \to 0$ no matter what $x$ is (pointwise convergence) and the inequality is for all $m$ this implies $|f_n(x) - f(x)| \le \epsilon$ for all $x$ as desired.
\end{solution}
\begin{exercise}
  Assume $f_{n} \rightarrow f$ on a set $A$. Theorem 6.2.6 is an example of a typical type of question which asks whether a trait possessed by each $f_{n}$ is inherited by the limit function. Provide an example to show that all of the following propositions are false if the convergence is only assumed to be pointwise on $A$. Then go back and decide which are true under the stronger hypothesis of uniform convergence.
  \enum {
  \item If each $f_{n}$ is uniformly continuous, then $f$ is uniformly continuous.
  \item If each $f_{n}$ is bounded, then $f$ is bounded.
  \item If each $f_{n}$ has a finite number of discontinuities, then $f$ has a finite number of discontinuities.
  \item If each $f_{n}$ has fewer than $M$ discontinuities (where $M \in \mathbf{N}$ is fixed), then $f$ has fewer than $M$ discontinuities.
  \item If each $f_{n}$ has at most a countable number of discontinuities, then $f$ has at most a countable number of discontinuities.

  }
\end{exercise}
\begin{solution}
  \enum{
  \item False pointwise when
    $$
    f_n(x) = \frac{1}{1+nx^2} \quad\text{and}\quad f(x) = \begin{cases}
      1 &\text{if $x = 0$} \\
      0 &\text{otherwise}
    \end{cases}
    $$
    Now suppose $(f_n)$ converges to $f$ uniformly. Theorem 6.2.6 implies $f$ will be continuous, combining this with Theorem 4.4.7 we see that $f$ must be uniformly continuous if the domain $A$ is compact.
    If $A$ is not compact this need not be the case, consider $A = (0,1)$ and $f_n(x) = 1/x = f(x)$. Clearly $(f_n) \to f$ uniformly (they're the same function!) but $f(x) = 1/x$ is not uniformly continuous on $A$.
  \item False pointwise when
    $$
    f_n(x)
    = \begin{cases}
      x &\text{if $x < n$} \\
      0 &\text{otherwise}
    \end{cases}
    \quad\text{and}\quad
    f(x) = x
    $$
    Now suppose $(f_n) \to f$ uniformly.
    We want to show $f$ is bounded, 
    Let $M_n$ bound $f_n$, ie. $|f_n(x)|<M_n$ for all $x \in A$.
    
    Set $\epsilon=1$ and apply the Cauchy Criterion to get $N$ so $m \ge n > N$ implies
    $$
    |f_n(x) - f_m(x)| < 1
    $$
    Setting $n=N$ and rearranging gives
    $$
    |f_m(x)| < |f_N(x)| + \epsilon < M_N + \epsilon
    $$
    implying $f_m$ is bounded by $M_N$ when $m \ge N$.

    Now set $\tilde N > N$ large enough that $m \ge N$ implies $|f(x)-f_m(x)|<1$ which, after rearranging gives
    $$
    |f(x)| < 1 + |f_m(x)| < 1 + M_N
    $$
    Implying $f$ is bounded.
  \item False when $f_n$ is the step function repeated $n$ times, then $f$ has infinitely many discontinuities.

    Now suppose $(f_n) \to f$ uniformly. Let $x_0$ be a discontinuity of $f$, meaning there exists an $\epsilon_0$ such that $|f(x_0)-f(x)| > \epsilon_0$ no matter how small $|x-x_0|$ is.
    I'd like to show $x_0$ is a discontinuity of $f_n$ for some $n$, ie. that there exists an $\epsilon_0'>0$ such that $|f_n(x_0)-f_n(x)|>\epsilon_0'$ no matter how small $|x-x_0|$ is.

    Pick $0<\epsilon<2\epsilon_0$ and set $N$ large enough that $n \ge N$ implies $|f_n(x)-f(x)|<\epsilon$. applying the three way triangle inequality gives
    $$
    \begin{aligned}
    \epsilon_0
    &< |f(x_0) - f(x)| \\
    &\le |f(x_0) - f_n(x_0)| + |f_n(x_0) - f_n(x)| + |f_n(x) - f(x)| \\
    &< 2\epsilon + |f_n(x_0) - f_n(x)|
    \end{aligned}
    $$
    Letting $\epsilon_0' = \epsilon_0-2\epsilon > 0$ we see $|f_n(x_0) - f_n(x)| > \epsilon_0'$ meaning $f_n$ is not continuous at $x_0$ for all $n \ge N$.

    This doesn't show $f$ has finitely many discontinuities however, because the amount of discontinuities in $(f_n)$ may be increasing (as opposed to (d) where they are all bounded)

    \TODO Finish, I think this one requires a counterexample, though this proof works for part (d)
  \item False pointwise, see the counterexample in (c)
    \TODO
  \item False pointwise when (using a modified version of Thomae's function for $f_n$)
    $$
    f_n(x) = \begin{cases}
      1 &\text{if $x = 0$} \\
      \frac{n}{n+q} &\text{if $x = p/q$ (in lowest terms)} \\
      0 &\text{otherwise}
    \end{cases}
    \quad\text{and}\quad
    f(x) = \begin{cases}
      1 &\text{if $x \in \mathbf{Q}$} \\
      0 &\text{if $x \notin \mathbf{Q}$} \\
    \end{cases}
    $$
    \TODO
  }
\end{solution}
\begin{exercise}
  Let $f$ be uniformly continuous on all of $\mathbf{R}$, and define a sequence of functions by $f_{n}(x)=f\left(x+\frac{1}{n}\right)$. Show that $f_{n} \rightarrow f$ uniformly. Give an example to show that this proposition fails if $f$ is only assumed to be continuous and not uniformly continuous on $\mathbf{R}$.

\end{exercise}
\begin{solution}
  Given $\epsilon>0$ set $\delta>0$ such that $|x-y|<\delta$ implies $|f(x)-f(y)|<\epsilon$.
  Then set $N > 1/\delta$ so that $n \ge N$ implies (since $1/n<\delta$)
  $$
  |f(x) - f_n(x)| = |f(x) - f(x+1/n)| < \epsilon
  $$
  Which shows $(f_n) \to f$ uniformly.

  To see this doesn't work if $f$ is only continuous, consider $f : (0,1)\to\mathbf{R}$ defined by $f(x) = 1/x$.
  We have
  $$
  |f(x)-f_n(x)| = \frac{1}{x} - \frac{1}{x+1/n} = \frac{1/n}{x(x+1/n)} = \frac{1}{x(nx+1)}
  $$
  Which given a fixed $n$ becomes arbitrarily big as $x$ goes to zero. hence $(f_n)$ does not converge uniformly.
\end{solution}
\begin{exercise}
  Let $\left(g_{n}\right)$ be a sequence of continuous functions that converges uniformly to $g$ on a compact set $K$. If $g(x) \neq 0$ on $K$, show $\left(1 / g_{n}\right)$ converges uniformly on $K$ to $1 / g$.

\end{exercise}
\begin{solution}
  \TODO
\end{solution}
\begin{exercise}
  Assume $\left(f_{n}\right)$ and $\left(g_{n}\right)$ are uniformly convergent sequences of functions.
  \enum {
  \item Show that $\left(f_{n}+g_{n}\right)$ is a uniformly convergent sequence of functions.
  \item Give an example to show that the product $\left(f_{n} g_{n}\right)$ may not converge uniformly.
  \item Prove that if there exists an $M>0$ such that $\left|f_{n}\right| \leq M$ and $\left|g_{n}\right| \leq M$ for all $n \in \mathbf{N}$, then $\left(f_{n} g_{n}\right)$ does converge uniformly.

  }
\end{exercise}
\begin{solution}
  \TODO
\end{solution}
\begin{exercise}
  This exercise and the next explore partial converses of the Continuous Limit Theorem (Theorem 6.2.6). Assume $f_{n} \rightarrow f$ pointwise on $[a, b]$ and the limit function $f$ is continuous on $[a, b]$. If each $f_{n}$ is increasing (but not necessarily continuous), show $f_{n} \rightarrow f$ uniformly.


\end{exercise}
\begin{solution}
  \TODO
\end{solution}
\begin{exercise}[Dini's Theorem]
  Assume $f_{n} \rightarrow f$ pointwise on a compact set $K$ and assume that for each $x \in K$ the sequence $f_{n}(x)$ is increasing. Follow these steps to show that if $f_{n}$ and $f$ are continuous on $K$, then the convergence is uniform.
  \enum {
  \item Set $g_{n}=f-f_{n}$ and translate the preceding hypothesis into statements about the sequence $\left(g_{n}\right)$.
  \item Let $\epsilon>0$ be arbitrary, and define $K_{n}=\left\{x \in K: g_{n}(x) \geq \epsilon\right\} .$ Argue that $K_{1} \supseteq K_{2} \supseteq K_{3} \supseteq \cdots$, and use this observation to finish the argument.
  }
\end{exercise}
\begin{solution}
  \TODO
\end{solution}
\begin{exercise}[Cantor Function]
  Review the construction of the Cantor set $C \subseteq[0,1]$ from Section 3.1. This exercise makes use of results and notation from this discussion.
  \enum {
  \item Define $f_{0}(x)=x$ for all $x \in[0,1]$. Now, let
    $$
    f_{1}(x)= \begin{cases}(3 / 2) x & \text { for } 0 \leq x \leq 1 / 3 \\ 1 / 2 & \text { for } 1 / 3<x<2 / 3 \\ (3 / 2) x-1 / 2 & \text { for } 2 / 3 \leq x \leq 1\end{cases}
    $$
    Sketch $f_{0}$ and $f_{1}$ over $[0,1]$ and observe that $f_{1}$ is continuous, increasing, and constant on the middle third $(1 / 3,2 / 3)=[0,1] \backslash C_{1}$.
  \item Construct $f_{2}$ by imitating this process of flattening out the middle third of each nonconstant segment of $f_{1}$. Specifically, let
    $$
    f_{2}(x)= \begin{cases}(1 / 2) f_{1}(3 x) & \text { for } 0 \leq x \leq 1 / 3 \\ f_{1}(x) & \text { for } 1 / 3<x<2 / 3 \\ (1 / 2) f_{1}(3 x-2)+1 / 2 & \text { for } 2 / 3 \leq x \leq 1\end{cases}
    $$
    If we continue this process, show that the resulting sequence $\left(f_{n}\right)$ converges uniformly on $[0,1]$.
  \item Let $f=\lim f_{n}$. Prove that $f$ is a continuous, increasing function on $[0,1]$ with $f(0)=0$ and $f(1)=1$ that satisfies $f^{\prime}(x)=0$ for all $x$ in the open set $[0,1] \backslash C$. Recall that the "length" of the Cantor set $C$ is 0 . Somehow, $f$ manages to increase from 0 to 1 while remaining constant on a set of "length 1."

  }
\end{exercise}
\begin{solution}
  \TODO
\end{solution}
\begin{exercise}
  Recall that the Bolzano-Weierstrass Theorem (Theorem 2.5.5) states that every bounded sequence of real numbers has a convergent subsequence. An analogous statement for bounded sequences of functions is not true in general, but under stronger hypotheses several different conclusions are possible. One avenue is to assume the common domain for all of the functions in the sequence is countable. (Another is explored in the next two exercises.)
  Let $A=\left\{x_{1}, x_{2}, x_{3}, \ldots\right\}$ be a countable set. For each $n \in \mathbf{N}$, let $f_{n}$ be defined on $A$ and assume there exists an $M>0$ such that $\left|f_{n}(x)\right| \leq M$ for all $n \in \mathbf{N}$ and $x \in A$. Follow these steps to show that there exists a subsequence of $\left(f_{n}\right)$ that converges pointwise on $A$.

  \enum {
  \item Why does the sequence of real numbers $f_{n}\left(x_{1}\right)$ necessarily contain a convergent subsequence $\left(f_{n_{k}}\right)$ ? To indicate that the subsequence of functions $\left(f_{n_{k}}\right)$ is generated by considering the values of the functions at $x_{1}$, we will use the notation $f_{n_{N}}=f_{1, k}$.
  \item Now, explain why the sequence $f_{1, k}\left(x_{2}\right)$ contains a convergent subsequence.
  \item Carefully construct a nested family of subsequences $\left(f_{m, k}\right)$, and show how this can be used to produce a single subsequence of $\left(f_{n}\right)$ that converges at every point of $A$.

  }
\end{exercise}
\begin{solution}
  \TODO
\end{solution}
\begin{exercise}
  A sequence of functions $\left(f_{n}\right)$ defined on a set $E \subseteq \mathbf{R}$ is called equicontinuous if for every $\epsilon>0$ there exists a $\delta>0$ such that $\left|f_{n}(x)-f_{n}(y)\right|<\epsilon$ for all $n \in \mathbf{N}$ and $|x-y|<\delta$ in $E$.
  \enum {
  \item What is the difference between saying that a sequence of functions $\left(f_{n}\right)$ is equicontinuous and just asserting that each $f_{n}$ in the sequence is individually uniformly continuous?
  \item Give a qualitative explanation for why the sequence $g_{n}(x)=x^{n}$ is not equicontinuous on $[0,1]$. Is each $g_{n}$ uniformly continuous on $[0,1]$ ?

  }
\end{exercise}
\begin{solution}
  \TODO
\end{solution}
\begin{exercise}[Arzela-Ascoli Theorem]
  For each $n \in \mathbf{N}$, let $f_{n}$ be a function defined on $[0,1]$. If $\left(f_{n}\right)$ is bounded on $[0,1]$-that is, there exists an $M>0$ such that $\left|f_{n}(x)\right| \leq M$ for all $n \in \mathbf{N}$ and $x \in[0,1]$-and if the collection of functions $\left(f_{n}\right)$ is equicontinuous (Exercise 6.2.14), follow these steps to show that $\left(f_{n}\right)$ contains a uniformly convergent subsequence.
  \enum {
  \item Use Exercise 6.2.13 to produce a subsequence $\left(f_{n_{k}}\right)$ that converges at every rational point in $[0,1]$. To simplify the notation, set $g_{k}=f_{n_{k}}$. It remains to show that $\left(g_{k}\right)$ converges uniformly on all of $[0,1]$.
  \item Let $\epsilon>0$. By equicontinuity, there exists a $\delta>0$ such that
    $$
    \left|g_{k}(x)-g_{k}(y)\right|<\frac{\epsilon}{3}
    $$
    for all $|x-y|<\delta$ and $k \in \mathbf{N}$. Using this $\delta$, let $r_{1}, r_{2}, \ldots, r_{m}$ be a finite collection of rational points with the property that the union of the neighborhoods $V_{\delta}\left(r_{i}\right)$ contains $[0,1]$.
    Explain why there must exist an $N \in \mathbf{N}$ such that
    $$
    \left|g_{s}\left(r_{i}\right)-g_{t}\left(r_{i}\right)\right|<\frac{\epsilon}{3}
    $$
    for all $s, t \geq N$ and $r_{i}$ in the finite subset of $[0,1]$ just described. Why does having the set $\left\{r_{1}, r_{2}, \ldots, r_{m}\right\}$ be finite matter?
  \item Finish the argument by showing that, for an arbitrary $x \in[0,1]$,
    $$
    \left|g_{s}(x)-g_{t}(x)\right|<\epsilon
    $$
    for all $s, t \geq N$.
  }
\end{exercise}
\begin{solution}
  \TODO
\end{solution}
