\section{Uniform Convergence of a Sequence of Functions}

\begin{exercise}
  Let
  $$
  f_{n}(x)=\frac{n x}{1+n x^{2}} .
  $$
  \enum {
  \item Find the pointwise limit of $\left(f_{n}\right)$ for all $x \in(0, \infty)$.
  \item Is the convergence uniform on $(0, \infty)$?
  \item Is the convergence uniform on $(0,1)$?
  \item Is the convergence uniform on $(1, \infty)$?
  }
\end{exercise}
\begin{solution}
  \enum{
  \item $\lim f_n(x) = \lim \frac{nx}{1+nx^2} = \lim \frac{x}{1/n + x^2} = 1/x$
  \item Examine the difference $|f_n(x) - f(x)|$
    $$
    \left|\frac{nx}{1+nx^2} - \frac{1}{x}\right|
    = \left|\frac{nx^2 - (1+nx^2)}{x(1+nx^2)}\right|
    = \frac{1}{x(1+nx^2)}
    $$
    Consider $x_n = 1/n$, then
    $$
    \left|f_n(x_n) - f(x_n)\right| = \frac{1}{(1/n)(1+n(1/n^2)} = \frac{1}{n/2} = \frac{n}{2}
    $$
    Which shows that no matter how big $n$ is, we can find $x = 1/n$ such that $|f_n(x) - f(x)| \ge 1/2$ meaning $\epsilon$ cannot be made smaller then $1/2$. So $f$ isn't uniformly continuous.
  \item No, same logic as (b)
  \item Yes, because $x \ge 1$ implies
    $$
    |f_n(x) - f(x)| = \frac{1}{x(1 + nx^2)} \le \frac{1}{n}
    $$
    Meaning forall $\epsilon > 0$, setting $N > 1/\epsilon$ implies every $n \ge N$ has $|f_n(x) - f(x)| \le 1/N < \epsilon$ for every $x \in (1,\infty)$.
  }
\end{solution}
\begin{exercise}

  \enum {
  \item Define a sequence of functions on $\mathbf{R}$ by
    $$
    f_{n}(x)= \begin{cases}1 & \text { if } x=1, \frac{1}{2}, \frac{1}{3}, \ldots, \frac{1}{n} \\ 0 & \text { otherwise }\end{cases}
    $$
    and let $f$ be the pointwise limit of $f_{n}$.
    Is each $f_{n}$ continuous at zero? Does $f_{n} \rightarrow f$ uniformly on $\mathbf{R} ?$ Is $f$ continuous at zero?
  \item Repeat this exercise using the sequence of functions
    $$
    g_{n}(x)= \begin{cases}x & \text { if } x=1, \frac{1}{2}, \frac{1}{3}, \ldots, \frac{1}{n} \\ 0 & \text { otherwise. }\end{cases}
    $$
  \item Repeat the exercise once more with the sequence
    $$
    h_{n}(x)= \begin{cases}1 & \text { if } x=\frac{1}{n} \\ x & \text { if } x=1, \frac{1}{2}, \frac{1}{3}, \ldots, \frac{1}{n-1} \\ 0 & \text { otherwise. }\end{cases}
    $$
    In each case, explain how the results are consistent with the content of the Continuous Limit Theorem (Theorem 6.2.6).
  }
\end{exercise}
\begin{solution}
  \TODO
\end{solution}
\begin{exercise}
  For each $n \in \mathbf{N}$ and $x \in[0, \infty)$, let
  $$
  g_{n}(x)=\frac{x}{1+x^{n}} \quad \text { and } \quad h_{n}(x)= \begin{cases}1 & \text { if } x \geq 1 / n \\ n x & \text { if } 0 \leq x<1 / n\end{cases}
  $$
  Answer the following questions for the sequences $\left(g_{n}\right)$ and $\left(h_{n}\right)$;
  \enum {
  \item Find the pointwise limit on $[0, \infty)$.
  \item Explain how we know that the convergence cannot be uniform on $[0, \infty)$.
  \item Choose a smaller set over which the convergence is uniform and supply an argument to show that this is indeed the case.
  }
\end{exercise}
\begin{solution}
  \enum{
  \item
    $$
    \lim g_n(x) = \begin{cases}
      x &\text{if } x \in [0, 1) \\
      0 &\text{if } x \in [1, \infty) \\
    \end{cases}
    \quad\text{and}\quad
    \lim h_n(x) = \begin{cases}
      1 &\text{if } x > 0 \\
      0 &\text{if } x = 0
    \end{cases}
    $$
  \item They can't converge uniformly since it would contradict Theorem 6.2.6 as both $g_n$ and $h_n$ are continuous but the limit functions are not.
  \item
    % TODO: Combine both into a single interval, the one I used for h_n is basically cheating

    Over $[1,\infty)$ we have $h_n(x) = 1$ for all $n$, thus $|h_n(x) - 1| = 0$ so $h_n$ obviously converges uniformly.

    Now for $g_n$. Let $t \in [0,1)$, I claim $g_n(x) \to x$ uniformly over $[0, t)$ since
    $$
    \left|\frac{x}{1 + x^n} - x\right|
    = \left|\frac{x - x(1+x^n)}{1 + x^n}\right|
    = \left|\frac{x^{n+1}}{1 + x^n}\right|
    < \left|t^{n+1}\right|
    < \epsilon \quad\forall x
    $$
    After setting $n > \log_t \epsilon$.
  }
\end{solution}

\begin{exercise}
  Review Exercise 5.2.8 which includes the definition for a uniformly differentiable function. Use the results discussed in Section 2 to show that if $f$ is uniformly differentiable, then $f^{\prime}$ is continuous.

\end{exercise}
\begin{solution}
  \TODO
\end{solution}
\begin{exercise}
  Using the Cauchy Criterion for convergent sequences of real numbers (Theorem 2.6.4), supply a proof for Theorem 6.2.5. (First, define a candidate for $f(x)$, and then argue that $f_{n} \rightarrow f$ uniformly.)

\end{exercise}
\begin{solution}
  \TODO
\end{solution}
\begin{exercise}
  Assume $f_{n} \rightarrow f$ on a set $A$. Theorem 6.2.6 is an example of a typical type of question which asks whether a trait possessed by each $f_{n}$ is inherited by the limit function. Provide an example to show that all of the following propositions are false if the convergence is only assumed to be pointwise on $A$. Then go back and decide which are true under the stronger hypothesis of uniform convergence.
  \enum {
  \item If each $f_{n}$ is uniformly continuous, then $f$ is uniformly continuous.
  \item If each $f_{n}$ is bounded, then $f$ is bounded.
  \item If each $f_{n}$ has a finite number of discontinuities, then $f$ has a finite number of discontinuities.
  \item If each $f_{n}$ has fewer than $M$ discontinuities (where $M \in \mathbf{N}$ is fixed), then $f$ has fewer than $M$ discontinuities.
  \item If each $f_{n}$ has at most a countable number of discontinuities, then $f$ has at most a countable number of discontinuities.

  }
\end{exercise}
\begin{solution}
  \TODO
\end{solution}
\begin{exercise}
  Let $f$ be uniformly continuous on all of $\mathbf{R}$, and define a sequence of functions by $f_{n}(x)=f\left(x+\frac{1}{n}\right)$. Show that $f_{n} \rightarrow f$ uniformly. Give an example to show that this proposition fails if $f$ is only assumed to be continuous and not uniformly continuous on $\mathbf{R}$.

\end{exercise}
\begin{solution}
  \TODO
\end{solution}
\begin{exercise}
  Let $\left(g_{n}\right)$ be a sequence of continuous functions that converges uniformly to $g$ on a compact set $K$. If $g(x) \neq 0$ on $K$, show $\left(1 / g_{n}\right)$ converges uniformly on $K$ to $1 / g$.

\end{exercise}
\begin{solution}
  \TODO
\end{solution}
\begin{exercise}
  Assume $\left(f_{n}\right)$ and $\left(g_{n}\right)$ are uniformly convergent sequences of functions.
  \enum {
  \item Show that $\left(f_{n}+g_{n}\right)$ is a uniformly convergent sequence of functions.
  \item Give an example to show that the product $\left(f_{n} g_{n}\right)$ may not converge uniformly.
  \item Prove that if there exists an $M>0$ such that $\left|f_{n}\right| \leq M$ and $\left|g_{n}\right| \leq M$ for all $n \in \mathbf{N}$, then $\left(f_{n} g_{n}\right)$ does converge uniformly.

  }
\end{exercise}
\begin{solution}
  \TODO
\end{solution}
\begin{exercise}
  This exercise and the next explore partial converses of the Continuous Limit Theorem (Theorem 6.2.6). Assume $f_{n} \rightarrow f$ pointwise on $[a, b]$ and the limit function $f$ is continuous on $[a, b]$. If each $f_{n}$ is increasing (but not necessarily continuous), show $f_{n} \rightarrow f$ uniformly.


\end{exercise}
\begin{solution}
  \TODO
\end{solution}
\begin{exercise}[Dini's Theorem]
  Assume $f_{n} \rightarrow f$ pointwise on a compact set $K$ and assume that for each $x \in K$ the sequence $f_{n}(x)$ is increasing. Follow these steps to show that if $f_{n}$ and $f$ are continuous on $K$, then the convergence is uniform.
  \enum {
  \item Set $g_{n}=f-f_{n}$ and translate the preceding hypothesis into statements about the sequence $\left(g_{n}\right)$.
  \item Let $\epsilon>0$ be arbitrary, and define $K_{n}=\left\{x \in K: g_{n}(x) \geq \epsilon\right\} .$ Argue that $K_{1} \supseteq K_{2} \supseteq K_{3} \supseteq \cdots$, and use this observation to finish the argument.
  }
\end{exercise}
\begin{solution}
  \TODO
\end{solution}
\begin{exercise}[Cantor Function]
  Review the construction of the Cantor set $C \subseteq[0,1]$ from Section 3.1. This exercise makes use of results and notation from this discussion.
  \enum {
  \item Define $f_{0}(x)=x$ for all $x \in[0,1]$. Now, let
    $$
    f_{1}(x)= \begin{cases}(3 / 2) x & \text { for } 0 \leq x \leq 1 / 3 \\ 1 / 2 & \text { for } 1 / 3<x<2 / 3 \\ (3 / 2) x-1 / 2 & \text { for } 2 / 3 \leq x \leq 1\end{cases}
    $$
    Sketch $f_{0}$ and $f_{1}$ over $[0,1]$ and observe that $f_{1}$ is continuous, increasing, and constant on the middle third $(1 / 3,2 / 3)=[0,1] \backslash C_{1}$.
  \item Construct $f_{2}$ by imitating this process of flattening out the middle third of each nonconstant segment of $f_{1}$. Specifically, let
    $$
    f_{2}(x)= \begin{cases}(1 / 2) f_{1}(3 x) & \text { for } 0 \leq x \leq 1 / 3 \\ f_{1}(x) & \text { for } 1 / 3<x<2 / 3 \\ (1 / 2) f_{1}(3 x-2)+1 / 2 & \text { for } 2 / 3 \leq x \leq 1\end{cases}
    $$
    If we continue this process, show that the resulting sequence $\left(f_{n}\right)$ converges uniformly on $[0,1]$.
  \item Let $f=\lim f_{n}$. Prove that $f$ is a continuous, increasing function on $[0,1]$ with $f(0)=0$ and $f(1)=1$ that satisfies $f^{\prime}(x)=0$ for all $x$ in the open set $[0,1] \backslash C$. Recall that the "length" of the Cantor set $C$ is 0 . Somehow, $f$ manages to increase from 0 to 1 while remaining constant on a set of "length 1."

  }
\end{exercise}
\begin{solution}
  \TODO
\end{solution}
\begin{exercise}
  Recall that the Bolzano-Weierstrass Theorem (Theorem 2.5.5) states that every bounded sequence of real numbers has a convergent subsequence. An analogous statement for bounded sequences of functions is not true in general, but under stronger hypotheses several different conclusions are possible. One avenue is to assume the common domain for all of the functions in the sequence is countable. (Another is explored in the next two exercises.)
  Let $A=\left\{x_{1}, x_{2}, x_{3}, \ldots\right\}$ be a countable set. For each $n \in \mathbf{N}$, let $f_{n}$ be defined on $A$ and assume there exists an $M>0$ such that $\left|f_{n}(x)\right| \leq M$ for all $n \in \mathbf{N}$ and $x \in A$. Follow these steps to show that there exists a subsequence of $\left(f_{n}\right)$ that converges pointwise on $A$.

  \enum {
  \item Why does the sequence of real numbers $f_{n}\left(x_{1}\right)$ necessarily contain a convergent subsequence $\left(f_{n_{k}}\right)$ ? To indicate that the subsequence of functions $\left(f_{n_{k}}\right)$ is generated by considering the values of the functions at $x_{1}$, we will use the notation $f_{n_{N}}=f_{1, k}$.
  \item Now, explain why the sequence $f_{1, k}\left(x_{2}\right)$ contains a convergent subsequence.
  \item Carefully construct a nested family of subsequences $\left(f_{m, k}\right)$, and show how this can be used to produce a single subsequence of $\left(f_{n}\right)$ that converges at every point of $A$.

  }
\end{exercise}
\begin{solution}
  \TODO
\end{solution}
\begin{exercise}
  A sequence of functions $\left(f_{n}\right)$ defined on a set $E \subseteq \mathbf{R}$ is called equicontinuous if for every $\epsilon>0$ there exists a $\delta>0$ such that $\left|f_{n}(x)-f_{n}(y)\right|<\epsilon$ for all $n \in \mathbf{N}$ and $|x-y|<\delta$ in $E$.
  \enum {
  \item What is the difference between saying that a sequence of functions $\left(f_{n}\right)$ is equicontinuous and just asserting that each $f_{n}$ in the sequence is individually uniformly continuous?
  \item Give a qualitative explanation for why the sequence $g_{n}(x)=x^{n}$ is not equicontinuous on $[0,1]$. Is each $g_{n}$ uniformly continuous on $[0,1]$ ?

  }
\end{exercise}
\begin{solution}
  \TODO
\end{solution}
\begin{exercise}[Arzela-Ascoli Theorem]
  For each $n \in \mathbf{N}$, let $f_{n}$ be a function defined on $[0,1]$. If $\left(f_{n}\right)$ is bounded on $[0,1]$-that is, there exists an $M>0$ such that $\left|f_{n}(x)\right| \leq M$ for all $n \in \mathbf{N}$ and $x \in[0,1]$-and if the collection of functions $\left(f_{n}\right)$ is equicontinuous (Exercise 6.2.14), follow these steps to show that $\left(f_{n}\right)$ contains a uniformly convergent subsequence.
  \enum {
  \item Use Exercise 6.2.13 to produce a subsequence $\left(f_{n_{k}}\right)$ that converges at every rational point in $[0,1]$. To simplify the notation, set $g_{k}=f_{n_{k}}$. It remains to show that $\left(g_{k}\right)$ converges uniformly on all of $[0,1]$.
  \item Let $\epsilon>0$. By equicontinuity, there exists a $\delta>0$ such that
    $$
    \left|g_{k}(x)-g_{k}(y)\right|<\frac{\epsilon}{3}
    $$
    for all $|x-y|<\delta$ and $k \in \mathbf{N}$. Using this $\delta$, let $r_{1}, r_{2}, \ldots, r_{m}$ be a finite collection of rational points with the property that the union of the neighborhoods $V_{\delta}\left(r_{i}\right)$ contains $[0,1]$.
    Explain why there must exist an $N \in \mathbf{N}$ such that
    $$
    \left|g_{s}\left(r_{i}\right)-g_{t}\left(r_{i}\right)\right|<\frac{\epsilon}{3}
    $$
    for all $s, t \geq N$ and $r_{i}$ in the finite subset of $[0,1]$ just described. Why does having the set $\left\{r_{1}, r_{2}, \ldots, r_{m}\right\}$ be finite matter?
  \item Finish the argument by showing that, for an arbitrary $x \in[0,1]$,
    $$
    \left|g_{s}(x)-g_{t}(x)\right|<\epsilon
    $$
    for all $s, t \geq N$.
  }
\end{exercise}
\begin{solution}
  \TODO
\end{solution}
