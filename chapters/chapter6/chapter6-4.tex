\section{Series of Functions}

\begin{exercise}
  Supply the details for the proof of the Weierstrass M-Test (Corollary 6.4.5).

\end{exercise}
\begin{solution}
Let \(\epsilon > 0\). Since \(\sum^\infty_{n=1}M_n\) converges, by the Cauchy Criterion for Series there must be some \(N\) where if \(n>m > N\) then \(\sum^{n}_{i=m+1}M_i < \epsilon\).
Then
\[\abs{\sum^n_{i=m+1}f_i(x)} \leq \sum^n_{i=m+1}\abs{f_i(x)} \leq \sum^n_{i=m+1} M_i < \epsilon\]
and applying the Cauchy Criterion, the proof is done.
\end{solution}
\begin{exercise}
  Decide whether each proposition is true or false, providing a short justification or counterexample as appropriate.
  \enum {
  \item If $\sum_{n=1}^{\infty} g_{n}$ converges uniformly, then $\left(g_{n}\right)$ converges uniformly to zero.
  \item If $0 \leq f_{n}(x) \leq g_{n}(x)$ and $\sum_{n=1}^{\infty} g_{n}$ converges uniformly, then $\sum_{n=1}^{\infty} f_{n}$ converges uniformly.
  \item If $\sum_{n=1}^{\infty} f_{n}$ converges uniformly on $A$, then there exist constants $M_{n}$ such that $\left|f_{n}(x)\right| \leq M_{n}$ for all $x \in A$ and $\sum_{n=1}^{\infty} M_{n}$ converges.
  }
\end{exercise}
\begin{solution}
  \enum{
    \item True: applying the Cauchy Criterion with \(n = m+1\) we have that \(|g_n(x)| < \epsilon\) for any \(\epsilon > 0\), therefore \((g_n) \to 0\).
    \item True: \[\abs{\sum^n_{i=m+1}f_i(x)} = \sum^n_{i=m+1}f_i(x) \leq \sum^n_{i=m+1}g_i(x) = \abs{\sum^n_{i=m+1} g_i(x)} < \epsilon\] and therefore \(\sum(f_i)\) converges uniformly.
    \item False: Consider the following sequence of functions, defined over \([0,1)\):
    \[g_{i,j}(x) = \begin{cases}
        2^{-i} & 2^{-i}(j-1) \leq x < 2^{-i}j \\
        0 & \text{otherwise}
    \end{cases}\]
    with \(i \geq 1\) and \(j\) an integer ranging from 1 to \(2^i\) inclusive. Each \(g_{i,j}(x)\) consists of a pulse of height and width \(2^{-i}\), at disjoint locations for each \(i\). Let \(f_n(x)\) be obtained by iterating through each \(g_{1,j}\), then through each \(g_{2,j}\), then through each \(g_{3,j}\), and so on.

   \(\sum^\infty_{n=1} f_n\) converges to 1 because
   \[\sum^{2^i}_{k=1} g_{i,k} = 2^{-i}\]
   , and uniform convergence is achieved when we include all of the \(g_{i,j}\) for a given \(g_i\). On the other hand, the upper bound (and therefore minimum value of the constant \(M_{n}\)) for each \(g_{i,j}\) is \(2^{-i}\), with
   \[\sum^{2^i}_{k=1} \max g_{i,k}(x) = 1\]
   which implies that \(\sum^\infty_{n=1} M_n\) will not converge.
  }
\end{solution}
\begin{exercise}
  \enum {
  \item Show that
    $$
    g(x)=\sum_{n=0}^{\infty} \frac{\cos \left(2^{n} x\right)}{2^{n}}
    $$
    is continuous on all of $\mathbf{R}$.
  \item The function $g$ was cited in Section 5.4 as an example of a continuous nowhere differentiable function. What happens if we try to use Theorem $6.4 .3$ to explore whether $g$ is differentiable?
  }
\end{exercise}
\begin{solution}
  \enum{
    \item Define \(g_n(x) = \frac{\cos(2^nx)}{2^n}\) and \(M_n = 2^{-n} > \abs{g_n(x)}\). By the Weierstrass M-test, \(g(x)\) converges uniformly on \(\mathbf{R}\). Since each \(g_n(x)\) is continuous and \(g(x)\) converges uniformly, \(g(x)\) must also be continuous.
    \item \(g'_n(x) = -\sin(2^nx)\) and thus \(\sum^\infty_{n=1} g'_n(x)\) does not converge uniformly by Exercise 6.4.2 part a. (It might not converge pointwise either, but that seems more difficult to prove.) Therefore we cannot use Theorem 6.4.3.
  }
\end{solution}
\begin{exercise}
  Define
  $$
  g(x)=\sum_{n=0}^{\infty} \frac{x^{2 n}}{\left(1+x^{2 n}\right)} .
  $$
  Find the values of $x$ where the series converges and show that we get a continuous function on this set.
\end{exercise}
\begin{solution}
  Let \(h_n(x) = \frac{x^{2n}}{(1+x^{2n})}\) be the terms being summed. For \(\abs{x} \geq 1\), \(h_n(x)\) does not approach 0 and therefore the series does not converge. For \(\abs{x} < 1\), \(\abs{h_n(x)} \leq x^{2n}\), which forms a geometric series in \(x^2\), which converges, so \(g(x)\) converges by the Order Limit Theorem.

  Note that for any \(0 \leq a < 1\), \(\abs{h(x)} \leq a^{2n} = M_n\) over \([-a,a]\), and thus by the Weierstrass M-test \(g(x)\) uniformly converges over \([-a, a]\) and is thus continuous over this interval. This last statement is equivalent to saying \(g(x)\) is continuous over \((-1,1)\), which is also the set where \(g(x)\) is well defined.
\end{solution}
\begin{exercise}
  \enum {
  \item Prove that
    $$
    h(x)=\sum_{n=1}^{\infty} \frac{x^{n}}{n^{2}}=x+\frac{x^{2}}{4}+\frac{x^{3}}{9}+\frac{x^{4}}{16}+\cdots
    $$
    is continuous on $[-1,1]$.
  \item The series
    $$
    f(x)=\sum_{n=1}^{\infty} \frac{x^{n}}{n}=x+\frac{x^{2}}{2}+\frac{x^{3}}{3}+\frac{x^{4}}{4}+\cdots
    $$
    converges for every $x$ in the half-open interval $[-1,1)$ but does not converge when $x=1$. For a fixed $x_{0} \in(-1,1)$, explain how we can still use the Weierstrass M-Test to prove that $f$ is continuous at $x_{0}$.
  }
\end{exercise}
\begin{solution}
\enum{
    \item For \(x \in [-1, 1]\), we have
    \[\abs{\frac{x^n}{n^2}} \leq \frac{1}{n^2} = M_n \]
    and since \(\sum \frac{1}{n^2}\) converges (Example 2.4.4), \(h\) converges uniformly and is therefore continuous.
    \item Given a fixed \(x_0\), we can consider the interval \((-a, a) \subset [-1, 1)\) where \(-1 < -a < \abs{x_0} < a < 1\). Then by setting \(M_n = \frac{a^n}{n}\) we will have \(M_n > \frac{x_0^n}{n}\) in a neighbourhood around \(x_0\), allowing us to show via the M-Test that \(f\) is continuous at \(x_0\).
}
\end{solution}

\begin{exercise}
  Let
  $$
  f(x)=\frac{1}{x}-\frac{1}{x+1}+\frac{1}{x+2}-\frac{1}{x+3}+\frac{1}{x+4}-\cdots .
  $$
  Show $f$ is defined for all $x>0$. Is $f$ continuous on $(0, \infty)$ ? How about differentiable?
\end{exercise}
\begin{solution}
 \(f(x)\) converges for any \(x > 0\) by the Alternating Series Test. Since \(f\) converges we are free to use associativity to group the terms in pairs, from which we get
 \[ \begin{aligned}
    f(x) &= \left(\frac{1}{x}-\frac{1}{x+1}\right) + \left(\frac{1}{x+2}-\frac{1}{x+3}\right)+\cdots \\
    &= \frac{1}{x^2 + x} + \frac{1}{(x+2)^2 + (x+2)}+ \cdots \\
    &< \frac{1}{x^2} + \frac{1}{(x+2)^2} + \cdots
 \end{aligned}
    \]
\end{solution}
Temporarily skipping the first term, we can use the Weierstrass M-Test with \(M_n = \frac{1}{(2n)^2}\), to show that \(f(x) - 1/x^2\) converges uniformly and is therefore continuous for \(x>0\). Since \(1/x^2\) is also continuous for \(x > 0\), \(f\) must be continuous on \((0, \infty)\).

Letting \(f'_n\) represent each term of \(f\),
\[\abs{f'_n(x)} = \abs{\frac{(-1)^{n}}{(x+n - 1)^2}} \leq \frac{1}{(n-1)^2}\]
(with the inequality only being meaningful for \(n \geq 2\)). Thus if we skip the first term, by the Weierstrass M-test we can be assured
\[\sum^\infty_{n=2}f'_n(x)\]
converges uniformly, and is therefore differentiable, and is equal to the derivative of
\[g(x) = \sum^\infty_{n=2}f_n(x)\]
by the Differentiable Limit Theorem. Since \(f(x) = g(x) + 1/x\) and both \(1/x\) and \(g(x)\) are differentiable over \((0, \infty)\), we have that \(f\) is differentiable as well.
\begin{exercise}
  Let
  $$
  f(x)=\sum_{k=1}^{\infty} \frac{\sin (k x)}{k^{3}}
  $$
  \enum {
  \item Show that $f(x)$ is differentiable and that the derivative $f^{\prime}(x)$ is contimuous.
  \item Can we determine if $f$ is twice-differentiable?
  }
\end{exercise}
\begin{solution}
 \enum{
    \item Let \(f_n(x) = \frac{\sin(nx)}{n^3}\). We have
    \[\abs{f'_n(x)} = \abs{\frac{\cos (nx)}{n^2}} \leq \frac{1}{n^2}\]
    and so \(\sum^\infty_n f'_n(x)\) converges uniformly by the Weierstrass M-Test. We also have \(f(x)\) converging at \(x=0\) (since every term is zero), so by the differentiable limit theorem we have \(f(x)\) differentiable with \(f'(x) = \sum^\infty_{n=1} f'_n(x)\). Since this converges uniformly, \(f'(x)\) is continuous.

    \item Probably not easily - trying the same trick leaves us with trying to bound \(\abs{\frac{\sin(kx)}{k}}\) with \(M_n\) where \(\sum M_n\) converges, but \(M_n = 1/k\) doesn't work as \(\sum^\infty_{k=1} 1/k\) diverges.
 }
\end{solution}
\begin{exercise}
  Consider the function
  $$
  f(x)=\sum_{k=1}^{\infty} \frac{\sin (x / k)}{k} .
  $$
  Where is $f$ defined? Continuous? Differentiable? Twice-differentiable?
\end{exercise}
\begin{solution}
  \TODO
\end{solution}
\begin{exercise}
  Let
  $$
  h(x)=\sum_{n=1}^{\infty} \frac{1}{x^{2}+n^{2}}
  $$
  \enum {
  \item Show that $h$ is a continuous function defined on all of $\mathbf{R}$.
  \item Is $h$ differentiable? If so, is the derivative function $h^{\prime}$ continuous?
  }
\end{exercise}
\begin{solution}
  \TODO
\end{solution}
\begin{exercise}
  Let $\left\{r_{1}, r_{2}, r_{3}, \ldots\right\}$ be an enumeration of the set of rational numbers. For each $r_{n} \in \mathbf{Q}$, define
  $$
  u_{n}(x)= \begin{cases}1 / 2^{n} & \text { for } x>r_{n} \\ 0 & \text { for } x \leq r_{n} .\end{cases}
  $$
  Now, let $h(x)=\sum_{n=1}^{\infty} u_{n}(x)$. Prove that $h$ is a monotone function defined on all of $\mathbf{R}$ that is continuous at every irrational point.
\end{exercise}
\begin{solution}
  \TODO
\end{solution}
