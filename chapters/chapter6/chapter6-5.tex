\section{Power Series}

\begin{exercise}
Consider the function $g$ defined by the power series
$$
g(x)=x-\frac{x^{2}}{2}+\frac{x^{3}}{3}-\frac{x^{4}}{4}+\frac{x^{5}}{5}-\cdots .
$$
\enum{
\item Is $g$ defined on $(-1,1)$ ? Is it continuous on this set? Is $g$ defined on $(-1,1]$ ? Is it continuous on this set? What happens on $[-1,1]$ ? Can the power series for $g(x)$ possibly converge for any other points $|x|>1$ ? Explain.
\item For what values of $x$ is $g^{\prime}(x)$ defined? Find a formula for $g^{\prime}$.
}
\end{exercise}

\begin{solution}
\enum{
    \item \(g(1)\) converges by the Alternating Series Test, so the radius of convergence is at least 1, and \(g\) must be defined on at least \((-1, 1]\). Theorem 6.5.1 and Abel's Theorem together indicate that indicate that \(g\) converges absolutely on \((-1,1]\) as well. Thus, since each term is continuous, \(g(x)\) is continuous on \((-1,1]\).

    \(g\) is not defined at \(-1\) since \(g(-1)\) would otherwise be
    \[\sum^\infty_{n=1} \frac{-1}{n}\]
    which diverges.

    \(g\) cannot converge at any point \(|x| > 1\) because if it did, that would imply the radius of convergence is strictly larger than 1, and thus \(g\) would need to converge at \(-1\), which it doesn't.

    \item \(g'(x)\) is at least defined on \((-1, 1)\), by Theorem 6.5.7, with the derivative given by
    \[g'(x) = \sum^\infty_{n=0}(-x)^n = \frac{1}{x + 1}\]
    \(g'(x)\) cannot be defined at \(x \leq -1\) since \(g\) isn't even defined there. To show that \(g'(1)\) is defined and is also given by this formula requires a bit more care, since the infinite sum does not actually converge for \(1\). We return to the definition of the derivative:
    \[ \begin{aligned}
g'(1) &= \lim_{x \to 1} \frac{\sum^\infty_{n=1} \frac{(-1)^{n+1}}{n} - \sum^\infty_{n=1} \frac{(-1)^{n+1}}{n}x^n}{1-x} = \lim_{x \to 1} \sum^\infty_n \frac{(-1)^{n+1}}{n} \frac{1-x^n}{1-x} \\
&= \lim_{x \to 1} \frac{1}{1-x}\sum^\infty_n \frac{(-1)^{n+1}}{n} \left(1 - x^n\right)
    \end{aligned}\]
With some algebra, we can show that this converges by the alternating series test, keeping in mind that we can assume \(x \in (0,1)\). We have \(\frac{1-x^n}{n} < \frac{1}{n} \to 0\), so we just need to show \(\frac{1-x^n}{n} \geq \frac{1 - x^{n+1}}{n+1}\):
    \[ \begin{aligned}
   \frac{1-x^n}{n} \geq \frac{1 - x^{n+1}}{n+1} & \Longleftrightarrow (1-x^n)(n+1) \geq n - nx^{n+1} \\
   & \Longleftrightarrow n - nx^n + 1 - x^n \geq n - nx^{n+1} \\
   & \Longleftrightarrow 1 - x^n \geq nx^n (1-x) \\
   & \Longleftrightarrow \frac{1-x^n}{1-x} = \sum^{n-1}_{i=0} x^i \geq \sum^{n-1}_{i=0} x^n = nx^n
    \end{aligned}\]
}

Now we know that \(g'(1)\) exists. We can show that \(g'(1) = \frac{1}{1+1} = 0.5\) by noting that \(\frac{1}{x+1}\) is strictly decreasing on \([0, 1)\), so in order for the derivative \(g'(x)\) to maintain the intermediate value property, \(g'(1) = 0.5\).

\end{solution}

\begin{exercise}
Find suitable coefficients $\left(a_{n}\right)$ so that the resulting power series $\sum a_{n} x^{n}$ has the given properties, or explain why such a request is impossible.

\enum{
 \item Converges for every value of $x \in \mathbf{R}$.
 \item Diverges for every value of $x \in \mathbf{R}$.
 \item Converges absolutely for all $x \in[-1,1]$ and diverges off of this set.
 \item Converges conditionally at $x=-1$ and converges absolutely at $x=1$.
 \item Converges conditionally at both $x=-1$ and $x=1$.
}
\end{exercise}

\begin{solution}
    \enum{
        \item \(a_n = 0\)
        \item Impossible as \(x = 0\) will always converge
        \item \(a_n = \frac{1}{n^2}\). For \(x=1\) this converges, while for \(x > 1\) the series diverges because
\[\frac{x^n}{n^2} <\frac{x^{2n}}{4n^2} \Longleftrightarrow 4 < a^n\]
meaning that once \(n > \log_x(4) = \ln(4) / \ln (x)\), the terms will start increasing (whereas they must approach 0 for the series to converge). A similar argument can be made for \(x < -1\).
        \item Impossible because \(\abs{a_n x^n} = \abs{a_n (-x)^n}\), and substituting \(x = 1\) shows that the series at \(-1\) is going to be the same as that at \(1\) considered absolutely.
        \item \(a_n = 0\) for odd \(n\) and \(a_n = (-1)^{n/2}/n\) for even \(n\). This in effect takes only the even-powered terms of the power series, which are always positive. We then get the alternating harmonic series (scaled by 0.5) in \(x^2\) which diverges absolutely but converges conditionally.
    }
\end{solution}



Exercise 6.5.3. Use the Weierstrass M-Test to prove Theorem $6.5 .2$.
Exercise $6.5 .4$ (Term-by-term Antidifferentiation). Assume $f(x)=$ $\sum_{n=0}^{\infty} a_{n} x^{n}$ converges on $(-R, R)$.
(a) Show
$$
F(x)=\sum_{n=0}^{\infty} \frac{a_{n}}{n+1} x^{n+1}
$$
is defined on $(-R, R)$ and satisfies $F^{\prime}(x)=f(x)$.
(b) Antiderivatives are not unique. If $g$ is an arbitrary function satisfying $g^{\prime}(x)=f(x)$ on $(-R, R)$, find a power series representation for $g$.

Exercise 6.5.5. (a) If $s$ satisfies $0<s<1$, show $n s^{n-1}$ is bounded for all $n \geq 1$.
(b) Given an arbitrary $x \in(-R, R)$, pick $t$ to satisfy $|x|<t<R$. Use this start to construct a proof for Theorem 6.5.6.

Exercise 6.5.6. Previous work on geometric series (Example 2.7.5) justifies the formula
$$
\frac{1}{1-x}=1+x+x^{2}+x^{3}+x^{4}+\cdots, \quad \text { for all }|x|<1 .
$$
Use the results about power series proved in this section to find values for $\sum_{n=1}^{\infty} n / 2^{n}$ and $\sum_{n=1}^{\infty} n^{2} / 2^{n}$. The discussion in Section $6.1$ may be helpful.
Exercise 6.5.7. Let $\sum a_{n} x^{n}$ be a power series with $a_{n} \neq 0$, and assume
$$
L=\lim _{n \rightarrow \infty}\left|\frac{a_{n+1}}{a_{n}}\right|
$$
exists.
(a) Show that if $L \neq 0$, then the series converges for all $x$ in $(-1 / L, 1 / L)$. (The advice in Exercise 2.7.9 may be helpful.)
(b) Show that if $L=0$, then the series converges for all $x \in \mathbf{R}$.
(c) Show that (a) and (b) continue to hold if $L$ is replaced by the limit.
$$
L^{\prime}=\lim _{n \rightarrow \infty} s_{n} \quad \text { where } \quad s_{n}=\sup \left\{\left|\frac{a_{k+1}}{a_{k}}\right|: k \geq n\right\} .
$$
(General properties of the limit superior are discussed in Exercise 2.4.7.)
Exercise 6.5.8. (a) Show that power series representations are unique. If we have
$$
\sum_{n=0}^{\infty} a_{n} x^{n}=\sum_{n=0}^{\infty} b_{n} x^{n}
$$
for all $x$ in an interval $(-R, R)$, prove that $a_{n}=b_{n}$ for all $n=0,1,2, \ldots$
(b) Let $f(x)=\sum_{n=0}^{\infty} a_{n} x^{n}$ converge on $(-R, R)$, and assume $f^{\prime}(x)=f(x)$ for all $x \in(-R, R)$ and $f(0)=1$. Deduce the values of $a_{n}$.

Exercise 6.5.9. Review the definitions and results from Section $2.8$ concerning products of series and Cauchy products in particular. At the end of Section 2.9, we mentioned the following result: If both $\sum a_{n}$ and $\sum b_{n}$ converge conditionally to $A$ and $B$ respectively, then it is possible for the Cauchy product,
$$
\sum d_{n} \quad \text { where } \quad d_{n}=a_{0} b_{n}+a_{1} b_{n-1}+\cdots+a_{n} b_{0}
$$
to diverge. However, if $\sum d_{n}$ does converge, then it must converge to $A B$. To prove this, set
$$
f(x)=\sum a_{n} x^{n}, \quad g(x)=\sum b_{n} x^{n}, \quad \text { and } \quad h(x)=\sum d_{n} x^{n} .
$$
Use Abel's Theorem and the result in Exercise $2.8 .7$ to establish this result.
Exercise 6.5.10. Let $g(x)=\sum_{n=0}^{\infty} b_{n} x^{n}$ converge on $(-R, R)$, and assume $\left(x_{n}\right) \rightarrow 0$ with $x_{n} \neq 0$. If $g\left(x_{n}\right)=0$ for all $n \in \mathbf{N}$, show that $g(x)$ must be identically zero on all of $(-R, R)$.

Exercise 6.5.11. A series $\sum_{n=0}^{\infty} a_{n}$ is said to be Abel-summable to $L$ if the power series
$$
f(x)=\sum_{n=0}^{\infty} a_{n} x^{n}
$$
converges for all $x \in[0,1)$ and $L=\lim _{x \rightarrow 1-} f(x)$.
(a) Show that any series that converges to a limit $L$ is also Abel-summable to $L$.
(b) Show that $\sum_{n=0}^{\infty}(-1)^{n}$ is Abel-summable and find the sum.
