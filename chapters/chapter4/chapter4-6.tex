\section{Sets of Discontinuity}


\begin{exercise}
  Using modifications of Dirichlet and Thomae's functions, construct a function $f: \mathbf{R} \rightarrow \mathbf{R}$ so that
  \enum{
  \item $D_{f}=\mathbf{Z}^{c}$.
  \item $D_{f}=\{x: 0<x \leq 1\}$.
  }
\end{exercise}
\begin{solution}
  \enum{
  \item Modify the function continuous only at zero to be continuous around integers.
    $$
    f(x) = \begin{cases}
      x - [[x]] &\text{if $x \in \mathbf{Q}$} \\
      0 &\text{if $x \in \mathbf{I}$}
    \end{cases}
    $$
  \item Just $1/x$ with any value for $0$
    $$
    g(x) = \begin{cases}
      1/x &\text{if $x \ne 0$} \\
      0   &\text{if $x = 0$}
    \end{cases}
    $$
  }
\end{solution}

\begin{exercise}
  Given a countable set $A=\left\{a_{1}, a_{2}, a_{3}, \ldots\right\}$, define $f\left(a_{n}\right)=1 / n$ and $f(x)=0$ for all $x \notin A$. Find $D_{f}$.
\end{exercise}
\begin{solution}
  To find $D_f$ consider $x \in D_f$ for the two cases $x \in A$ and $x \notin A$.
  If $x \in A$ then $f$ is not continuous, since $f(x) > 0$ but for any $\delta > 0$ we can find $y \in V_\delta(x)$ with $y \notin A$ (because $A$ is countable, $A^c$ must be dense) hence there is an unavoidable error of $|f(x)-f(y)| = f(x) > 0$.

  Now consider $x \notin A$, using the sequential criterion for continuity notice every sequence $(x_n) \to x$ has $f(x_n) \to 0$ (since if $x_n \in A$ converge to $0$ as $n \to \infty$, and $x_n \notin A$ are always $0$), now since $f(x)=0$ this shows $f$ is continuous at $x \notin A$.

  Together we've shown $D_f = A$. Setting $A = \mathbf{Q}$ and using a particular ordering recovers Thomae's function. Hence we can view this as a generalization of Thomae's function for arbitrary countable sets.
\end{solution}

\begin{exercise}
  State a similar definition for the left-hand limit
  $$
  \lim _{x \rightarrow c^{-}} f(x)=L
  $$
\end{exercise}
\begin{solution}
  For all $\epsilon > 0$ there exists a $\delta > 0$ such that $0 < c-x < \delta$ implies $|f(x)-L|<\epsilon$.
\end{solution}

\begin{exercise}
  Given $f: A \rightarrow \mathbf{R}$ and a limit point $c$ of $A, \lim _{x \rightarrow c} f(x)=L$ if and only if
  $$
  \lim _{x \rightarrow c^{-}} f(x)=L \quad \text { and } \quad \lim _{x \rightarrow c^{+}} f(x)=L .
  $$
  Supply a proof for this proposition.
\end{exercise}
\begin{solution}
  Let $\epsilon > 0$, pick $\delta_1$ so $0 < x-c < \delta_1$ implies $|f(x)-L|<\epsilon$, pick $\delta_2$ so $0 < c-x < \delta_2$ implies $|f(x)-L|<\epsilon$. Finally, set $\delta = \min\{\delta_1, \delta_2\}$ to get $|f(x)-L|<\epsilon$ when $0 < |x-c| < \delta$, as desired.
\end{solution}

\begin{exercise}
  Prove that the only type of discontinuity a monotone function can have is a jump discontinuity.
\end{exercise}
\begin{solution}
  \TODO
\end{solution}

\begin{exercise}
  Construct a bijection between the set of jump discontinuities of a monotone function $f$ and a subset of $\mathbf{Q}$. Conclude that $D_{f}$ for a monotone function $f$ must either be finite or countable, but not uncountable.
\end{exercise}
\begin{solution}
  \TODO
\end{solution}

\begin{exercise}
  In Section 4.1 we constructed functions where the set of discontinuity was $\mathbf{R}$ (Dirichlet's function), $\mathbf{R} \backslash\{0\}$ (modified Dirichlet function), and $\mathbf{Q}$ (Thomae's function).

  \enum{
  \item Show that in each of the above cases we get an $F_{\sigma}$ set as the set where the function is discontinuous.
  \item Show that the two sets of discontinuity in Exercise $4.6 .1$ are $F_{\sigma}$ sets.
  }
\end{exercise}
\begin{solution}
  \TODO
\end{solution}

\begin{exercise}
  Prove that, for a fixed $\alpha>0$, the set $D_{f}^{\alpha}$ is closed.
\end{exercise}
\begin{solution}
  \TODO
\end{solution}

\begin{exercise}
  If $\alpha<\alpha^{\prime}$, show that $D_{f}^{\alpha^{\prime}} \subseteq D_{f}^{\alpha}$.
\end{exercise}
\begin{solution}
  \TODO
\end{solution}

\begin{exercise}
  Let $\alpha>0$ be given. Show that if $f$ is continuous at $x$, then it is $\alpha$-continuous at $x$ as well. Explain how it follows that $D_{f}^{\alpha} \subseteq D_{f}$.
\end{exercise}
\begin{solution}
  \TODO
\end{solution}

\begin{exercise}
  Show that if $f$ is not continuous at $x$, then $f$ is not $\alpha$-continuous for some $\alpha>0$. Now explain why this guarantees that
  $$
  D_{f}=\bigcup_{n=1}^{\infty} D_{f}^{\alpha_{n}}
  $$
  where $\alpha_{n}=1 / n$.
\end{exercise}
\begin{solution}
  \TODO
\end{solution}
