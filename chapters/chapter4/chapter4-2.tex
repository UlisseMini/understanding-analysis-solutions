\section{Functional Limits}

\begin{exercise}
  \enum{
  \item Supply the details for how Corollary $4.2 .4$ part (ii) follows from the Sequential Criterion for Functional Limits in Theorem $4.2 .3$ and the Algebraic Limit Theorem for sequences proved in Chapter $2 .$
  \item Now, write another proof of Corollary $4.2 .4$ part (ii) directly from Definition $4.2 .1$ without using the sequential criterion in Theorem $4.2 .3$.
  \item Repeat (a) and (b) for Corollary $4.2 .4$ part (iii).
  }

\end{exercise}

\begin{solution}
  \enum{
  \item \TODO
  \item \TODO
  \item \TODO
  }
\end{solution}

\begin{exercise}
  For each stated limit, find the largest possible $\delta$-neighborhood that is a proper response to the given $\epsilon$ challenge.
  \enum{
  \item $\lim _{x \rightarrow 3}(5 x-6)=9$, where $\epsilon=1$.
  \item $\lim _{x \rightarrow 4} \sqrt{x}=2$, where $\epsilon=1$.
  \item $\lim _{x \rightarrow \pi}[[x]]=3$, where $\epsilon=1$. (The function $[[x]]$ returns the greatest integer less than or equal to $x$.)
  \item $\lim _{x \rightarrow \pi}[[x]]=3$, where $\epsilon=.01$.
  }
\end{exercise}

\begin{solution}
  \enum{
  \item \TODO
  \item \TODO
  \item \TODO
  \item \TODO
  }
\end{solution}

\begin{exercise}
  Review the definition of Thomae's function $t(x)$ from Section 4.1.
  \enum{
  \item Construct three different sequences $\left(x_{n}\right),\left(y_{n}\right)$, and $\left(z_{n}\right)$, each of which converges to 1 without using the number 1 as a term in the sequence.
  \item Now, compute $\lim t\left(x_{n}\right), \lim t\left(y_{n}\right)$, and $\lim t\left(z_{n}\right)$.
  \item Make an educated conjecture for $\lim _{x \rightarrow 1} t(x)$, and use Definition $4.2 .1 \mathrm{~B}$ to verify the claim. (Given $\epsilon>0$, consider the set of points $\{x \in \mathbf{R}: t(x) \geq \epsilon\}$ Argue that all the points in this set are isolated.)
  }
\end{exercise}

\begin{solution}
  \enum {
  \item \TODO
  \item \TODO
  \item \TODO
  }
\end{solution}

\begin{exercise}
  Consider the reasonable but erroneous claim that
  $$
  \lim _{x \rightarrow 10} 1 /[[x]]=1 / 10
  $$
  \enum{
  \item Find the largest $\delta$ that represents a proper response to the challenge of $\epsilon=1 / 2$
  \item Find the largest $\delta$ that represents a proper response to $\epsilon=1 / 50$.
  \item Find the largest $\epsilon$ challenge for which there is no suitable $\delta$ response possible.
  }
\end{exercise}

\begin{solution}
  \enum{
  \item \TODO
  \item \TODO
  \item \TODO
  }
\end{solution}

\begin{exercise}
  Use Definition 4.2.1 to supply a proper proof for the following limit statements.
  \enum{
  \item $\lim _{x \rightarrow 2}(3 x+4)=10$
  \item $\lim _{x \rightarrow 0} x^{3}=0$
  \item $\lim _{x \rightarrow 2}\left(x^{2}+x-1\right)=5$.
  \item $\lim _{x \rightarrow 3} 1 / x=1 / 3$
  }
\end{exercise}

\begin{solution}
  \enum{
  \item \TODO
  \item \TODO
  \item \TODO
  \item \TODO
  }
\end{solution}


\begin{exercise}
  Decide if the following claims are true or false, and give short justifications for each conclusion.
  \enum{
  \item If a particular $\delta$ has been constructed as a suitable response to a particular $\epsilon$ challenge, then any smaller positive $\delta$ will also suffice.
  \item If $\lim _{x \rightarrow a} f(x)=L$ and $a$ happens to be in the domain of $f$, then $L=f(a)$
  \item If $\lim _{x \rightarrow a} f(x)=L$, then $\lim _{x \rightarrow a} 3[f(x)-2]^{2}=3(L-2)^{2}$
  \item If $\lim _{x \rightarrow a} f(x)=0$, then $\lim _{x \rightarrow a} f(x) g(x)=0$ for any function $g$ (with domain equal to the domain of $f$.)
  }
\end{exercise}

\begin{solution}
  \enum{
  \item \TODO
  \item \TODO
  \item \TODO
  \item \TODO
  }
\end{solution}

\begin{exercise}
  Let $g: A \rightarrow \mathbf{R}$ and assume that $f$ is a bounded function on $A$ in the sense that there exists $M>0$ satisfying $|f(x)| \leq M$ for all $x \in A$.
Show that if $\lim _{x \rightarrow c} g(x)=0$, then $\lim _{x \rightarrow c} g(x) f(x)=0$ as well.
\end{exercise}

\begin{solution}
  \TODO
\end{solution}

\begin{exercise}
  Compute each limit or state that it does not exist. Use the tools developed in this section to justify each conclusion.
  \enum{
  \item $\lim _{x \rightarrow 2} \frac{|x-2|}{x-2}$
  \item $\lim _{x \rightarrow 7 / 4} \frac{|x-2|}{x-2}$
  \item $\lim _{x \rightarrow 0}(-1)^{[[1 / x]]}$
  \item $\lim _{x \rightarrow 0} \sqrt[3]{x}(-1)^{[[1 / x]]}$
  }
\end{exercise}

\begin{solution}
  \enum{
  \item \TODO
  \item \TODO
  \item \TODO
  \item \TODO
  }
\end{solution}

