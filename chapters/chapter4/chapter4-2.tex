\section{Functional Limits}

\begin{exercise}
  \enum{
  \item Supply the details for how Corollary 4.2.4 part (ii) follows from the Sequential Criterion for Functional Limits in Theorem 4.2.3 and the Algebraic Limit Theorem for sequences proved in Chapter 2.
  \item Now, write another proof of Corollary 4.2.4 part (ii) directly from Definition 4.2.1 without using the sequential criterion in Theorem 4.2.3.
  \item Repeat (a) and (b) for Corollary 4.2.4 part (iii).
  }

\end{exercise}

\begin{solution}
  \enum{
  \item Assuming $\lim_{x\to c} f(x) = L$ and $\lim_{x\to c} g(x) = M$ we have, by the sequential definition of continuity $f(x_n) \to L$ and $g(x_n) \to M$ for all $(x_n) \to c$ (where every $x_n \ne c$). Hence, we also have $(f + g)(x_n) \to L + M$ (by the algebraic limit theorem for sequences) meaning $\lim_{x \to c}\left[f(x)+g(x)\right] = L+M$ by the sequential definition of continuity (i.e. since this is true \emph{for all} sequences $(x_n)$).
  \item Let $\epsilon > 0$, set $\delta_1$ such that $0<|x-c|<\delta_1$ implies $|f(x)-f(c)|<\epsilon/2$ and set $\delta_2$ such that $0<|x-c|<\delta_2$ implies $|g(x)-g(c)|<\epsilon/2$. Now let $\delta = \min\{\delta_1,\delta_2\}$ and use the triangle inequality to get
    $$
    |f(x)+g(x) - f(c) + g(c)| \le |f(x)-f(c) | + |g(x)-g(c)| < \epsilon/2 + \epsilon/2 = \epsilon
    $$
    for all $0<|x-c|<\delta$.
  \item (a) is the same, if $f(x_n) \to L$ and $g(x_n) \to M$ then $(f(x_n)g(x_n)) \to LM$ by the sequential criterion for functional limits.

    For (b) we add and subtract $f(c)g(x)$ then factor and use the triangle inequality (this is a common trick)
    $$
    \begin{aligned}
      |f(x)g(x) - f(c)g(c)| &= |g(x)(f(x)-f(c)) - f(c)(g(c)-g(x))| \\
                            &\le |g(x)| |f(x)-f(c)| - |f(c)| |g(c)-g(x)|
    \end{aligned}
    $$
    Now we want a few things, (1) to bound $|g(x)|$ (2) to make $|f(x)-f(c)|$ small and (3) to make $|g(x)-g(c)|$ small. Whenever you want multiple things start thinking min/max!

    In this case, set $\delta_1$ so $|g(x)-g(c)|<1$ giving the bound $|g(x)| < M+1$. Set $\delta_2$ so $|g(x)-g(c)|<\frac{\epsilon/2}{M+1}$ and set $\delta_3$ so $|f(x)-f(c)|< \frac{\epsilon/2}{f(c)}$. Finally set $\delta = \min\{\delta_1,\delta_2,\delta_3\}$ to get
    $$
    |f(x)g(x) - f(c)g(c)| < \epsilon/2 + \epsilon/2 = \epsilon
    $$
  }
\end{solution}

\begin{exercise}
  For each stated limit, find the largest possible $\delta$-neighborhood that is a proper response to the given $\epsilon$ challenge.
  \enum{
  \item $\lim _{x \rightarrow 3}(5 x-6)=9$, where $\epsilon=1$.
  \item $\lim _{x \rightarrow 4} \sqrt{x}=2$, where $\epsilon=1$.
  \item $\lim _{x \rightarrow \pi}[[x]]=3$, where $\epsilon=1$. (The function $[[x]]$ returns the greatest integer less than or equal to $x$.)
  \item $\lim _{x \rightarrow \pi}[[x]]=3$, where $\epsilon=.01$.
  }
\end{exercise}

\begin{solution}
  \enum{
  \item $|(5x-6)-9| = |5x-15| = 5|x-3| < 5\delta$ implies $\delta = 1/5$ for $\epsilon=1$.
  \item Consider edge cases: We have $|\sqrt{9} - 2| = 1$ ($x$ is $5$ above) and $|\sqrt{1} - 2| = 1$ ($x$ is $3$ below) leading us to set $\delta = 3$. This $\delta$ must work since $\sqrt{x}$ is monotone.
  \item We want $x \in (2, 5)$ as that is the largest neighborhood with $|[[x]]-3| < 1$. Set $\delta = \min\{|\pi-2|, |\pi-5|\} = \pi-2$ to get this maximum neighborhood.
  \item Since $[[x]]$ is an integer $\epsilon = .01$ is the same as saying $[[x]] = 3$. This happens precisely when $x \in (3,4)$ hence we need $\delta = \min\{|\pi-3|, |\pi-4|\} = \pi-3$.
  }
\end{solution}

\begin{exercise}
  Review the definition of Thomae's function $t(x)$ from Section 4.1.
  $$
  t(x)= \begin{cases}1 & \text { if } x=0 \\ 1 / n & \text { if } x=m / n \in \mathbf{Q} \backslash\{0\} \text { is in lowest terms with } n>0 \\ 0 & \text { if } x \notin \mathbf{Q} .\end{cases}
  $$

  \enum{
  \item Construct three different sequences $\left(x_{n}\right),\left(y_{n}\right)$, and $\left(z_{n}\right)$, each of which converges to 1 without using the number 1 as a term in the sequence.
  \item Now, compute $\lim t\left(x_{n}\right), \lim t\left(y_{n}\right)$, and $\lim t\left(z_{n}\right)$.
  \item Make an educated conjecture for $\lim _{x \rightarrow 1} t(x)$, and use Definition $4.2 .1 \mathrm{~B}$ to verify the claim. (Given $\epsilon>0$, consider the set of points $\{x \in \mathbf{R}: t(x) \geq \epsilon\}$ Argue that all the points in this set are isolated.)
  }
\end{exercise}

\begin{solution}
  \enum {
  \item $x_n = (1+n)/n$, $y_n = 1 - 1/n^2$ and $z_n = 1 + 1/2^n$.
  \item $\lim t(x_n) = 0$ since the size of the denominator becomes arbitrarily large. Same for the others
  \item I claim $\lim_{x \to 1} t(x) = 0$. let $\epsilon > 0$ be arbitrary, we must show there exists a $\delta$ where every $|x| < \delta$ has $t(x) < \epsilon$. for $x \notin \mathbf{Q}$ we have $t(x) = 0 < \epsilon$, and we can easily set $\delta$ small enough that $t(0)=1$ is excluded. That leaves us with the case $x \in \mathbf{Q}$ in which case we can write $x = m/n$ in lowest terms.

    To get $t(x) = 1/n < \epsilon$ we observe that $|m/n| < \delta$ implies $t(x) = 1/n \le |m/n| < \delta$ so setting $\delta = \epsilon$ gives $t(x) < \epsilon$. To complete the proof set $\delta = \min\{\epsilon, 1\}$
  }
\end{solution}

\begin{exercise}
  Consider the reasonable but erroneous claim that
  $$
  \lim _{x \rightarrow 10} 1 /[[x]]=1 / 10
  $$
  \enum{
  \item Find the largest $\delta$ that represents a proper response to the challenge of $\epsilon=1 / 2$
  \item Find the largest $\delta$ that represents a proper response to $\epsilon=1 / 50$.
  \item Find the largest $\epsilon$ challenge for which there is no suitable $\delta$ response possible.
  }
\end{exercise}

\begin{solution}
  \enum{
  \item \TODO
  \item \TODO
  \item No matter how small $\delta$ is, $[[10-\delta/2]] = 9$ can be obtained, meaning
    $$
    |1/9 - 1/10| = 1/90
    $$
    Is the largest $\epsilon$ with no suitable $\delta$ response.
  }
\end{solution}

\begin{exercise}
  Use Definition 4.2.1 to supply a proper proof for the following limit statements.
  \enum{
  \item $\lim _{x \rightarrow 2}(3 x+4)=10$
  \item $\lim _{x \rightarrow 0} x^{3}=0$
  \item $\lim _{x \rightarrow 2}\left(x^{2}+x-1\right)=5$.
  \item $\lim _{x \rightarrow 3} 1 / x=1 / 3$
  }
\end{exercise}

\begin{solution}
  (Note that I use the largest $\delta$ choice that's easy to use)
  \enum{
  \item Since $|3x-6| = 3|x-2|$ setting $\delta = \epsilon/3$ gives $|3x-6|<\epsilon$ as desired.
  \item Since $|x^3| = |x|^3$ setting $\delta = \epsilon^{1/3}$ gives $|x|^3<\epsilon$ as desired.
  \item Since $|x^2+x-6| = |x-2||x+3| < \delta(5 + \delta)$ setting $\delta=\min\{1,\epsilon/6\}$ gives $\delta(5+\delta) < \delta(6) < \epsilon$ as desired.

    Another approach is to write $|x^2+x-6|$ in the ``$(x-2)^n$ basis''
    $$
    |x^2+x-6| = |(x-2)^2 + 5(x-2)| < \delta^2 + 5\delta = \delta(5 + \delta)
    $$
  \item We have $|1/x-1/3| = \frac{|3 - x|}{3|x|}$ setting $\delta = \min\{1,6\epsilon\}$ gives $1/3|x| < 1/6$ (because $|x| \in (2,4)$) and $|x-3|<6\epsilon$ meaning
    $$
    |1/x-1/3| = \frac{|3-x|}{3|x|} < \frac{|3-x|}{6} < \epsilon
    $$
    as desired.
  }
\end{solution}


\begin{exercise}
  Decide if the following claims are true or false, and give short justifications for each conclusion.
  \enum{
  \item If a particular $\delta$ has been constructed as a suitable response to a particular $\epsilon$ challenge, then any smaller positive $\delta$ will also suffice.
  \item If $\lim _{x \rightarrow a} f(x)=L$ and $a$ happens to be in the domain of $f$, then $L=f(a)$
  \item If $\lim _{x \rightarrow a} f(x)=L$, then $\lim _{x \rightarrow a} 3[f(x)-2]^{2}=3(L-2)^{2}$
  \item If $\lim _{x \rightarrow a} f(x)=0$, then $\lim _{x \rightarrow a} f(x) g(x)=0$ for any function $g$ (with domain equal to the domain of $f$.)
  }
\end{exercise}

\begin{solution}
  \enum{
  \item Obviously, since if $\delta' < \delta$ then $|x - a| < \delta'$ implies $|x - a| < \delta$.
  \item False, consider $f(0) = 1$ and $f(x) = 0$ otherwise, the definition of a functional limit requires $|x - a| < \delta$ to imply $|f(x) - L| < \epsilon$ \emph{for all $x$ not equal to $a$} (This is the $0<|x-a|$ part)
  \item True by the algebraic limit theorem for functional limits. (or composition of continuous functions, but that's unnecessary here)
  \item False, consider how $f(x) = x$ has $\lim_{x \to 0} f(x) = 0$ but $g(x) = 1/x$ has $\lim_{n \to 0} f(x)g(x) = 1$. (Fundementally this is because $1/x$ is not continuous at $0$)
  }
\end{solution}

\begin{exercise}
  Let $g: A \rightarrow \mathbf{R}$ and assume that $f$ is a bounded function on $A$ in the sense that there exists $M>0$ satisfying $|f(x)| \leq M$ for all $x \in A$.
Show that if $\lim _{x \rightarrow c} g(x)=0$, then $\lim _{x \rightarrow c} g(x) f(x)=0$ as well.
\end{exercise}

\begin{solution}
  We have $|g(x)f(x)| \le M|g(x)|$, set $\delta$ small enough that $|g(x)| < \epsilon/M$ to get
  $$|g(x)f(x)| \le M|g(x)| < M\frac{\epsilon}{M} = \epsilon$$
  for all $|x - a| < \delta$.
\end{solution}

\begin{exercise}
  Compute each limit or state that it does not exist. Use the tools developed in this section to justify each conclusion.
  \enum{
  \item $\lim _{x \rightarrow 2} \frac{|x-2|}{x-2}$
  \item $\lim _{x \rightarrow 7 / 4} \frac{|x-2|}{x-2}$
  \item $\lim _{x \rightarrow 0}(-1)^{[[1 / x]]}$
  \item $\lim _{x \rightarrow 0} \sqrt[3]{x}(-1)^{[[1 / x]]}$
  }
\end{exercise}

\begin{solution}
  \enum{
  \item Does not exist, the sequence $x_n = 2 + 1/n$ makes $|x-2|/(x-2)$ converge to $1$, but $x_n = 2 - 1/n$ makes $|x-2|/x-2$ converge to $-1$. ($x \to |x|$ is not differentiable at zero for the same reason)
  \item Exists and has limit $1$ since $|x-2|/(x-2) = 1$ for all $x \in (1,1.8)$ (note $7/4 = 1.75$)
  \item Does not exist, the sequence $x_n = 1/n$ gives $[[1/x_n]] = n$ and $\lim_{n\to\infty} (-1)^n$ obviously doesn't exist.
  \item Exists and has limit zero, since $|\sqrt[3]{x}(-1)^{[[1/x]]}| = |\sqrt[3]{x}| < \epsilon$ when $\delta=\epsilon^{3}$
  }
\end{solution}

