\section{Functional Limits}

\begin{exercise}
  \enum{
  \item Supply the details for how Corollary $4.2 .4$ part (ii) follows from the Sequential Criterion for Functional Limits in Theorem $4.2 .3$ and the Algebraic Limit Theorem for sequences proved in Chapter $2 .$
  \item Now, write another proof of Corollary $4.2 .4$ part (ii) directly from Definition $4.2 .1$ without using the sequential criterion in Theorem $4.2 .3$.
  \item Repeat (a) and (b) for Corollary $4.2 .4$ part (iii).
  }

\end{exercise}

\begin{solution}
  \enum{
  \item By the Sequential Criterion for Functional Limits, since \(\lim_{x\to c} f(x) = L\), \(\lim_{x\to c} g(x) = M\), all sequences \((x_n) \to c\) have \(f(x_n) \to L\) and \(g(x_n) \to M\), which implies that \(f(x_n) + g(x_n) \to L + M\) by the Algebraic Limit Theorem, which implies that \(\lim_{x \to c} [f(x) + g(x)] = L + M\).

  \item For arbitrary \(\epsilon > 0\), let \(\epsilon' = \epsilon/2\). Identify \(\delta_f\) so that \(0 < |x - c| < \delta_f \) implies \(|f(x) - L| < \epsilon'\). Identify \(\delta_g\) so that \(0 < |x - c| < \delta_g \) implies \(\abs{g(x) - M} < \epsilon'\).

  Choose \(\delta = \min \{\delta_f, \delta_g\}\), then \(0 < |x-c| < \delta\) implies \[
      \begin{aligned}
      |f(x) + g(x) - (L + M)| &=|f(x) - L + g(x) -  M| \leq |f(x) - L| + |g(x) - M| \\
      &< 2 \epsilon' = \epsilon
      \end{aligned}
    \], as desired.

  \item By the Sequential Criterion for Functional Limits, since \(\lim_{x\to c} f(x)= L\), \(\lim_{x\to c} g(x) = M\), all sequences \((x_n) \to c\) have \(f(x_n) \to L\) and \(g(x_n) \to M\), which implies that \(f(x_n) g(x_n) \to LM\) by the Algebraic Limit Theorem, and \(\lim_{x \to c} [f(x) g(x)] = L M\).

  For arbitrary \(\epsilon > 0\), let \(\epsilon_f = \epsilon / 2M\) and \(\epsilon_g = \epsilon / 2(|L| + \epsilon)\). Identify \(\delta_f\) so that \(0 < |x - c| < \delta_f \) implies \(|f(x) - L| < \epsilon_f\). Identify \(\delta_g\) so that \(0 < |x - c| < \delta_g \) implies \(\abs{g(x) - M} < \epsilon_g\).

  Choose \(\delta = \min \{\delta_f, \delta_g\}\), then \(0 < |x-c| < \delta\) implies
  \[ \begin{aligned}
|f(x)g(x) - LM| &=|f(g-M) + (f-L)M| \leq |f||g-M| + |f-L||M| \\
& < (|L| + \epsilon)\epsilon_g + \epsilon_f |M| = \epsilon/2 + \epsilon/2  = \epsilon
  \end{aligned}\]
  , as desired.
  }
\end{solution}

\begin{exercise}
  For each stated limit, find the largest possible $\delta$-neighborhood that is a proper response to the given $\epsilon$ challenge.
  \enum{
  \item $\lim _{x \rightarrow 3}(5 x-6)=9$, where $\epsilon=1$.
  \item $\lim _{x \rightarrow 4} \sqrt{x}=2$, where $\epsilon=1$.
  \item $\lim _{x \rightarrow \pi}[[x]]=3$, where $\epsilon=1$. (The function $[[x]]$ returns the greatest integer less than or equal to $x$.)
  \item $\lim _{x \rightarrow \pi}[[x]]=3$, where $\epsilon=.01$.
  }
\end{exercise}

\begin{solution}
  \enum{
  \item \(\delta = 1/5\)
  \item \(\delta = 3\)
  \item \(\delta = \pi - 3\)
  \item \(\delta = \pi - 3\)
  }
\end{solution}

\begin{exercise}
  Review the definition of Thomae's function $t(x)$ from Section 4.1.
  $$
  t(x)= \begin{cases}1 & \text { if } x=0 \\ 1 / n & \text { if } x=m / n \in \mathbf{Q} \backslash\{0\} \text { is in lowest terms with } n>0 \\ 0 & \text { if } x \notin \mathbf{Q} .\end{cases}
  $$

  \enum{
  \item Construct three different sequences $\left(x_{n}\right),\left(y_{n}\right)$, and $\left(z_{n}\right)$, each of which converges to 1 without using the number 1 as a term in the sequence.
  \item Now, compute $\lim t\left(x_{n}\right), \lim t\left(y_{n}\right)$, and $\lim t\left(z_{n}\right)$.
  \item Make an educated conjecture for $\lim _{x \rightarrow 1} t(x)$, and use Definition $4.2 .1 \mathrm{~B}$ to verify the claim. (Given $\epsilon>0$, consider the set of points $\{x \in \mathbf{R}: t(x) \geq \epsilon\}$ Argue that all the points in this set are isolated.)
  }
\end{exercise}

\begin{solution}
  \enum {
  \item $x_n = (1+n)/n$, $y_n = 1 - 1/n^2$ and $z_n = 1 + 1/2^n$.
  \item $\lim t(x_n) = 0$ since the size of the denominator becomes arbitrarily large. Same for the others
  \item I claim $\lim_{x \to 1} t(x) = 0$. Let $\epsilon > 0$ be arbitrary; we must show there exists a $\delta$ where every $|x - 1| < \delta$ has $t(x) < \epsilon$. For $x \notin \mathbf{Q}$ we have $t(x) = 0 < \epsilon$, and we can easily set $\delta$ small enough that $t(0)=1$ is excluded. That leaves us with the case $x \in \mathbf{Q}$ in which case we can write $x - 1 = m/n$ in lowest terms.

    To get $t(x) = 1/n < \epsilon$ we observe that $|m/n| < \delta$ implies $t(x) = 1/n \le |m/n| < \delta$ so setting $\delta = \epsilon$ gives $t(x) < \epsilon$. To complete the proof set $\delta = \min\{\epsilon, 1\}$.
  }
\end{solution}

\begin{exercise}
  Consider the reasonable but erroneous claim that
  $$
  \lim _{x \rightarrow 10} 1 /[[x]]=1 / 10
  $$
  \enum{
  \item Find the largest $\delta$ that represents a proper response to the challenge of $\epsilon=1 / 2$
  \item Find the largest $\delta$ that represents a proper response to $\epsilon=1 / 50$.
  \item Find the largest $\epsilon$ challenge for which there is no suitable $\delta$ response possible.
  }
\end{exercise}

\begin{solution}
  \enum{
  \item \TODO
  \item \TODO
  \item \TODO
  }
\end{solution}

\begin{exercise}
  Use Definition 4.2.1 to supply a proper proof for the following limit statements.
  \enum{
  \item $\lim _{x \rightarrow 2}(3 x+4)=10$
  \item $\lim _{x \rightarrow 0} x^{3}=0$
  \item $\lim _{x \rightarrow 2}\left(x^{2}+x-1\right)=5$.
  \item $\lim _{x \rightarrow 3} 1 / x=1 / 3$
  }
\end{exercise}

\begin{solution}
  \enum{
  \item \TODO
  \item \TODO
  \item \TODO
  \item \TODO
  }
\end{solution}


\begin{exercise}
  Decide if the following claims are true or false, and give short justifications for each conclusion.
  \enum{
  \item If a particular $\delta$ has been constructed as a suitable response to a particular $\epsilon$ challenge, then any smaller positive $\delta$ will also suffice.
  \item If $\lim _{x \rightarrow a} f(x)=L$ and $a$ happens to be in the domain of $f$, then $L=f(a)$
  \item If $\lim _{x \rightarrow a} f(x)=L$, then $\lim _{x \rightarrow a} 3[f(x)-2]^{2}=3(L-2)^{2}$
  \item If $\lim _{x \rightarrow a} f(x)=0$, then $\lim _{x \rightarrow a} f(x) g(x)=0$ for any function $g$ (with domain equal to the domain of $f$.)
  }
\end{exercise}

\begin{solution}
  \enum{
  \item Obviously, since if $\delta' < \delta$ then $|x - a| < \delta'$ implies $|x - a| < \delta$.
  \item False, consider $f(0) = 1$ and $f(x) = 0$ otherwise, the definition of a functional limit requires $|x - a| < \delta$ to imply $|f(x) - L| < \epsilon$ \emph{for $x$ not equal to $a$}
  \item True by the algebraic limit theorem for functional limits. (or composition of continuous functions, but that's unnecessary here)
  \item False, consider how $f(x) = x$ has $\lim_{x \to 0} f(x) = 0$ but $g(x) = 1/x$ has $\lim_{n \to 0} f(x)g(x) = 1$. (Fundementally this is because $1/x$ is not continuous at $0$)
  }
\end{solution}

\begin{exercise}
  Let $g: A \rightarrow \mathbf{R}$ and assume that $f$ is a bounded function on $A$ in the sense that there exists $M>0$ satisfying $|f(x)| \leq M$ for all $x \in A$.
Show that if $\lim _{x \rightarrow c} g(x)=0$, then $\lim _{x \rightarrow c} g(x) f(x)=0$ as well.
\end{exercise}

\begin{solution}
  We have $|g(x)f(x)| \le M|g(x)|$, set $\delta$ small enough that $|g(x)| < \epsilon/M$ to get
  $$|g(x)f(x)| \le M|g(x)| < M\frac{\epsilon}{M} = \epsilon$$
  for all $|x - a| < \delta$.
\end{solution}

\begin{exercise}
  Compute each limit or state that it does not exist. Use the tools developed in this section to justify each conclusion.
  \enum{
  \item $\lim _{x \rightarrow 2} \frac{|x-2|}{x-2}$
  \item $\lim _{x \rightarrow 7 / 4} \frac{|x-2|}{x-2}$
  \item $\lim _{x \rightarrow 0}(-1)^{[[1 / x]]}$
  \item $\lim _{x \rightarrow 0} \sqrt[3]{x}(-1)^{[[1 / x]]}$
  }
\end{exercise}

\begin{solution}
  \enum{
  \item \TODO
  \item \TODO
  \item \TODO
  \item \TODO
  }
\end{solution}

