\section{The Intermediate Value Theorem}

\begin{exercise}
  Show how the Intermediate Value Theorem follows as a corollary to preservation of connected sets (Theorem 4.5.2).
\end{exercise}
\begin{solution}
  Let $f$ be a continuous function defined over $[a,b]$. And let $L$ be a real number such that $f(a) < L < f(b)$.

  Define $A = \{x \in [a,b] : f(x) \le L\}$ and $B = \{x \in [a,b] : f(x) > L\}$. $A,B$ are clearly disjoint, they are also nonempty since $a \in A$ and $b \in B$. Since $[a,b]$ is connected there exists a sequence in $A$ or $B$ with limit in the other. Let $(x_n)$ be such a sequence, I claim $(x_n) \to x$ with $f(x) = L$.

  $(x_n)$ cannot be contained in $A$ since $f(x_n) \le L$ implies $f(x) \le L$ and $x \in A$ ($A$ is closed)

  If $(x_n)$ is contained in $B$ then $f(x_n) > L$ implies $f(x) \ge L$ by the Order Limit Theorem. But since $x \in A$ we have $f(x) \le A$ as well. Thus $f(x) = L$ completing the proof.
\end{solution}

\begin{exercise}
  Provide an example of each of the following, or explain why the request is impossible
  \enum{
  \item A continuous function defined on an open interval with range equal to a closed interval.
  \item A continuous function defined on a closed interval with range equal to an open interval.
  \item A continuous function defined on an open interval with range equal to an unbounded closed set different from $\mathbf{R}$.
  \item A continuous function defined on all of $\mathbf{R}$ with range equal to $\mathbf{Q}$.
  }
\end{exercise}
\begin{solution}
  \enum{
  \item Possible, see Exercise 4.4.8 (b)
  \item Impossible by preservation of compact sets
  \item Let $f : (0,1) \to [2, \infty)$ be defined by
    $$
    f(x) = \begin{cases}
      \frac 1x &\text{ if } x \in (0, 1/2] \\
      \frac{1}{1-x} &\text{ if } x \in (1/2, 1)
    \end{cases}
    $$
    This works since $[2,\infty)$ is closed, unbounded and different from $\mathbf{R}$.
  \item Impossible as this contradicts the intermediate value theorem.
  }
\end{solution}

\begin{exercise}
  A function $f$ is increasing on $A$ if $f(x) \leq f(y)$ for all $x<y$ in $A$. Show that if $f$ is increasing on $[a, b]$ and satisfies the intermediate value property (Definition 4.5.3), then $f$ is continuous on $[a, b]$.
\end{exercise}
\begin{solution}
  \TODO
\end{solution}

\begin{exercise}
  Let $g$ be continuous on an interval $A$ and let $F$ be the set of points where $g$ fails to be one-to-one; that is,
  $$
  F=\{x \in A: f(x)=f(y) \text { for some } y \neq x \text { and } y \in A\} \text {. }
  $$
  Show $F$ is either empty or uncountable.
\end{exercise}
\begin{solution}
  % TODO: Cleanup
  Suppose $F$ is nonempty, let $x,y \in A$ with $x \ne y$ and $f(x) = f(y)$. Pick $z \in (x,y)$ such that $f(z) \ne f(x)$ (if $z$ does not exist $f$ is constant over $(x,y)$ and we are finished early). By the Intermediate Value Theorem every $L \in (f(x), f(z))$ has an $x' \in (x, z)$ with $f(x') = L$. And since $L \in (f(z), f(y))$ as well we can find $y' \in (z,y)$ with $f(y') = L$, thus $f(y') = f(x')$ and so $f$ is not 1-1 at every $L \in (f(x), f(z))$ which is uncountable.
\end{solution}

\begin{exercise}
  \enum{
  \item Finish the proof of the Intermediate Value Theorem using the Axiom of Completeness started previously.
  \item Finish the proof of the Intermediate Value Theorem using the Nested Interval Property started previously.
  }
\end{exercise}
\begin{solution}
  \TODO
\end{solution}

\begin{exercise}
  Let $f:[0,1] \rightarrow \mathbf{R}$ be continuous with $f(0)=f(1)$
  \enum{
  \item Show that there must exist $x, y \in[0,1]$ satisfying $|x-y|=1 / 2$ and $f(x)=f(y) .$
  \item Show that for each $n \in \mathbf{N}$ there exist $x_{n}, y_{n} \in[0,1]$ with $\left|x_{n}-y_{n}\right|=1 / n$ and $f\left(x_{n}\right)=f\left(y_{n}\right)$.
  \item If $h \in(0,1 / 2)$ is not of the form $1 / n$, there does not necessarily exist $|x-y|=h$ satisfying $f(x)=f(y)$. Provide an example that illustrates this using $h=2 / 5$.
  }
\end{exercise}
\begin{solution}
  \TODO
\end{solution}

\begin{exercise}
  Let $f$ be a continuous function on the closed interval $[0,1]$ with range also contained in $[0,1]$. Prove that $f$ must have a fixed point; that is, show $f(x)=x$ for at least one value of $x \in[0,1]$.
\end{exercise}
\begin{solution}
  \TODO
\end{solution}

\begin{exercise}[Inverse functions]
  If a function $f: A \rightarrow \mathbf{R}$ is one-to-one, then we can define the inverse function $f^{-1}$ on the range of $f$ in the natural way: $f^{-1}(y)=x$ where $y=f(x)$.

  Show that if $f$ is continuous on an interval $[a, b]$ and one-to-one, then $f^{-1}$ is also continuous.
\end{exercise}
\begin{solution}
  \TODO
\end{solution}
