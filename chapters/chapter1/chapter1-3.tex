\section{The Axiom of Completeness}

\begin{exercise}
  \enum{
  \item Write a formal definition in the style of Definition 1.3.2 for the \emph{infimum} or \emph{greatest lower bound} of a set.
  \item Now, state and prove a version of Lemma 1.3.8 for greatest lower bounds.
  }
\end{exercise}

\begin{solution}
  \enum{
  \item We have $i = \inf A$ if and only if
    \enumr{
      \item Lower bound, $a \ge i$ for all $a \in A$
      \item Greatest lower bound, If $b$ is a lower bound on $A$ then $b \le i$
    }
  \item Suppose $i$ is a lower bound for $A$, it is the greatest lower bound if and only if for all $\epsilon>0$, there exists an $a\in A$ such that $i + \epsilon > a$. \par
    First suppose $i = \inf A$, then for all $\epsilon > 0$, $i+\epsilon$ cannot be a lower bound on $A$ because (ii) implies all lower bounds $b$ obey $b \le i$, therefore there must be some $a \in A$ such that $i+\epsilon > a$.

    Second suppose for all $\epsilon > 0$ there exists an $a \in A$ such that $i+\epsilon > a$. In other words $i+\epsilon$ is not a lower bound for all $\epsilon$, which is the same as saying every lower bound $b$ must have $b \le i$ implying (ii).
  }
\end{solution}


\begin{exercise}
  Give an example of each of the following, or state that the request is impossible.
  \enum{
  \item A set $B$ with inf $B \geq \sup B$.
  \item A finite set that contains its infimum but not its supremum.
  \item A bounded subset of $\mathbf{Q}$ that contains its supremum but not its infimum.
  }
\end{exercise}

\begin{solution}
  \enum{
  \item Let $B = \{0\}$ we have $\inf B = 0$ and $\sup B = 0$ thus $\inf B \ge \sup B$.
  \item Impossible, finite sets must contain their infimum and supremum.
  \item Let $B = \{r \in \mathbf{Q} \mid 1 < r \le 2\}$ we have $\inf B = 1 \notin B$ and $\sup B = 2 \in B$.
  }
\end{solution}

\begin{exercise}
  \enum{
  \item Let $A$ be nonempty and bounded below, and define $B=$ $\{b \in \mathbf{R}: b$ is a lower bound for $A\}$. Show that $\sup B=\inf A$.
  \item Use (a) to explain why there is no need to assert that greatest lower bounds exist as part of the Axiom of Completeness.
  }
\end{exercise}

\begin{solution}
  \enum{
  \item By definition $\sup B$ is the greatest lower bound for $A$, meaning it equals $\inf A$.
  \item (a) Proves the greatest lower bound exists using the least upper bound.
  }
\end{solution}

\begin{exercise}
  Let $A_{1}, A_{2}, A_{3}, \ldots$ be a collection of nonempty sets, each of which is bounded above.
  \enum{
  \item Find a formula for $\sup \left(A_{1} \cup A_{2}\right)$. Extend this to $\sup \left(\bigcup_{k=1}^{n} A_{k}\right)$.
  \item Consider $\sup \left(\bigcup_{k=1}^{\infty} A_{k}\right)$. Does the formula in (a) extend to the infinite case?
  }
\end{exercise}

\begin{solution}
  \enum{
  \item $\sup \left(\bigcup_{k=1}^n A_k\right) = \sup \left\{\sup A_k \mid k=1,\dots,n\right\}$
  \item Yes. Let $S = \left\{\sup A_k \mid k=1,\dots,\right\}$ and $s = \sup S$. $s$ is obviously an upper bound for $\bigcup_{k=1}^\infty A_k$. to see it is the least upper bound suppose $s' < s$, then by definition there exists a $k$ such that $\sup A_k > s'$ implying $s'$ is not an upper bound for $A_k$. therefore $s$ is the least upper bound.
  }
\end{solution}


\begin{exercise}
  As in Example 1.3.7, let $A \subseteq \mathbf{R}$ be nonempty and bounded above, and let $c \in \mathbf{R}$. This time define the set $c A=\{c a: a \in A\}$.
  \enum{
  \item If $c \geq 0$, show that $\sup (c A)=c \sup A$.
  \item Postulate a similar type of statement for $\sup (c A)$ for the case $c<0$.
  }
\end{exercise}

\begin{solution}
  \enum{
  \item Let $s = c \sup A$. Suppose $ca > s$, then $a > \sup A$ which is impossible, meaning $s$ is an upper bound on $cA$. Now suppose $s'$ is an upper bound on $cA$ and $s' < s$. Then $s'/c < s/c$ and $s'/c < \sup A$ meaning $s'/c$ cannot bound $A$, so there exists $a \in A$ such that $s'/c > a$ meaning $s' > ca$ thus $s'$ cannot be an upper bound on $cA$, and so $s = c \sup A$ is the least upper bound.
  \item $\sup(cA) = c \inf(A)$ for $c < 0$
  }
\end{solution}

\begin{exercise}
  Given sets $A$ and $B$, define $A+B=\{a+b: a \in A$ and $b \in B\}$. Follow these steps to prove that if $A$ and $B$ are nonempty and bounded above then $\sup (A+B)=\sup A+\sup B$
  \enum{
  \item Let $s=\sup A$ and $t=\sup B$. Show $s+t$ is an upper bound for $A+B$.
  \item Now let $u$ be an arbitrary upper bound for $A+B$, and temporarily fix $a \in A$. Show $t \leq u-a$.
  \item Finally, show $\sup (A+B)=s+t$.
  \item Construct another proof of this same fact using Lemma 1.3.8.
  }
\end{exercise}

\begin{solution}
  \enum{
  \item We have $a \le s$ and $b \le t$, adding the equations gives $a + b \le s + t$.
  \item $t \le u - a$ should be true since $u - a$ is an upper bound on $b$, meaning it is greater then or equal to the least upper bonud $t$. Formally $a + b \le u$ implies $b \le u - a$ and since $t$ is the least upper bound on $b$ we have $t \le u - a$.
  \item From (a) we know $s + t$ is an upper bound, so we must only show it is the least upper bound. \par
    Let $u = \sup(A+B)$, from (a) we have $t \le u-a$ and $s \le u-b$ adding and rearranging gives $a+b \le 2u - s - t$. since $2u - s - t$ is an upper bound on $A+B$ it is less then the least upper bound, so $u \le 2u - s -t$ implying $s + t \le u$. and since $u$ is the least upper bound $s+t$ must equal $u$. \par
    Stepping back, the key to this proof is that $a + b \le s,\forall a,b$ implying $\sup(A + B) \le s$ can be used to transition from all $a+b$ to a single value $\sup(A+B)$, avoiding the $\epsilon$-hackery I would otherwise use.
  \item Showing $s + t - \epsilon$ is not an upper bound for any $\epsilon > 0$ proves it is the least upper bound by Lemma 1.3.8. Rearranging gives $(s - \epsilon/2) + (t - \epsilon/2)$ we know there exists $a > (s - \epsilon/2)$ and $b > (t - \epsilon/2)$ therefore $a + b > s + t - \epsilon$ meaning $s+t$ cannot be made smaller, and thus is the least upper bound.
  }
\end{solution}

\begin{exercise}
  Prove that if $a$ is an upper bound for $A$, and if $a$ is also an element of $A$, then it must be that $a=\sup A$.
\end{exercise}

\begin{solution}
  $a$ is the least upper bound since any smaller bound $a' < a$ would not bound $a$.
\end{solution}

\begin{exercise}
  Compute, without proofs, the suprema and infima (if they exist) of the following sets:
  \enum{
  \item $\{m / n: m, n \in \mathbf{N}$ with $m<n\}$.
  \item $\left\{(-1)^{m} / n: m, n \in \mathbf{N}\right\}$.
  \item $\{n /(3 n+1): n \in \mathbf{N}\}$
  \item $\{m /(m+n): m, n \in \mathbf{N}\}$
  }
\end{exercise}

\begin{solution}
  \enum{
  \item $\sup = 1$, $\inf = 0$
  \item $\sup = 1$, $\inf = -1$
  \item $\sup = 1/3$, $\inf = 1/4$
  \item $\sup = 1$, $\inf = 0$
  }
\end{solution}

\begin{exercise}
  \enum{
  \item If $\sup A<\sup B$, show that there exists an element $b \in B$ that is an upper bound for $A$.
  \item Give an example to show that this is not always the case if we only assume $\sup A \leq \sup B$
  }
\end{exercise}

\begin{solution}
  \enum{
  \item By Lemma 1.3.8 we know there exists a $b$ such that $(\sup B) - \epsilon < b$ for any $\epsilon > 0$, We set $\epsilon$ to be small enough that $\sup A < (\sup B)-\epsilon$ meaning $\sup A < b$ for some $b$, and thus $b$ is an upper bound on $A$.
  \item $A = \{x \mid x \le 1\}$, $B = \{x \mid x < 1\}$ no $b \in B$ is an upper bound since $1 \in A$ and $1 > b$.
  }
\end{solution}

\begin{exercise}[Cut Property]

  The Cut Property of the real numbers is the following:

  If $A$ and $B$ are nonempty, disjoint sets with $A \cup B=\mathbf{R}$ and $a<b$ for all $a \in A$ and $b \in B$, then there exists $c \in \mathbf{R}$ such that $x \leq c$ whenever $x \in A$ and $x \geq c$ whenever $x \in B$.

  \enum{
  \item Use the Axiom of Completeness to prove the Cut Property.
  \item Show that the implication goes the other way; that is, assume $\mathbf{R}$ possesses the Cut Property and let $E$ be a nonempty set that is bounded above. Prove $\sup E$ exists.
  \item The punchline of parts (a) and (b) is that the Cut Property could be used in place of the Axiom of Completeness as the fundamental axiom that distinguishes the real numbers from the rational numbers. To drive this point home, give a concrete example showing that the Cut Property is not a valid statement when $\mathbf{R}$ is replaced by $\mathbf{Q}$.
  }
\end{exercise}

\begin{solution}
  \enum{
  \item If $c = \sup A = \inf B$ then $a \le c \le b$ is obvious. So we must only prove $\sup A = \inf B$. If $\sup A < \inf B$ then consider $c=\frac{\sup A + \inf B}{2}$. $c > \sup A$ and therefore $c \notin A$; similarly $c < \inf B$ and therefore $c \notin B$, implying $A \cup B \ne \mathbf{R}$. If $\sup A > \inf B$ then we can find $a$ such that $a > b$ by subtracting $\epsilon > 0$ and using the least upper/lower bound facts, similarly to Lemma 1.3.8. Thus $\sup A$ must equal $\inf B$ since we have shown both alternatives are impossible.
  \item Let $B = \{x \mid e < x,\forall e\in E\}$ and let $A = B^c$. Clearly $a < b$ so the cut property applies. We have $a \le c \le b$ and must show the two conditions for $c = \sup E$
    \enumr{
    \item Since $E \subseteq A$, $a \le c$ implies $e \le c$ thus $c$ is an upper bound.
    \item $c \le b$ implies $c$ is the smallest upper bound.
    }

    Note: Using (a) here would be wrong, it assumes the axiom of completeness so we would be making a circular argument.
  \item $A = \{r \in \mathbf{Q} \mid r^2 < 2\}$, $B = A^c$ does not satisfy the cut property in $\mathbf{Q}$ since $\sqrt 2 \notin \mathbf{Q}$
  }
\end{solution}


\begin{exercise}
  Decide if the following statements about suprema and infima are true or false. Give a short proof for those that are true. For any that are false, supply an example where the claim in question does not appear to hold.
  \enum{
  \item If $A$ and $B$ are nonempty, bounded, and satisfy $A \subseteq B$, then $\sup A \leq$ $\sup B .$
  \item If $\sup A<\inf B$ for sets $A$ and $B$, then there exists a $c \in \mathbf{R}$ satisfying $a<c<b$ for all $a \in A$ and $b \in B$.
  \item If there exists a $c \in \mathbf{R}$ satisfying $a<c<b$ for all $a \in A$ and $b \in B$, then $\sup A<\inf B$.
  }
\end{exercise}

\begin{solution}
  \enum{
  \item True. We know $a \le \sup A$ and $a \le \sup B$ since $A \subseteq B$. since $\sup A$ is the least upper bound on $A$ we have $\sup A \le \sup B$.
  \item True. Let $c = (\sup A + \inf B)/2$, $c > \sup A$ implies $a < c$ and $c < \inf B$ implies $c < b$ giving $a < c < b$ as desired.
  \item False. consider $A = \{x \mid x < 1\}$, $B = \{x \mid x > 1\}$, $a < 1 < b$ but $\sup A \not < \inf B$ since $1 \not < 1$.
  }
\end{solution}
