\section{Consequences of Completeness}


\begin{exercise}
  Recall that $\mathbf{I}$ stands for the set of irrational numbers.
  \enum{
  \item Show that if $a, b \in \mathbf{Q}$, then $a b$ and $a+b$ are elements of $\mathbf{Q}$ as well.
  \item Show that if $a \in \mathbf{Q}$ and $t \in \mathbf{I}$, then $a+t \in \mathbf{I}$ and $a t \in \mathbf{I}$ as long as $a \neq 0$.
  \item Part (a) can be summarized by saying that $\mathbf{Q}$ is closed under addition and multiplication. Is $\mathbf{I}$ closed under addition and multiplication? Given two irrational numbers $s$ and $t$, what can we say about $s+t$ and $s t$?
  }
\end{exercise}

\begin{solution}
  \enum{
    \item Trivial.
    \item Suppose $a + t \in \mathbf{Q}$, then by (a) $(a + t) - a = t \in \mathbf{Q}$ contradicting $t \in \mathbf{I}$.
    \item $\mathbf{I}$ is not closed under addition or multiplication. consider $(1 - \sqrt 2) \in \mathbf{I}$ by (b), and $\sqrt{2} \in \mathbf{I}$. the sum $(1-\sqrt2)+\sqrt2 = 1 \in \mathbf{Q} \notin \mathbf{I}$. Also $\sqrt 2 \cdot \sqrt 2 = 2 \in \mathbf Q \notin \mathbf I$.
  }
\end{solution}


\begin{exercise}
  Let $A \subseteq \mathbf{R}$ be nonempty and bounded above, and let $s \in \mathbf{R}$ have the property that for all $n \in \mathbf{N}, s+\frac{1}{n}$ is an upper bound for $A$ and $s-\frac{1}{n}$ is not an upper bound for $A$. Show $s=\sup A$.
\end{exercise}

\begin{solution}
  This is basically a rephrasing of Lemma 1.3.8 using the archimedean property. The most straightforward approach is to argue by contradiction:
  \enumr{
  \item If $s < \sup A$ then there exists an $n$ such that $s + 1/n < \sup A$ contradicting $\sup A$ being the least upper bound.
  \item If $s > \sup A$ then there exists an $n$ such that $s - 1/n > \sup A$ where $s - 1/n$ is not an upper bound, contradicting $\sup A$ being an upper bound.
  }
  Thus $s = \sup A$ is the only remaining possibility.
\end{solution}


\begin{exercise}
  Prove that $\bigcap_{n=1}^{\infty}(0,1 / n)=\emptyset$. Notice that this demonstrates that the intervals in the Nested Interval Property must be closed for the conclusion of the theorem to hold.
\end{exercise}

\begin{solution}
  Suppose $x \in \bigcap_{n=1}^\infty (0,1/n)$, then we have $0 < x < 1/n$ forall $n$, which is impossible by the archimedean property, In other words we can always set $n$ large enough that $x$ lies outside the interval.
\end{solution}


\begin{exercise}
  Let $a<b$ be real numbers and consider the set $T=\mathbf{Q} \cap[a, b]$. Show $\sup T=b$
\end{exercise}

\begin{solution}
  We must show the two conditions for a least upper bound
  \enumr{
  \item Clearly $t \le b$ forall $t \in T$
  \item Let $a < b' < b$. $b'$ Cannot be an upper bound for $T$ since the density theorem tells us we can find $r \in \mathbf{Q} \cap[a,b]$ such that $b' < r < b$.
  }
\end{solution}

\begin{exercise}
  Using Exercise 1.4.1, supply a proof that $\mathbf{I}$ is dense in $\mathbf{R}$ by considering the real numbers $a-\sqrt{2}$ and $b-\sqrt{2}$. In other words show for every two real numbers $a<b$ there exists an irrational number $t$ with $a<t<b$.
\end{exercise}

\begin{solution}
  The density theorem lets us find a rational number $r$ with $a-\sqrt2 < r < b-\sqrt2$, adding $\sqrt 2$ to both sides gives $a < r +\sqrt 2< b$. From 1.4.1 we know $r+\sqrt2$ is irrational, so setting $t = r+\sqrt2$ gives $a<t<b$ as desired.
\end{solution}

\begin{exercise}
  Recall that a set $B$ is dense in $\mathbf{R}$ if an element of $B$ can be found between any two real numbers $a<b$. Which of the following sets are dense in $\mathbf{R}$ ? Take $p \in \mathbf{Z}$ and $q \in \mathbf{N}$ in every case.
  \enum{
  \item The set of all rational numbers $p / q$ with $q \leq 10$.
  \item The set of all rational numbers $p / q$ with $q$ a power of 2 .
  \item The set of all rational numbers $p / q$ with $10|p| \geq q$.
  }
\end{exercise}

\begin{solution}
  \enum{
  \item Not dense since we cannot make $|p|/q$ smaller then $1/10$.
  \item Dense.
  \item Not dense since we cannot make $|p|/q$ smaller then $1/10$.
  }
\end{solution}

% TODO: Add theorem 1.4.5 to the appendix or something
\begin{exercise}
  Finish the proof of Theorem 1.4.5 by showing that the assumption $\alpha^{2}>2$ leads to a contradiction of the fact that $\alpha=\sup T$
\end{exercise}

\begin{solution}
  Recall $T = \{t \in \mathbf{R} \mid t^2 < 2\}$ and $\alpha = \sup T$. suppose $\alpha^2>2$, we will show there exists an $n \in \mathbf{N}$ such that $(\alpha - 1/n)^2 > 2$ contradicting the assumption that $\alpha$ is the least upper bound.

  We expand $(\alpha - 1/n)^2$ to find $n$ such that $(\alpha^2 - 1/n) > 2$
  $$2 < (\alpha - 1/n)^2 = \alpha^2 - \frac{2\alpha}{n} + \frac{1}{n^2} < \alpha^2 + \frac{1 - 2\alpha}{n}$$
  Then
  $$2 < \alpha^2 + \frac{1 - 2\alpha}{n} \implies n(2 - \alpha^2) < 1 - 2\alpha$$
  Since $2 - \alpha^2 < 0$ dividing reverses the inequality gives us
  $$n > \frac{1-2\alpha}{2 - \alpha^2}$$

  This contradicts $\alpha^2 > 2$ since we have shown $n$ can be picked such that $(\alpha^2 - 1/n) > 2$ meaning $\alpha$ is not the least upper bound.
\end{solution}

\begin{exercise}
  Give an example of each or state that the request is impossible. When a request is impossible, provide a compelling argument for why this is the case.
  \enum{
  \item Two sets $A$ and $B$ with $A \cap B=\emptyset, \sup A=\sup B, \sup A \notin A$ and $\sup B \notin B$.
  \item A sequence of nested open intervals $J_{1} \supseteq J_{2} \supseteq J_{3} \supseteq \cdots$ with $\bigcap_{n=1}^{\infty} J_{n}$ nonempty but containing only a finite number of elements.
  \item A sequence of nested unbounded closed intervals $L_{1} \supseteq L_{2} \supseteq L_{3} \supseteq \cdots$ with $\bigcap_{n=1}^{\infty} L_{n}=\emptyset$. (An unbounded closed interval has the form $[a, \infty)=$ $\{x \in R: x \geq a\} .)$
  \item A sequence of closed bounded (not necessarily nested) intervals $I_{1}, I_{2}$, $I_{3}, \ldots$ with the property that $\bigcap_{n=1}^{N} I_{n} \neq \emptyset$ for all $N \in \mathbf{N}$, but $\bigcap_{n=1}^{\infty} I_{n}=\emptyset$.
  }
\end{exercise}

\begin{solution}
  \enum{
  \item $A = \mathbf{Q} \cap (0, 1)$, $B = \mathbf{I} \cap (0, 1)$. $A \cap B = \emptyset$, $\sup A = \sup B = 1$ and $1 \notin A$, $1 \notin B$.
  \item Defining $J_i = (a_i, b_i)$, $A = \{a_n : n \in \mathbf{N}\}$, $B = \{b_n : n \in \mathbf{N}\}$, $\bigcap_{n=1}^\infty J_n$ will at least contain $(\sup A, \inf B)$. Thus, a necessary condition to meet the request is $\sup A = \inf B$.

  $J_i = (-2^{-i}, 2^{-i})$ satisfies this condition ($\sup A = \inf B = 0$) and by inspection, $\bigcap_{n=1}^\infty J_n = \{0\}$, which meets the request.

  \item $L_n = [n, \infty)$ has $\bigcap_{n=1}^\infty L_n = \emptyset$
  \item Impossible. Let $J_n = \bigcap_{k=1}^n I_k$ and observe the following
    \enumr{
      \item Since $\bigcap_{n=1}^N I_n \ne \emptyset$ we have $J_n \ne \emptyset$.
      \item $J_n$ being the intersection of closed intervals makes it a closed interval.
      \item $J_{n+1} \subseteq J_n$ since $I_{n+1} \cap J_n \subseteq J_n$
      \item $\bigcap_{n=1}^\infty J_n = \bigcap_{n=1}^\infty\left(\bigcap_{k=1}^n I_k\right) = \bigcap_{n=1}^\infty I_n$
    }

    By (i), (ii) and (iii) the Nested Interval Property tells us $\bigcap_{n=1}^\infty J_n \ne \emptyset$. Therefore by (iv) $\bigcap_{n=1}^\infty I_n \ne \emptyset$.
  }
\end{solution}


