\section{Cantor's theorem}

\begin{exercise}
  Show that $(0,1)$ is uncountable if and only if $\mathbf{R}$ is uncountable.
\end{exercise}

\begin{solution}
  In Exercise 1.5.4 (a) we found a bijection $f : (0,1) \to \mathbf R$. Now suppose $g : (0,1) \to \mathbf N$ is some map, we must show $g$ is bijective if and only if $(g \circ f) : \mathbf R \to \mathbf N$ is bijective.
  This is clearly true as if $g$ is bijective then $(g \circ f)$ is bijective (composition of bijective functions), and if $(g \circ f)$ is bijective then $(g \circ f) \circ f^{-1} = g$ is bijective.
\end{solution}

\begin{exercise}
  Let $f : \mathbf{N} \to \mathbf{R}$ be a way to list every real number (hence show $\mathbf R$ is countable).

  Define a new number $x$ with digits $b_1b_2\ldots$ given by
  $$
  b_{n}= \begin{cases}2 & \text { if } a_{n n} \neq 2 \\ 3 & \text { if } a_{n n}=2\end{cases}
  $$

  \enum{
  \item Explain why the real number $x=. b_{1} b_{2} b_{3} b_{4} \ldots$ cannot be $f(1)$.
  \item Now, explain why $x \neq f(2)$, and in general why $x \neq f(n)$ for any $n \in \mathbf{N}$.
  \item Point out the contradiction that arises from these observations and conclude that $(0,1)$ is uncountable.
  }
\end{exercise}



\begin{solution}
  \enum{
  \item The first digit is different
  \item The nth digit is different
  \item Therefore $x$ is not in the list, since the nth digit is different
  }
\end{solution}


\begin{exercise}
  Supply rebuttals to the following complaints about the proof of Theorem 1.6.1.
  \enum{
  \item Every rational number has a decimal expansion, so we could apply this same argument to show that the set of rational numbers between 0 and 1 is uncountable. However, because we know that any subset of $\mathbf{Q}$ must be countable, the proof of Theorem 1.6.1 must be flawed.
  \item Some numbers have two different decimal representations. Specifically, any decimal expansion that terminates can also be written with repeating 9's. For instance, $1 / 2$ can be written as $.5$ or as $.4999 \ldots$ Doesn't this cause some problems?
  }
\end{exercise}

\begin{solution}
  \enum{
  \item False, since the constructed number has an infinite number of decimals it is irrational.
  \item No, since if we have $9999\dots$ and change the nth digit $9992999 = 9993$ is still different.
  }
\end{solution}

\begin{exercise}
  Let $S$ be the set consisting of all sequences of 0 's and 1 's. Observe that $S$ is not a particular sequence, but rather a large set whose elements are sequences; namely,
  $$
  S=\left\{\left(a_{1}, a_{2}, a_{3}, \ldots\right): a_{n}=0 \text { or } 1\right\}
  $$
  As an example, the sequence $(1,0,1,0,1,0,1,0, \ldots)$ is an element of $S$, as is the sequence $(1,1,1,1,1,1, \ldots)$.
  Give a rigorous argument showing that $S$ is uncountable.
\end{exercise}

\begin{solution}
  We flip every bit in the diagonal just like with $\mathbf{R}$. Another way would be to show $S \sim \mathbf{R}$ by writing real numbers in base 2.
\end{solution}

\begin{exercise}
  \enum{
  \item Let $A=\{a, b, c\}$. List the eight elements of $P(A)$. (Do not forget that $\emptyset$ is considered to be a subset of every set.)
  \item If $A$ is finite with $n$ elements, show that $P(A)$ has $2^{n}$ elements.
  }
\end{exercise}

\begin{solution}
  \enum{
  \item $A = \left\{\emptyset, \{a\}, \{b\}, \{c\}, \{a, b\}, \{a, c\}, \{b, c\}, \{a,b,c\}\right\}$.
  \item There are $n$ elements, we can include or exclude each element so there are $2^n$ subsets.
  }
\end{solution}

\begin{exercise}
  \enum{
  \item Using the particular set $A=\{a, b, c\}$, exhibit two different $1-1$ mappings from $A$ into $P(A)$.
  \item Letting $C=\{1,2,3,4\}$, produce an example of a $1-1$ map $g: C \rightarrow P(C)$.
  \item Explain why, in parts (a) and (b), it is impossible to construct mappings that are onto.
  }
\end{exercise}

\begin{solution}
  \enum{
  \item $f(x) = \{x\}$, $f(x) = \{x, b\}$ for $x \ne b$ and $f(x) = \{a,b,c\}$ for $x=b$.
  \item $f(x) = \{x\}$.
  \item We can hit at most $n$ elements in the power set out of the $2^n$ total elements.
  }
\end{solution}


\begin{theorem}[Cantor's Theorem]
  Given any set $A$, there does not exist a function $f: A \rightarrow P(A)$ that is onto.
\end{theorem}

\begin{proof}
  Suppose $f : A \to P(A)$ is onto. We want to use the self referential nature of $P(A)$ to find a contradiction. Define
  $$
  B = \{a : a \notin f(a)\}
  $$
  Since $f$ is onto we must have $f(a) = B$ for some $a \in A$. Then either
  \enumr{
  \item $a \in B$ implies $a \in f(a)$ which by the definition of $B$ implies $a \notin B$, so $a \in B$ is impossible.
  \item $a \notin B$ implies $a \notin f(a)$ since $f(a) = B$. but if $a \notin f(a)$ then $a \in B$ by the definition of $B$, contradicting $a \notin B$.
  }
  Therefore $f$ cannot be onto, since we have found a $B \in P(A)$ where $f(a) = B$ is impossible.

  Stepping back, the pearl of the argument is that if $B = f(a)$ then $B = \{a : a \notin B\}$ is undecidable/impossible.
\end{proof}

\begin{exercise}
  See the proof of Cantor's theorem above (the rest is a computation)
\end{exercise}

\begin{exercise}
  See the proof of Cantor's theorem above
\end{exercise}

\begin{exercise}
  Using the various tools and techniques developed in the last two sections (including the exercises from Section 1.5), give a compelling argument showing that $P(\mathbf{N}) \sim \mathbf{R}$.
\end{exercise}

\begin{solution}
Recall from Exercise 1.6.4 that if
  $$
  S=\left\{\left(a_{1}, a_{2}, a_{3}, \ldots\right): a_{n}=0 \text { or } 1\right\}
  $$
then $S \sim \mathbf R$. Define $f : P(\mathbf N) \rightarrow S$ as $f(x) = (a_1, a_2, \ldots)$ where $a_i = 1$ if $i \in x$ and $a_i = 0$ otherwise. $f$ is thus a one-to-one, onto map between $P(\mathbf N)$ and $S$, hence $P(\mathbf N) \sim S$. Since $\sim$ is an equivalence relation, $P(\mathbf N) \sim \mathbf R$.

%   Partial solution below:
%   I will show $P(\mathbf N) \sim [0, 1]$ then use 1.5.3 to conclude $P(\mathbf N) \sim \mathbf R$.

%   Let $A \subseteq \mathbf N$ and let $a_n$ be the nth smallest element of $A$.
%   We can write $a_n$ via the digit representation as $a_n = d_1d_2d_3\dots d_m$, concatinating the digits of every $a_n$ in order gives a possibly infinite sequence of digits $d_1d_2d_3\dots$

%   This process is clearly 1-1, however it is not onto as $\{1, 2\}$ and $\{12\}$ both give the same digits. Thus $P(\mathbf N)$ is ``greater then or equal'' $\mathbf{R}$, if we show a 1-1 map $\mathbf R \to P(\mathbf{N})$ we can complete the proof using 1.5.11.

\end{solution}

\begin{exercise}
  As a final exercise, answer each of the following by establishing a $1-1$ correspondence with a set of known cardinality.
  \enum{
  \item Is the set of all functions from $\{0,1\}$ to $\mathbf{N}$ countable or uncountable?
  \item Is the set of all functions from $\mathbf{N}$ to $\{0,1\}$ countable or uncountable?
  \item Given a set $B$, a subset $\mathcal{A}$ of $P(B)$ is called an antichain if no element of $\mathcal{A}$ is a subset of any other element of $\mathcal{A} .$ Does $P(\mathbf{N})$ contain an uncountable antichain?
  }
\end{exercise}

\begin{solution}
  \enum{
  \item The set of functions from $\{0, 1\}$ to $\mathbf{N}$ is the same as $\mathbf N^2$ which we found was countable in Exercise 1.5.9.
  \item This is the same as an infinite list of zeros and ones which we showed was uncountable in Exercise 1.6.4.
  \item Let $\mathcal A$ be an antichain of $P(\mathbf N)$, let $\mathcal A_l$ be the sets in $\mathcal A$ of size $l$.
    For finite $l$, $\mathcal A_l$ is countable since $\mathcal A_l \subseteq \mathbf N^l$ is countable (shown in 1.5.9). Therefore the countable union
    $\bigcup_{l=0}^\infty \mathcal A_l = \mathcal A$
    is countable. Thus, if $P(\mathbf N)$ contains an uncountable antichain, ``most" sets in the antichain must be infinite (in that there will be uncountably many sets in the antichain will be infinite, whereas only countably many sets in the antichain will be finite).

    This observation inspires the following construction, using a variant of the set $S$ from Exercise 1.6.4. Define the set
    $$\mathcal A = \left\{ \{10n + d(x, n) : n \in \mathbf N\}: x \in (0, 1)\right\}$$
    where $d(x, n)$ is the $n$'th digit in the decimal expansion of $x$. To avoid the issue of some numbers having two equivalent decimal representations, always use the representation with repeating 9's. In this manner, each element of $\mathcal A$ encodes a particular real number, in a similar way that each element of $S$ encodes a particular real number through its binary expansion.

    Note that each element of $\mathcal A$ is infinite. Note also that since any two distinct real numbers will differ in at least one place in their decimal expansions, the corresponding elements in $\mathcal A$ will differ there as well, and hence $\mathcal A$ is an antichain.

    Formally, let $x_1, x_2$ be two distinct real numbers, $A_1, A_2$ be the elements of $\mathcal A$ corresponding to $x_1, x_2$ respectively, and $n$ be the first decimal position where $x_1$ and $x_2$ differ. Then $10n + d(x_1, n)$ will be in $A_1$ but not $A_2$, and $10n + d(x_2, n)$ will be in $A_2$ but not $A_1$. Thus, neither $A_1 \subseteq A_2$ nor $A_2 \subseteq A_1$. Since $(0, 1)$ is uncountable, $\mathcal A$ is an uncountable antichain in $P(\mathbf N)$.
  }
\end{solution}
