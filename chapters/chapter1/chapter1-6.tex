\section{Cantor's theorem}

\begin{exercise}
  Show that $(0,1)$ is uncountable if and only if $\mathbf{R}$ is uncountable.
\end{exercise}

\begin{solution}
  Exercise 1.5.4 tells us $(0, 1)$ has the same cardinality as $\mathbf{R}$.

  \TODO Prove without using exercise 1.5.4 (probably what was intended)
\end{solution}

\begin{exercise}
  \enum{
  \item Explain why the real number $x=. b_{1} b_{2} b_{3} b_{4} \ldots$ cannot be $f(1)$.
  \item Now, explain why $x \neq f(2)$, and in general why $x \neq f(n)$ for any $n \in \mathbf{N}$.
  \item Point out the contradiction that arises from these observations and conclude that $(0,1)$ is uncountable.
  }

  \TODO Make question self contained
\end{exercise}



\begin{solution}
  \enum{
  \item The first digit is different
  \item The nth digit is different
  \item Therefor $x$ is not in the list, since the nth digit would be different by definition
  }
\end{solution}


\begin{exercise}
  Supply rebuttals to the following complaints about the proof of Theorem 1.6.1.
  \enum{
  \item Every rational number has a decimal expansion, so we could apply this same argument to show that the set of rational numbers between 0 and 1 is uncountable. However, because we know that any subset of $\mathbf{Q}$ must be countable, the proof of Theorem 1.6.1 must be flawed.
  \item Some numbers have two different decimal representations. Specifically, any decimal expansion that terminates can also be written with repeating 9's. For instance, $1 / 2$ can be written as $.5$ or as $.4999 \ldots$ Doesn't this cause some problems?
  }
\end{exercise}

\begin{solution}
  \enum{
  \item False, since the constructed number has an infinite number of decimals it is irrational.
  \item No, since if we have $9999\dots$ and change the nth digit $9992999 = 9993$ is still different.
  }
\end{solution}

\begin{exercise}
  Let $S$ be the set consisting of all sequences of 0 's and 1 's. Observe that $S$ is not a particular sequence, but rather a large set whose elements are sequences; namely,
  $$
  S=\left\{\left(a_{1}, a_{2}, a_{3}, \ldots\right): a_{n}=0 \text { or } 1\right\}
  $$
  As an example, the sequence $(1,0,1,0,1,0,1,0, \ldots)$ is an element of $S$, as is the sequence $(1,1,1,1,1,1, \ldots)$.
  Give a rigorous argument showing that $S$ is uncountable.
\end{exercise}

\begin{solution}
  We flip every bit in the diagonal just like with $\mathbf{R}$. Another way would be to show $S \sim \mathbf{R}$ by writing real numbers in base 2.
\end{solution}

\begin{exercise}
  \enum{
  \item Let $A=\{a, b, c\}$. List the eight elements of $P(A)$. (Do not forget that $\emptyset$ is considered to be a subset of every set.)
  \item If $A$ is finite with $n$ elements, show that $P(A)$ has $2^{n}$ elements.
  }
\end{exercise}

\begin{solution}
  \enum{
  \item $A = \left\{\emptyset, \{a\}, \{b\}, \{c\}, \{a, b\}, \{a, c\}, \{b, c\}, \{a,b,c\}\right\}$.
  \item There are $n$ elements, we can include or exclude each element so there are $2^n$ subsets.
  }
\end{solution}

\begin{exercise}
  \enum{
  \item Using the particular set $A=\{a, b, c\}$, exhibit two different $1-1$ mappings from $A$ into $P(A)$.
  \item Letting $C=\{1,2,3,4\}$, produce an example of a $1-1$ map $g: C \rightarrow P(C)$.
  \item Explain why, in parts (a) and (b), it is impossible to construct mappings that are onto.
  }
\end{exercise}

\begin{solution}
  \enum{
  \item $f(x) = \{x\}$, $f(x) = \{x, b\}$ for $x \ne b$ and $f(x) = \{a,b,c\}$ for $x=b$.
  \item $f(x) = \{x\}$.
  \item We can hit at most $n$ elements in the power set out of the $2^n$ total elements.
  }
\end{solution}

