\section{The Cauchy Criterion}

\begin{exercise}
  Proving the Alternating Series Test (Theorem 2.7.7) amounts to showing that the sequence of partial sums
  $$
  s_{n}=a_{1}-a_{2}+a_{3}-\cdots \pm a_{n}
  $$
  converges. (The opening example in Section $2.1$ includes a typical illustration of $\left(s_{n}\right)$.) Different characterizations of completeness lead to different proofs.

  \enum{
  \item Prove the Alternating Series Test by showing that $\left(s_{n}\right)$ is a Cauchy sequence.
  \item Supply another proof for this result using the Nested Interval Property (Theorem 1.4.1).
  \item Consider the subsequences $\left(s_{2 n}\right)$ and $\left(s_{2 n+1}\right)$, and show how the Monotone Convergence Theorem leads to a third proof for the Alternating Series Test.
  }
\end{exercise}

\begin{solution}
  The fundemental reason the alternating series test works is that the partial sums stay inside intervals which are getting smaller. Always keep the fundemental reason in mind when proving things.
  \enum{
  \item We would like to show $|a_{m+1} - a_{m+2} + \dots \pm a_{n}|$ becomes arbitrarily small. \TODO
  \item Let $I_1$ be the interval $[a_1 - a_2, a_1]$ and in general $I_n = [a_n - a_{n+1}, a_n]$ we have $I_{n+1} \subseteq I_n$ since $(a_n)$ is decreasing. The nested interval property gives
    $$\bigcap_{n=1}^\infty I_n \ne \emptyset$$
    Let $x \in \bigcap_{n=1}^\infty I_n$, since $a_n \in I_n$ and $x \in I_n$ the distance $|a_n - x|$ must be less then the length $|I_n|$. and since the length goes to zero $|a_n - x|$ can be made less then any $\epsilon$.
  \item If we can show $\lim s_{2n} = \lim s_{2n+1} = s$ that will imply $\lim s_n = s$ since each $s_n$ is either in $(s_{2n+1})$ or in $(s_{2n})$ as $n$ is must be even or odd.

    We have $s_{2n+1} \le a_1$ since
    $$
    s_{2n+1} = a_1 - (a_2 - a_3) - \dots - (a_{2n} - a_{2n+1}) \le a_1
    $$
    Thus $s_{2n+1} \to s$ by the Monotone Convergence Theorem, to show $(s_{2n}) \to s$ notice $s_{2n} = s_{2n+1} - a_{2n+1}$ with $(a_{2n+1}) \to 0$ meaning we can use the triangle inequality
    $$
    |s_{2n} - s| \le \underbrace{|s_{2n} - s_{2n+1}|}_{a_{2n+1}} + |s_{2n+1} - s| < \epsilon/2 + \epsilon/2 < \epsilon
    $$
    Thus $(s_{2n}) \to s$ aswell finally implying $(s_n) \to s$.

    \textbf{Summary:} Partition the alternating series into two subsequences of partial sums, then use MCT to show they both converge to the same limit.
  }
\end{solution}

\begin{exercise}
  Decide whether each of the following series converges or diverges:
  \enum{
  \item $\sum_{n=1}^{\infty} \frac{1}{2^{n}+n}$
  \item $\sum_{n=1}^{\infty} \frac{\sin (n)}{n^{2}}$
  \item $1-\frac{3}{4}+\frac{4}{6}-\frac{5}{8}+\frac{6}{10}-\frac{7}{12}+\cdots$
  \item $1+\frac{1}{2}-\frac{1}{3}+\frac{1}{4}+\frac{1}{5}-\frac{1}{6}+\frac{1}{7}+\frac{1}{8}-\frac{1}{9}+\cdots$
  \item $1-\frac{1}{2^{2}}+\frac{1}{3}-\frac{1}{4^{2}}+\frac{1}{5}-\frac{1}{6^{2}}+\frac{1}{7}-\frac{1}{8^{2}}+\cdots$
  }
\end{exercise}

\begin{solution}
  \enum{
  \item Converges by a comparison test with $\sum_{n=1}^\infty \frac{1}{2^n}$.
  \item Converges by a comparison test with $\sum_{n=1}^\infty \frac{1}{n^2}$.
  \item Diverges since $(n+1)/2n = 1/2 + 1/2n$ never gets smaller then $1/2$.
  \item The first thing I notice is the series cannot converge absolutely (as then it would be harmonic).

    Grouping terms gives
    $$
    \begin{aligned}
    \frac{1}{n} + \frac{1}{n+1} - \frac{1}{n+2}
    &= \frac{(n+1)(n+2) + n(n+2) - n(n+1)}{n(n+1)(n+2)} \\
    &= \frac{(n^2 + 3n + 2) + (n^2 + 2n) - (n^2 + n)}{n(n+1)(n+2)} \\
    &= \frac{n^2 + 4n + 2}{n(n+1)(n+2)}
    \end{aligned}
    $$
    Which diverges since
    $$
    \frac{n^2 + 4n + 2}{n(n+1)(n+2)} \ge \frac{n^2 + 4n + 2}{n^3} \ge \frac 1n
    $$
    In other words, If we take terms three at a time (a subsequence of the partial sums) grows faster then the harmonic series implying the subsequence (and thus the parent sequence) of partial sums diverges.

    \TODO Find better proof
  \item \TODO
  }
\end{solution}

\begin{exercise}
  \enum{
  \item Provide the details for the proof of the Comparison Test (Theorem 2.7.4) using the Cauchy Criterion for Series.
  \item Give another proof for the Comparison Test, this time using the Monotone Convergence Theorem.
  }
\end{exercise}

\begin{solution}
  Suppose $a_n, b_n \ge 0$, $a_n \le b_n$ and define $s_n = a_1 + \dots + a_n$, $t_n = b_1 + \dots + b_n$.

  \enum{
  \item We have $|a_{m+1} + \dots + a_n| \le |b_{m+1} + \dots + b_n| < \epsilon$ implying $\sum_{n=1}^\infty a_n$ converges by the cauchy criterion.
    The other direction is analogous, if $(s_n)$ diverges then $(t_n)$ must also diverge since $s_n \le t_n$.
  \item Since $(t_n) \to t$. This implies that $s_n$ is bounded, and since $s_n \le t_n$ implies $s_n \le t$ by the order limit theorem, we can use the monotone convergence theorem to conclude $(s_n)$ converges.
  }
\end{solution}

\begin{exercise}
  Give an example of each or explain why the request is impossible referencing the proper theorem(s).
  \enum{
  \item Two series $\sum x_{n}$ and $\sum y_{n}$ that both diverge but where $\sum x_{n} y_{n}$ converges.
  \item A convergent series $\sum x_{n}$ and a bounded sequence $\left(y_{n}\right)$ such that $\sum x_{n} y_{n}$ diverges.
  \item Two sequences $\left(x_{n}\right)$ and $\left(y_{n}\right)$ where $\sum x_{n}$ and $\sum\left(x_{n}+y_{n}\right)$ both converge but $\sum y_{n}$ diverges.
  \item A sequence $\left(x_{n}\right)$ satisfying $0 \leq x_{n} \leq 1 / n$ where $\sum(-1)^{n} x_{n}$ diverges.
  }
\end{exercise}

\begin{solution}
  \enum{
  \item $x_n = 1/n$ and $y_n = 1/n$ have their respective series diverge, but $\sum x_ny_n = \sum 1/n^2$ converges since it is a p-series with $p > 1$.
  \item Let $x_n = (-1)^n/n$ and $y_n = (-1)^n$. $\sum x_n$ converges but $\sum x_ny_n = \sum 1/n$ diverges. 
  \item Impossible as the algebraic limit theorem for series implies $\sum(x_n+y_n) - \sum x_n = \sum y_n$ converges.
  \item Impossible as the alternating series test implies it converges.
  }
\end{solution}

\begin{exercise}
  Prove the series $\sum_{n=1}^{\infty} 1 / n^{p}$ converges if and only if $p>1$. (Corollary 2.4.7)
\end{exercise}

\begin{solution}
  Eventually we have $1/n^p < 1/p^n$ for $p > 1$ (polynomial vs exponential) meaning we can use the comparison test to conclude $\sum_{n=1}^\infty 1/n^p$ converges if $p > 1$.

  Now suppose $p \le 1$, since $1/n^p \le 1/n$ a comparsion test with the harmonic series implies $p \le 1$ diverges.
\end{solution}

\begin{exercise}
  Let's say that a series subverges if the sequence of partial sums contains a subsequence that converges. Consider this (invented) definition for a moment, and then decide which of the following statements are valid propositions about subvergent series:
  \enum{
  \item If $\left(a_{n}\right)$ is bounded, then $\sum a_{n}$ subverges.
  \item All convergent series are subvergent.
  \item If $\sum\left|a_{n}\right|$ subverges, then $\sum a_{n}$ subverges as well.
  \item If $\sum a_{n}$ subverges, then $\left(a_{n}\right)$ has a convergent subsequence.
  }
\end{exercise}

\begin{solution}
  \enum{
    \item False, consider $a_n = 1$ then $s_n = n$ does not have a convergent subsequence.
    \item True, every subsequence converges to the same limit in fact.
    \item True, since $s_n = \sum_{k=1}^n |a_k|$ converges it is bounded $|s_n| \le M$, and since $t_n = \sum_{k=1}^n a_k$ is smaller $t_n \le s_n$ it is bounded $t_n \le M$ which by BW implies there exists a convergent subsequence $(t_{n_k})$.
    \item False, $a_n = (1, -1, 2, -2, \dots)$ has no convergent subsequence but the sum $s_n = \sum_{k=1}^n a_k$ has the subsequence $(s_{2n}) \to 0$.
  }
\end{solution}

\begin{exercise}
  \enum{
  \item Show that if $a_{n}>0$ and $\lim \left(n a_{n}\right)=l$ with $l \neq 0$, then the series $\sum a_{n}$ diverges.
  \item Assume $a_{n}>0$ and $\lim \left(n^{2} a_{n}\right)$ exists. Show that $\sum a_{n}$ converges.
  }
\end{exercise}

\begin{solution}
  \enum{
  \item \TODO
  \item \TODO
  }
\end{solution}

\begin{exercise}
  Consider each of the following propositions. Provide short proofs for those that are true and counterexamples for any that are not.
  \enum{
  \item If $\sum a_{n}$ converges absolutely, then $\sum a_{n}^{2}$ also converges absolutely.
  \item If $\sum a_{n}$ converges and $\left(b_{n}\right)$ converges, then $\sum a_{n} b_{n}$ converges.
  \item If $\sum a_{n}$ converges conditionally, then $\sum n^{2} a_{n}$ diverges.
  }
\end{exercise}

\begin{solution}
  \enum{
  \item True since $(a_n) \to 0$ so eventually $a_n^2 \le |a_n|$ meaning $\sum a_n^2$ converges by a comparsion test with $\sum |a_n|$. (and clearly $|a_n^2| = a_n^2$ also converges.)
  \item \TODO
  \item \TODO
  }
\end{solution}

\begin{exercise}[Ratio Test]
  Given a series $\sum_{n=1}^{\infty} a_{n}$ with $a_{n} \neq 0$, the Ratio Test states that if $\left(a_{n}\right)$ satisfies
  $$
  \lim \left|\frac{a_{n+1}}{a_{n}}\right|=r<1
  $$
  then the series converges absolutely.

  \enum{
  \item Let $r^{\prime}$ satisfy $r<r^{\prime}<1$. Explain why there exists an $N$ such that $n \geq N$ implies $\left|a_{n+1}\right| \leq\left|a_{n}\right| r^{\prime}$.
  \item Why does $\left|a_{N}\right| \sum\left(r^{\prime}\right)^{n}$ converge?
  \item Now, show that $\sum\left|a_{n}\right|$ converges, and conclude that $\sum a_{n}$ converges.
  }
\end{exercise}

\begin{solution}
  \enum{
  \item \TODO
  \item \TODO
  \item \TODO
  }
\end{solution}

\begin{exercise}[Infinite Products]
  Review Exercise 2.4.10 about infinite products and then answer the following questions:
  \enum{
  \item Does $\frac{2}{1} \cdot \frac{3}{2} \cdot \frac{5}{4} \cdot \frac{9}{8} \cdot \frac{17}{16} \cdots$ converge?
  \item The infinite product $\frac{1}{2} \cdot \frac{3}{4} \cdot \frac{5}{6} \cdot \frac{7}{8} \cdot \frac{9}{10} \cdots$ certainly converges. (Why?) Does it converge to zero?
  \item In 1655, John Wallis famously derived the formula
    $$
    \left(\frac{2 \cdot 2}{1 \cdot 3}\right)\left(\frac{4 \cdot 4}{3 \cdot 5}\right)\left(\frac{6 \cdot 6}{5 \cdot 7}\right)\left(\frac{8 \cdot 8}{7 \cdot 9}\right) \cdots=\frac{\pi}{2}
    $$
    Show that the left side of this identity at least converges to something. (A complete proof of this result is taken up in Section 8.3.)
  }
\end{exercise}

\begin{solution}
  \enum{
  \item \TODO
  \item \TODO
  \item \TODO
  }
\end{solution}

\begin{exercise}
  Find examples of two series $\sum a_{n}$ and $\sum b_{n}$ both of which diverge but for which $\sum \min \left\{a_{n}, b_{n}\right\}$ converges. To make it more challenging, produce examples where $\left(a_{n}\right)$ and $\left(b_{n}\right)$ are strictly positive and decreasing.
\end{exercise}

\begin{solution}
  \TODO
\end{solution}

\begin{exercise}[Summation-by-parts]
  Let $\left(x_{n}\right)$ and $\left(y_{n}\right)$ be sequences, let $s_{n}=x_{1}+x_{2}+\cdots+x_{n}$ and set $s_{0}=0 .$ Use the observation that $x_{j}=s_{j}-s_{j-1}$ to verify the formula
  $$
  \sum_{j=m}^{n} x_{j} y_{j}=s_{n} y_{n+1}-s_{m-1} y_{m}+\sum_{j=m}^{n} s_{j}\left(y_{j}-y_{j+1}\right)
  $$
\end{exercise}

\begin{solution}
  Since $x_j = s_j - s_{j-1}$ we can rewrite the sum as
  $$
  \begin{aligned}
    &= \sum_{j=m}^n (s_j - s_{j-1})y_j \\
    &= \sum_{j=m}^n s_j y_j - \sum_{j=m}^n s_{j-1} y_j \\
    &= (s_my_m + \dots + s_ny_n) - (s_{m-1}y_m + \dots + s_{n-1}y_n)                  &&\text{Factor out each } s_j\\
    &= s_n(y_n) - s_{m-1}(y_m) + s_m(y_m - y_{m+1}) + \dots + s_{n-1}(y_{n-1} - y_n)  &&\text{Add and subtract } s_ny_{n+1} \text{ for factoring} \\
    &= s_n(y_{n+1}) - s_{m-1}(y_m) + s_m(y_m - y_{m+1}) + \dots + s_n(y_n - y_{n+1})  &&\text{Rewrite as a sum} \\
    &= s_n(y_{n+1}) - s_{m-1}(y_m) + \sum_{j=m}^n s_j(y_j - y_{j+1})                  &&\text{Done :)} \\
  \end{aligned}
  $$
\end{solution}

\begin{exercise}[Abel's Test]
  Abel's Test for convergence states that if the series $\sum_{k=1}^{\infty} x_{k}$ converges, and if $\left(y_{k}\right)$ is a sequence satisfying
  $$
  y_{1} \geq y_{2} \geq y_{3} \geq \cdots \geq 0
  $$
  then the series $\sum_{k=1}^{\infty} x_{k} y_{k}$ converges.
  \enum{
  \item Use Exercise $2.7 .12$ to show that
    $$
    \sum_{k=1}^{n} x_{k} y_{k}=s_{n} y_{n+1}+\sum_{k=1}^{n} s_{k}\left(y_{k}-y_{k+1}\right)
    $$
    where $s_{n}=x_{1}+x_{2}+\cdots+x_{n}$.
  \item Use the Comparison Test to argue that $\sum_{k=1}^{\infty} s_{k}\left(y_{k}-y_{k+1}\right)$ converges absolutely, and show how this leads directly to a proof of Abel's Test.
  }
\end{exercise}

\begin{solution}
  \enum{
  \item \TODO
  \item \TODO
  }
\end{solution}

\begin{exercise}[Dirichlet's Test]
  Dirichlet's Test for convergence states that if the partial sums of $\sum_{k=1}^{\infty} x_{k}$ are bounded (but not necessarily convergent), and if $\left(y_{k}\right)$ is a sequence satisfying $y_{1} \geq y_{2} \geq y_{3} \geq \cdots \geq 0$ with $\lim y_{k}=0$, then the series $\sum_{k=1}^{\infty} x_{k} y_{k}$ converges.
  \enum{
  \item Point out how the hypothesis of Dirichlet's Test differs from that of Abel's Test in Exercise 2.7.13, but show that essentially the same strategy can be used to provide a proof.
  \item Show how the Alternating Series Test (Theorem 2.7.7) can be derived as a special case of Dirichlet's Test.
  }
\end{exercise}

\begin{solution}
  \enum{
  \item \TODO
  \item \TODO
  }
\end{solution}
