\section{The Algebraic and Order Limit Theorems}

\begin{exercise}
  Let $x_{n} \geq 0$ for all $n \in \mathbf{N}$.
  \enum{
  \item If $\left(x_{n}\right) \rightarrow 0$, show that $\left(\sqrt{x_{n}}\right) \rightarrow 0$.
  \item If $\left(x_{n}\right) \rightarrow x$, show that $\left(\sqrt{x_{n}}\right) \rightarrow \sqrt{x}$.
  }
\end{exercise}

\begin{solution}
  \enum{
  \item Setting $x_n < \epsilon^2$ implies $\sqrt{x_n} < \epsilon$ (for all $n \ge N$ of course)

  \item We want $|\sqrt{x_n} - \sqrt x| < \epsilon$ multiplying by $(\sqrt{x_n} + \sqrt{x})$ gives $|x_n - x| < (\sqrt{x_n} + \sqrt{x})\epsilon$ since $x_n$ is convergent, it is bounded $|x_n| \le M$ implying $\sqrt{|x_n|} \le \sqrt{M}$, multiplying gives
    $$
    |x_n - x| < \left(\sqrt{x_n} + \sqrt{x}\right)\epsilon \le \left(\sqrt{M} + \sqrt x\right)\epsilon
    $$
    Since $|x_n - x|$ can be made arbitrarily small we can make this true for some $n \ge N$. Now dividing by $\sqrt{M} + \sqrt x$ gives us
    $$|\sqrt{x_n} - \sqrt{x}| \le \frac{|x_n - x|}{\sqrt{M} + \sqrt{x}} < \epsilon$$
    therefore $|\sqrt{x_n} - \sqrt x| < \epsilon$ completing the proof.
  }
\end{solution}

\begin{exercise}
  Using only Definition 2.2.3, prove that if $\left(x_{n}\right) \rightarrow 2$, then
  \enum{
  \item $\left(\frac{2 x_{n}-1}{3}\right) \rightarrow 1$;
  \item $\left(1 / x_{n}\right) \rightarrow 1 / 2$.
  }
  (For this exercise the Algebraic Limit Theorem is off-limits, so to speak.)
\end{exercise}

\begin{solution}
  \enum{
  \item We have $\left|\frac{2}{3} x_n - \frac 43\right| = \frac 23\left|x_n - 2\right| < \epsilon$ which can always be done since $|x_n - 2|$ can be made arbitrarily small.
  \item Let $N$ be such that $|x_n - 2| < \min\{1, \epsilon\}$. Since $x_n$ is at least $1$ we can bound $|1/x_n| \le 1$, giving
    $$
    |1/x_n - 1/2| = \frac{|2-x_n|}{|2x_n|} \le \frac{|x_n-2|}{2} \le \frac{\epsilon}{2} < \epsilon.
    $$
  }
\end{solution}

\begin{exercise}[Squeeze Theorem]
  Show that if $x_{n} \leq y_{n} \leq z_{n}$ for all $n \in \mathbf{N}$, and if $\lim x_{n}=\lim z_{n}=l$, then $\lim y_{n}=l$ as well.
\end{exercise}

\begin{solution}
  Let $\epsilon > 0$, set $N$ so that $|x_n - l| < \epsilon/4$ and $|z_n - l| < \epsilon/4$. Use the triangle inequality to see $|x_n - z_n| < |x_n-l|+|l-z_n|<\epsilon/2$.
  Note that since $x_n \leq y_n \leq z_n$, $|y_n - x_n| = y_n - x_n \leq z_n - x_n = |z_n - x_n|$.
  Apply the triangle inequality again to get
  $$|y_n - l| \le |y_n - x_n| + |x_n - l| \leq |z_n - x_n| + |x_n - l|< \epsilon/2 + \epsilon/4 < \epsilon$$
\end{solution}

\begin{exercise}
  Let $\left(a_{n}\right) \rightarrow 0$, and use the Algebraic Limit Theorem to compute each of the following limits (assuming the fractions are always defined):
  \enum{
  \item $\lim \left(\frac{1+2 a_{n}}{1+3 a_{n}-4 a_{n}^{2}}\right)$
  \item $\lim \left(\frac{\left(a_{n}+2\right)^{2}-4}{a_{n}}\right)$
  \item $\lim \left(\frac{\frac{2}{a_{n}}+3}{\frac{1}{a_{n}}+5}\right)$.
  }
\end{exercise}

\begin{solution}
  \enum{
  \item Apply the ALT
    $$
    \begin{aligned}
    \lim \left(\frac{1+2 a_{n}}{1+3 a_{n}-4 a_{n}^{2}}\right)
    &= \frac{\lim\left(1 + 2a_n\right)}{\lim \left(1 + 3a_n - 4a_n^2\right)} \\
    &= \frac{1 + 2\lim (a_n)}{1 + 3\lim a_n - 4 \lim a_n^2} \\
    &= 1
    \end{aligned}
    $$
    Showing $a_n^2 \to 0$ is easy so I've omitted it
  \item Apply the ALT
    $$
    \begin{aligned}
    \lim \left(\frac{\left(a_{n}+2\right)^{2}-4}{a_{n}}\right)
    &= \lim \left(\frac{a_{n}^2 + 4a_n}{a_{n}}\right) \\
    &= \lim(a_n + 4) = 4 + \lim a_n = 4
    \end{aligned}
    $$
  \item Multiply the top and bottom by $a_n$ then apply the ALT
    $$
    \begin{aligned}
    \lim \left(\frac{\frac{2}{a_{n}}+3}{\frac{1}{a_{n}}+5}\right)
    &= \lim \left(\frac{2 + 3a_n}{1 + 5a_n}\right) \\
    &= \frac{2 + 3 \lim a_n}{1 + 5 \lim a_n} \\
    &= 2
    \end{aligned}
    $$
  }
\end{solution}


\begin{exercise}
  Let $\left(x_{n}\right)$ and $\left(y_{n}\right)$ be given, and define $\left(z_{n}\right)$ to be the ``shuffled'' sequence $\left(x_{1}, y_{1}, x_{2}, y_{2}, x_{3}, y_{3}, \ldots, x_{n}, y_{n}, \ldots\right)$. Prove that $\left(z_{n}\right)$ is convergent if and only if $\left(x_{n}\right)$ and $\left(y_{n}\right)$ are both convergent with $\lim x_{n}=\lim y_{n}$.
\end{exercise}


\begin{solution}
  Obviously if $\lim x_n = \lim y_n = l$ then $z_n \to l$. To show the other way suppose $(z_n) \to l$, then $|z_n - l| < \epsilon$ for all $n \ge N$ meaning $|y_n - l| < \epsilon$ and $|x_n - l| < \epsilon$ for $n \ge N$ aswell. Thus $\lim x_n = \lim y_n = l$.
\end{solution}

\begin{exercise}
  Consider the sequence given by $b_{n}=n-\sqrt{n^{2}+2 n}$. Taking $(1 / n) \rightarrow 0$ as given, and using both the Algebraic Limit Theorem and the result in Exercise 2.3.1, show $\lim b_{n}$ exists and find the value of the limit.
\end{exercise}

\begin{solution}
  I'm going to find the value of the limit before proving it. We have
  $$
  n - \sqrt{n^2 + 2n} = n - \sqrt{(n + 1)^2 - 1}
  $$
  For large $n$, $\sqrt{(n + 1)^2 - 1} \approx n + 1$ so $\lim b_n = -1$.

  Factoring out $n$ we get $n\left(1 - \sqrt{1 + 2/n}\right)$. Tempting as it is to apply the ALT here to say $(b_n) \to 0$ it doesn't work since $n$ diverges.

  How about if I get rid of the radical, then use the ALT to go back to what we had before?
  $$
  (n - \sqrt{n^2 + 2n})(n + \sqrt{n^2 + 2n}) = n^2 - (n^2 + 2n) = -2n
  $$
  Then we have
  $$
  b_n = n - \sqrt{n^2 + 2n} = \frac{-2n}{n + \sqrt{n^2 + 2n}} = \frac{-2}{1 + \sqrt{1 + 2/n}}
  $$
  Now we can finally use the algebraic limit theorem!
  $$
  \lim\left(\frac{-2}{1 + \sqrt{1 + 2/n}}\right) = \frac{-2}{1 + \sqrt{1 + \lim\left(2/n\right)}} = \frac{-2}{1 + \sqrt{1 + 0}} = -1
  $$

  Stepping back the key to this technique is removing the radicals via a difference of squares, then dividing both sides by the growth rate $n$ and applying the ALT.
\end{solution}

\begin{exercise}
  Give an example of each of the following, or state that such a request is impossible by referencing the proper theorem(s):
  \enum{
  \item sequences $\left(x_{n}\right)$ and $\left(y_{n}\right)$, which both diverge, but whose sum $\left(x_{n}+y_{n}\right)$ converges;
  \item sequences $\left(x_{n}\right)$ and $\left(y_{n}\right)$, where $\left(x_{n}\right)$ converges, $\left(y_{n}\right)$ diverges, and $\left(x_{n}+y_{n}\right)$ converges;
  \item a convergent sequence $\left(b_{n}\right)$ with $b_{n} \neq 0$ for all $n$ such that $\left(1 / b_{n}\right)$ diverges;
  \item an unbounded sequence $\left(a_{n}\right)$ and a convergent sequence $\left(b_{n}\right)$ with $\left(a_{n}-b_{n}\right)$ bounded;
  \item two sequences $\left(a_{n}\right)$ and $\left(b_{n}\right)$, where $\left(a_{n} b_{n}\right)$ and $\left(a_{n}\right)$ converge but $\left(b_{n}\right)$ does not.
  }
\end{exercise}

\begin{solution}
  \enum{
  \item $(x_n) = n$ and $(y_n) = -n$ diverge but $x_n + y_n = 0$ converges
  \item Impossible, the algebraic limit theorem implies $\lim (x_n + y_n) - \lim (x_n) = \lim y_n$ therefore $(y_n)$ must converge if $(x_n)$ and $(x_n + y_n)$ converge.
  \item $b_n = 1/n$ has $b_n \to 0$ and $1/b_n$ diverges. If $b_n \to b \ne 0$ then $1/b_n \to 1/b$, but since $b = 0$ ALT doesn't apply.
  \item Impossible, $|b_n|$ is convergent and therefore bounded (Theorem 2.3.2) so $|b_n| \le M_1$, and $|a_n - b_n| \le M_2$ is bounded, therefore
  $$|a_n| \leq |a_n - b_n| + |b_n| \le M_1 + M_2$$
  must be bounded.
  \item $b_n = n$ and $a_n = 0$ works. However if $(a_n) \to a$, $a \ne 0$ and $(a_nb_n) \to p$ then the ALT would imply $(b_n) \to p/a$.
  }
\end{solution}

\begin{exercise}
  Let $\left(x_{n}\right) \rightarrow x$ and let $p(x)$ be a polynomial.
  \enum{
  \item Show $p\left(x_{n}\right) \rightarrow p(x)$.
  \item Find an example of a function $f(x)$ and a convergent sequence $\left(x_{n}\right) \rightarrow x$ where the sequence $f\left(x_{n}\right)$ converges, but not to $f(x)$.
  }
\end{exercise}

\begin{solution}
  \enum{
  \item Applying the algebraic limit theorem multiple times gives $(x_n^d) \to x^d$ meaning
    $$
    \lim p(x_n) = \lim \left(a_{d}x_n^d + a_{d-1}x_n^{d-1} + \dots + a_0\right) = a_d x^d + a_{d-1}x^{d-1} + \dots + a_0 = p(x).
    $$
    As a cute corollary, any continuous function $f$ has $\lim f(x_n) = f(x)$ since polynomials can approximate continuous functions arbitrarily well by the Weierstrass approximation theorem.
  \item Let $(x_n) = 1/n$ and define $f$ as
    $$
    f(x) = \begin{cases}
      0 &\text{if } x = 0 \\
      1 &\text{otherwise}
    \end{cases}
    $$
    We have $f(1/n) = 1$ for all $n$, meaning $\lim f(1/n) = 1$ but $f(0) = 0$.
  }
\end{solution}

\begin{exercise}
  \enum{
  \item Let $\left(a_{n}\right)$ be a bounded (not necessarily convergent) sequence, and assume $\lim b_{n}=0$. Show that $\lim \left(a_{n} b_{n}\right)=0$. Why are we not allowed to use the Algebraic Limit Theorem to prove this?
  \item Can we conclude anything about the convergence of $\left(a_{n} b_{n}\right)$ if we assume that $\left(b_{n}\right)$ converges to some nonzero limit $b$ ?
  \item Use (a) to prove Theorem 2.3.3, part (iii), for the case when $a=0$.
  }
\end{exercise}

\begin{solution}
  \enum{
  \item We can't use the ALT since $a_n$ is not necessarily convergent. $a_n$ being bounded gives $|a_n| \le M$ for some $M$ giving
    $$|a_nb_n| \le M|b_n| < \epsilon$$
    Which can be accomplished by letting $|b_n| < \epsilon/M$ since $(b_n) \to 0$.
  \item No
  \item In (a) we showed $\lim (a_nb_n) = 0 = ab$ for $b = 0$ which proves part (iii) of the ALT.
  }
\end{solution}

\begin{exercise}
  Consider the following list of conjectures. Provide a short proof for those that are true and a counterexample for any that are false.
  \enum{
  \item If $\lim \left(a_{n}-b_{n}\right)=0$, then $\lim a_{n}=\lim b_{n}$.
  \item If $\left(b_{n}\right) \rightarrow b$, then $\left|b_{n}\right| \rightarrow|b|$.
  \item If $\left(a_{n}\right) \rightarrow a$ and $\left(b_{n}-a_{n}\right) \rightarrow 0$, then $\left(b_{n}\right) \rightarrow a$.
  \item If $\left(a_{n}\right) \rightarrow 0$ and $\left|b_{n}-b\right| \leq a_{n}$ for all $n \in \mathbf{N}$, then $\left(b_{n}\right) \rightarrow b$.
  }
\end{exercise}

\begin{solution}
  \enum{
  \item False, consider $a_n = n$ and $b_n = -n$.
  \item True since if $|b_n - b| < \epsilon$ then $||b_n| - |b|| \le |b_n - b| < \epsilon$ by Exercise 1.2.6 (d).
  \item True by ALT since $\lim (b_n - a_n) + \lim a_n = \lim b_n = a$.
  \item True, since $0 \le |b_n - b| \le a_n$ we have $a_n \ge 0$. Let $\epsilon > 0$ and pick $N$ such that $a_n < \epsilon$ for all $n \ge N$. Therefor
    $$
    |b_n - b| \le a_n < \epsilon
    $$
    Proving $(b_n) \to b$.
  }
\end{solution}

\begin{exercise}[Cesaro Means]
  \enum {
  \item Show that if $\left(x_{n}\right)$ is a convergent sequence, then the sequence given by the averages
    $$
    y_{n}=\frac{x_{1}+x_{2}+\cdots+x_{n}}{n}
    $$
    also converges to the same limit.
  \item Give an example to show that it is possible for the sequence $\left(y_{n}\right)$ of averages to converge even if $\left(x_{n}\right)$ does not.
  }
\end{exercise}

\begin{solution}
  \enum{
  \item
    Let $D = \sup\{|x_n - x| : n \in \mathbf N\}$ and let $0<\epsilon<D$, we have
    $$
    |y_n - x| = \left|\frac{x_1 + \dots + x_n}{n} - x\right| \le \left|\frac{|x_1 - x| + \dots + |x_n - x|}{n}\right| \le D
    $$

    Let $|x_n - x| < \epsilon/2$ for $n > N_1$ giving
    $$
    |y_n - x| \le \left|\frac{|x_1-x| + \dots + |x_{N_1}-x| + \dots + |x_n-x|}{n}\right| \le \left|\frac{N_1 D + (n-N_1)\epsilon/2}{n}\right|
    $$
    Let $N_2$ be large enough that for all $n > N_2$ (remember $0 < \epsilon < D$ so $(D-\epsilon/2)>0$.)
    $$
    0 < \frac{N_1(D - \epsilon/2)}{n} < \epsilon/2
    $$
    Therefor
    $$
    |y_n - x| \le \left|\frac{N_1(D - \epsilon/2)}{n} + \epsilon/2\right| < \epsilon
    $$
    Letting $N = \max\{N_1, N_2\}$ completes the proof as $|y_n - x| < \epsilon$ for all $n > N$.

    (Note: I could have used any $\epsilon' < \epsilon$ instead of $\epsilon/2$, I just needed some room.)
  \item $x_n = (-1)^n$ diverges but $(y_n) \to 0$.
  }
\end{solution}


\begin{exercise}
  A typical task in analysis is to decipher whether a property possessed by every term in a convergent sequence is necessarily inherited by the limit. Assume $\left(a_{n}\right) \rightarrow a$, and determine the validity of each claim. Try to produce a counterexample for any that are false.

  \enum{
  \item If every $a_{n}$ is an upper bound for a set $B$, then $a$ is also an upper bound for $B$.
  \item If every $a_{n}$ is in the complement of the interval $(0,1)$, then $a$ is also in the complement of $(0,1)$.
  \item If every $a_{n}$ is rational, then $a$ is rational.
  }
\end{exercise}

\begin{solution}
  \enum{
  \item True, let $s = \sup B$, we know $s \le a_n$ so by the order limit theorem $s \le a$ meaning $a$ is also an upper bound on $B$.
  \item True, since if $a \in (0,1)$ then there would exist an $\epsilon$-neighborhood inside $(0,1)$ that $a_n$ would have to fall in, contradicting the fact that $a_n \notin (0,1)$.
  \item False, consider the sequence of rational approximations to $\sqrt 2$
  }
\end{solution}

\begin{exercise}[Iterated Limits]
  Given a doubly indexed array $a_{m n}$ where $m, n \in \mathbf{N}$, what should $\lim _{m, n \rightarrow \infty} a_{m n}$ represent?
  \enum{
  \item Let $a_{m n}=m /(m+n)$ and compute the iterated limits
  $$
  \lim _{n \rightarrow \infty}\left(\lim _{m \rightarrow \infty} a_{m n}\right) \quad \text { and } \lim _{m \rightarrow \infty}\left(\lim _{n \rightarrow \infty} a_{m n}\right)
  $$
  Define $\lim _{m, n \rightarrow \infty} a_{m n}=a$ to mean that for all $\epsilon>0$ there exists an $N \in \mathbf{N}$ such that if both $m, n \geq N$, then $\left|a_{m n}-a\right|<\epsilon$
  \item Let $a_{m n}=1 /(m+n)$. Does $\lim _{m, n \rightarrow \infty} a_{m n}$ exist in this case? Do the two iterated limits exist? How do these three values compare? Answer these same questions for $a_{m n}=m n /\left(m^{2}+n^{2}\right)$
  \item Produce an example where $\lim _{m, n \rightarrow \infty} a_{m n}$ exists but where neither iterated limit can be computed.
  \item Assume $\lim _{m, n \rightarrow \infty} a_{m n}=a$, and assume that for each fixed $m \in \mathbf{N}$, $\lim _{n \rightarrow \infty}\left(a_{m n}\right) \rightarrow b_{m}$. Show $\lim _{m \rightarrow \infty} b_{m}=a$
  \item Prove that if $\lim _{m, n \rightarrow \infty} a_{m n}$ exists and the iterated limits both exist, then all three limits must be equal.
  }
\end{exercise}

\begin{solution}
  \enum{
  \item
    $$
    \lim_{n \to \infty}\left(\lim_{m \to \infty} \frac{m}{m+n}\right) = 1, \text{ and }
    \lim_{m \to \infty}\left(\lim_{n \to \infty} \frac{m}{m+n}\right) = 0
    $$
  \item
    For $a_{mn} = 1/(m+n)$ all three limits are zero.
    For $a_{mn} = mn/(m^2 + n^2)$ iterated limits are zero, and $\lim_{m,n\to \infty} a_{mn}$ does not exist since for $m,n \ge N$ setting $m=n$ gives
    $$
    \frac{n^2}{n^2 + n^2} = \frac{1}{2}
    $$
    Which cannot be made smaller then $\epsilon = 1/2$.

    The reason you would think to set $m=n$ is in trying to maximize $mn/(m^2 + n^2)$ notice if $m>n$ then $mn>n^2$ so we are adding more to the numerator then the denominator, hence the ratio is increasing. And if $m<n$ then the ratio is decreasing. therefore the maximum point is at $m=n$.
  \item Intuitively, in order for $\lim _{m, n \to \infty} a_{m n}$ to exist, neither iterated limit can diverge to infinity - otherwise, $a_{mn}$ can also diverge to infinity, by holding letting the index that causes divergence to grow while holding the other index fixed. Therefore, we must rely on each iterated limit diverging due to oscillation.

  The key additional ``ability'' that $\lim _{m, n \to \infty}$ gives over an iterated limit is that both $m$ and $n$ can be forced to grow big at the same time, whereas with an iterated limit only one of them is forced to grow big.

  Note that since iterated limits can only increase one of $m$ and $n$, $\min\{m, n\}$ can't be increased indefinitely - but with $\lim _{m, n \to \infty}$, it can. Thus, the idea is to introduce oscillation in the sequence, then use $\min\{m, n\}$ to cause the oscillation to die out. Define
$$ a_{mn} = \frac{(-1)^{m + n}}{\min\{m, n\}}$$

For a fixed $m$, once $n>m$, $a_{mn}$ will oscillate between $1/m$ and $-1/m$, and thus $\lim_{n \to \infty} a_{mn}$ does not exist.
Similar reasoning shows that for a fixed $n$, $\lim_{m \to \infty} a_{mn}$ does not exist either. But clearly $\lim_{m, n \to \infty} a_{mn} = 0$.


  \item
    Choose $\epsilon > 0$ and let $0 < \epsilon' < \epsilon$.
    We need to find $N$ so that $|b_m - a| < \epsilon$ for all $m > N$.

    Set $N$ such that $|a_{mn} - a| < \epsilon'$ when $n,m \ge N$.
    Then fix $m \ge N$, I will show $|b_m-a|<\epsilon$. apply the triangle inequality to get
    $$
    |b_m - a| \le |b_m - a_{mn}| + |a_{mn} - a|\quad\forall n \in \mathbf{N}
    $$
    This inequality is true for all $n$, we will pick $n$ to make it strict enough to complete the proof.
    Set $n \ge \max\{N, N_m\}$ where $N_m$ (dependent on $m$) is big enough that $|b_m - a_{mn}| < \epsilon-\epsilon'$.
    We also have $|a_{mn} - a| < \epsilon'$ since $m \ge N$ and $n \ge N$. So finally
    $$
    |b_m - a| \le |b_m - a_{mn}| + |a_{mn} - a| < (\epsilon-\epsilon') + \epsilon' = \epsilon
    $$
    And we are done. The key is that we can make $|b_{m} - a_{mn}|$ as small as we want \emph{independent of m}, so we take the limit as $n \to \infty$ to show $|b_m - a| \le |a_{mn} - a|$.
    % TODO: Replace proof by taking limit of both sides?
  \item Let $b_m = \lim_{n \to \infty} (a_{mn})$, $c_n = \lim_{m \to \infty} (a_{mn})$, and $a = \lim_{m,n\to\infty} (a_{mn})$. In (d) we showed $(b_m) \to a$; a similar argument shows $(c_n) \to a$. Thus all three limits are equal to $a$.
  }
\end{solution}
