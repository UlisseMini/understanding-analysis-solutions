\section{The Algebraic and Order Limit Theorems}

\begin{exercise}
  Let $x_{n} \geq 0$ for all $n \in \mathbf{N}$.
  \enum{
  \item If $\left(x_{n}\right) \rightarrow 0$, show that $\left(\sqrt{x_{n}}\right) \rightarrow 0$.
  \item If $\left(x_{n}\right) \rightarrow x$, show that $\left(\sqrt{x_{n}}\right) \rightarrow \sqrt{x}$.
  }
\end{exercise}

\begin{solution}
  \enum{
  \item Setting $x_n < \epsilon^2$ implies $\sqrt{x_n} < \epsilon$ (for all $n \ge N$ of course)

  \item We want $|\sqrt{x_n} - \sqrt x| < \epsilon$ multiplying by $(\sqrt{x_n} + \sqrt{x})$ gives $|x_n - x| < (\sqrt{x_n} + \sqrt{x})\epsilon$ since $x_n$ is convergent, it is bounded $|x_n| \le M$ implying $\sqrt{|x_n|} \le \sqrt{M}$, multiplying gives
    $$
    |x_n - x| < \left(\sqrt{x_n} + \sqrt{x}\right)\epsilon \le \left(\sqrt{M} + \sqrt x\right)\epsilon
    $$
    Since $|x_n - x|$ can be made arbitrarily small we can make this true for some $n \ge N$. Now dividing by $\sqrt{M} + \sqrt x$ gives us
    $$|\sqrt{x_n} - \sqrt{x}| \le \frac{|x_n - x|}{\sqrt{M} + \sqrt{x}} < \epsilon$$
    Therefor $|\sqrt{x_n} - \sqrt x| < \epsilon$ completing the proof.
  }
\end{solution}

\begin{exercise}
  Using only Definition 2.2.3, prove that if $\left(x_{n}\right) \rightarrow 2$, then
  \enum{
  \item $\left(\frac{2 x_{n}-1}{3}\right) \rightarrow 1$;
  \item $\left(1 / x_{n}\right) \rightarrow 1 / 2$.
  }
  (For this exercise the Algebraic Limit Theorem is off-limits, so to speak.)
\end{exercise}

\begin{solution}
  \enum{
  \item We have $\left|\frac{2}{3} x_n - \frac 43\right| = \frac 23\left|x_n - 2\right| < \epsilon$ which can always be done since $|x_n - 2|$ can be made arbitrarily small.
  \item Want $|(1/x_n) - 1/2| < \epsilon$ have $|x_n - 2| < \epsilon$ \TODO
  }
\end{solution}

\begin{exercise}[Squeeze Theorem]
  Show that if $x_{n} \leq y_{n} \leq z_{n}$ for all $n \in \mathbf{N}$, and if $\lim x_{n}=\lim z_{n}=l$, then $\lim y_{n}=l$ as well.
\end{exercise}

\begin{solution}
  Let $y = \lim y_n$. By the order limit theorem we have $l \le y \le l$ implying $y = l$.
\end{solution}

\begin{exercise}
  Let $\left(a_{n}\right) \rightarrow 0$, and use the Algebraic Limit Theorem to compute each of the following limits (assuming the fractions are always defined):
  \enum{
  \item $\lim \left(\frac{1+2 a_{n}}{1+3 a_{n}-4 a_{n}^{2}}\right)$
  \item $\lim \left(\frac{\left(a_{n}+2\right)^{2}-4}{a_{n}}\right)$
  \item $\lim \left(\frac{\frac{2}{a_{n}}+3}{\frac{1}{a_{n}}+5}\right)$.
  }
\end{exercise}

\begin{solution}
  \enum{
  \item I'm not sure how much work I have to show, many of these steps are obvious
    $$
    \begin{aligned}
      \lim \left(\frac{1+2 a_{n}}{1+3 a_{n}-4 a_{n}^{2}}\right)
      &= \lim \left(\frac{1}{1+3 a_{n}-4 a_{n}^{2}}\right) + 2\lim \left(\frac{a_n}{1+3 a_{n}-4 a_{n}^{2}}\right) \\
      &= 2\lim \left(\frac{1}{1/a_n + 3 - 4 a_{n}}\right) \\
      &= 0
    \end{aligned}
    $$
    \TODO Show this more rigorously
  \item
    $$
    \lim \left(\frac{\left(a_{n}+2\right)^{2}-4}{a_{n}}\right) = \lim \left(\frac{a_{n}^2 + 2a_n}{a_{n}}\right) = \lim \left(a_n + 2\right) = \infty
    $$
  \item This one is a straightforward application of the algebraic limit theorem
    $$
    \lim \left(\frac{\frac{2}{a_{n}}+3}{\frac{1}{a_{n}}+5}\right) = 3/5
    $$
  }
\end{solution}


\begin{exercise}
  Let $\left(x_{n}\right)$ and $\left(y_{n}\right)$ be given, and define $\left(z_{n}\right)$ to be the ``shuffled'' sequence $\left(x_{1}, y_{1}, x_{2}, y_{2}, x_{3}, y_{3}, \ldots, x_{n}, y_{n}, \ldots\right)$. Prove that $\left(z_{n}\right)$ is convergent if and only if $\left(x_{n}\right)$ and $\left(y_{n}\right)$ are both convergent with $\lim x_{n}=\lim y_{n}$.
\end{exercise}


\begin{solution}
  \TODO
\end{solution}

\begin{exercise}
  Consider the sequence given by $b_{n}=n-\sqrt{n^{2}+2 n}$. Taking $(1 / n) \rightarrow 0$ as given, and using both the Algebraic Limit Theorem and the result in Exercise 2.3.1, show $\lim b_{n}$ exists and find the value of the limit.
\end{exercise}

\begin{solution}
  \TODO
\end{solution}

\begin{exercise}
  Give an example of each of the following, or state that such a request is impossible by referencing the proper theorem(s):
  \enum{
  \item sequences $\left(x_{n}\right)$ and $\left(y_{n}\right)$, which both diverge, but whose sum $\left(x_{n}+y_{n}\right)$ converges;
  \item sequences $\left(x_{n}\right)$ and $\left(y_{n}\right)$, where $\left(x_{n}\right)$ converges, $\left(y_{n}\right)$ diverges, and $\left(x_{n}+y_{n}\right)$ converges;
  \item a convergent sequence $\left(b_{n}\right)$ with $b_{n} \neq 0$ for all $n$ such that $\left(1 / b_{n}\right)$ diverges;
  \item an unbounded sequence $\left(a_{n}\right)$ and a convergent sequence $\left(b_{n}\right)$ with $\left(a_{n}-b_{n}\right)$ bounded;
  \item two sequences $\left(a_{n}\right)$ and $\left(b_{n}\right)$, where $\left(a_{n} b_{n}\right)$ and $\left(a_{n}\right)$ converge but $\left(b_{n}\right)$ does not.
  }
\end{exercise}

\begin{solution}
  \enum{
  \item \TODO
  \item \TODO
  \item \TODO
  \item \TODO
  }
\end{solution}

\begin{exercise}
  Let $\left(x_{n}\right) \rightarrow x$ and let $p(x)$ be a polynomial.
  \enum{
  \item Show $p\left(x_{n}\right) \rightarrow p(x)$.
  \item Find an example of a function $f(x)$ and a convergent sequence $\left(x_{n}\right) \rightarrow x$ where the sequence $f\left(x_{n}\right)$ converges, but not to $f(x)$.
  }
\end{exercise}

\begin{solution}
  \enum{
  \item \TODO
  \item \TODO
  }
\end{solution}

\begin{exercise}
  \enum{
  \item Let $\left(a_{n}\right)$ be a bounded (not necessarily convergent) sequence, and assume $\lim b_{n}=0$. Show that $\lim \left(a_{n} b_{n}\right)=0$. Why are we not allowed to use the Algebraic Limit Theorem to prove this?
  \item Can we conclude anything about the convergence of $\left(a_{n} b_{n}\right)$ if we assume that $\left(b_{n}\right)$ converges to some nonzero limit $b$ ?
  \item Use (a) to prove Theorem 2.3.3, part (iii), for the case when $a=0$.
  }
\end{exercise}

\begin{solution}
  \enum{
  \item \TODO
  \item \TODO
  \item \TODO
  }
\end{solution}

\begin{exercise}
  Consider the following list of conjectures. Provide a short proof for those that are true and a counterexample for any that are false.
  \enum{
  \item If $\lim \left(a_{n}-b_{n}\right)=0$, then $\lim a_{n}=\lim b_{n}$.
  \item If $\left(b_{n}\right) \rightarrow b$, then $\left|b_{n}\right| \rightarrow|b|$.
  \item If $\left(a_{n}\right) \rightarrow a$ and $\left(b_{n}-a_{n}\right) \rightarrow 0$, then $\left(b_{n}\right) \rightarrow a$.
  \item If $\left(a_{n}\right) \rightarrow 0$ and $\left|b_{n}-b\right| \leq a_{n}$ for all $n \in \mathbf{N}$, then $\left(b_{n}\right) \rightarrow b$.
  }
\end{exercise}

\begin{solution}
  \enum{
  \item \TODO
  \item \TODO
  \item \TODO
  \item \TODO
  }
\end{solution}

\begin{exercise}[Cesaro Means]
  \enum {
  \item Show that if $\left(x_{n}\right)$ is a convergent sequence, then the sequence given by the averages
    $$
    y_{n}=\frac{x_{1}+x_{2}+\cdots+x_{n}}{n}
    $$
    also converges to the same limit.
  \item Give an example to show that it is possible for the sequence $\left(y_{n}\right)$ of averages to converge even if $\left(x_{n}\right)$ does not.
  }
\end{exercise}

\begin{solution}
  \enum{
  \item \TODO
  \item \TODO
  }
\end{solution}


\begin{exercise}
  A typical task in analysis is to decipher whether a property possessed by every term in a convergent sequence is necessarily inherited by the limit. Assume $\left(a_{n}\right) \rightarrow a$, and determine the validity of each claim. Try to produce a counterexample for any that are false.

  \enum{
  \item If every $a_{n}$ is an upper bound for a set $B$, then $a$ is also an upper bound for $B$.
  \item If every $a_{n}$ is in the complement of the interval $(0,1)$, then $a$ is also in the complement of $(0,1)$.
  \item If every $a_{n}$ is rational, then $a$ is rational.
  }
\end{exercise}

\begin{solution}
  \enum{
  \item \TODO
  \item \TODO
  \item \TODO
  }
\end{solution}

\begin{exercise}[Iterated Limits]
  Given a doubly indexed array $a_{m n}$ where $m, n \in \mathbf{N}$, what should $\lim _{m, n \rightarrow \infty} a_{m n}$ represent?
  \enum{
  \item Let $a_{m n}=m /(m+n)$ and compute the iterated limits
  $$
  \lim _{n \rightarrow \infty}\left(\lim _{n \rightarrow \infty} a_{m n}\right) \quad \text { and } \lim _{m \rightarrow \infty}\left(\lim _{n \rightarrow \infty} a_{m n}\right)
  $$
  Define $\lim _{m, n \rightarrow \infty} a_{m n}=a$ to mean that for all $\epsilon>0$ there exists an $N \in \mathbf{N}$ such that if both $m, n \geq N$, then $\left|a_{m n}-a\right|<\epsilon$
  \item Let $a_{m n}=1 /(m+n)$. Does $\lim _{m, n \rightarrow \infty} a_{m n}$ exist in this case? Do the two iterated limits exist? How do these three values compare? Answer these same questions for $a_{m n}=m n /\left(m^{2}+n^{2}\right)$
  \item Produce an example where $\lim _{m, n \rightarrow \infty} a_{m n}$ exists but where neither iterated limit can be computed.
  \item Assume $\lim _{m, n \rightarrow \infty} a_{m n}=a_{1}$ and assume that for each fixed $m \in \mathbf{N}$, $\lim _{n \rightarrow \infty}\left(a_{m n}\right) \rightarrow b_{m}$. Show $\lim _{m \rightarrow \infty} b_{m}=a$
  \item Prove that if $\lim _{m, n \rightarrow \infty} a_{m n}$ exists and the iterated limits both exist, then all three limits must be equal.
  }
\end{exercise}

\begin{solution}
  \enum{
  \item \TODO
  \item \TODO
  \item \TODO
  \item \TODO
  \item \TODO
  }
\end{solution}
