\section{Subsequences and the Bolzano–Weierstrass Theorem}


\begin{exercise}
  Give an example of each of the following, or argue that such a request is impossible.
  \enum{
  \item A sequence that has a subsequence that is bounded but contains no subsequence that converges.
  \item A sequence that does not contain 0 or 1 as a term but contains subsequences converging to each of these values.
  \item A sequence that contains subsequences converging to every point in the infinite set $\{1,1 / 2,1 / 3,1 / 4,1 / 5, \ldots\}$.
  \item A sequence that contains subsequences converging to every point in the infinite set $\{1,1 / 2,1 / 3,1 / 4,1 / 5, \ldots\}$, and no subsequences converging to points outside of this set.
  }
\end{exercise}

\begin{solution}
  \enum{
    \item Impossible, the Bolzano–Weierstrass theorem tells us a convergent subsequence of that subsequence exists, and that sub-sub sequence is also a subsequence of the original sequence.
    \item $(1 + 1/n) \to 1$ and $(1/n) \to 0$ so $(1/2, 1+1/2, 1/3, 1 + 1/3, \dots)$ has subsequences converging to $0$ and $1$.
    \item Copy the finitely many previous terms before proceeding to a new term
      $$(1, 1/2, 1, 1/3, 1, 1/2, 1/4, 1, 1/2, 1/3, \dots)$$
      The sequence contains infinitely many terms in $\{1, 1/2, 1/3, \dots\}$ hence subsequences exist converging to each of these values.
    \item Impossible, the sequence must converge to zero which is not in the set.

      Proof: Let $\epsilon > 0$ be arbitrary, pick $N$ large enough that $1/n < \epsilon/2$ for $n > N$.
      We can find a subsequence $(b_m) \to 1/n$ meaning $|b_m - 1/n| < \epsilon/2$ for some $m$. using the triangle inequality we get
      $$
      |b_m - 0| \le |b_n - 1/n| + |1/n - 0| < \epsilon/2 + \epsilon/2 = \epsilon
      $$
      Therefor we have found a number $b_m$ in the sequence $a_m$ with $|b_m| < \epsilon$.
      This process can be repeated for any $\epsilon$ therefor a sequence which converges to zero can be constructed.
  }
\end{solution}


\begin{exercise}
  Decide whether the following propositions are true or false, providing a short justification for each conclusion.
  \enum{
  \item If every proper subsequence of $\left(x_{n}\right)$ converges, then $\left(x_{n}\right)$ converges as well.
  \item If $\left(x_{n}\right)$ contains a divergent subsequence, then $\left(x_{n}\right)$ diverges.
  \item If $\left(x_{n}\right)$ is bounded and diverges, then there exist two subsequences of $\left(x_{n}\right)$ that converge to different limits.
  \item If $\left(x_{n}\right)$ is monotone and contains a convergent subsequence, then $\left(x_{n}\right)$ converges.
  }
\end{exercise}

\begin{solution}
  \enum{
  \item True, removing the first term gives us the proper subsequence $(x_2, x_3, \dots)$ which converges, implying $(x_1, x_2, \dots)$ also converges.
  \item True, the divergent subsequence is unbounded, hence $(x_n)$ is also unbounded and divergent.
  \item True, since $x_n$ is bounded $\lim \sup x_n$ and $\lim \inf x_n$ both converge. And since $x_n$ diverges Exercise \ref{ex:lim_sup} tells us $\lim \sup x_n \ne \lim \inf x_n$.
  \item True, The subsequence $(x_{n_k})$ converges meaning it is bounded $|x_{n_k}| \le M$. Suppose $(x_n)$ is increasing, then $x_n$ is bounded since picking $k$ so that $n_k > n$ we have $x_n \le x_{n_k} \le M$. A similar argument applies if $x_n$ is decreasing, Therefor $x_n$ is monotonic bounded and so must converge.
  }
\end{solution}

\begin{exercise}
  \enum{
  \item Prove that if an infinite series converges, then the associative property holds. Assume $a_{1}+a_{2}+a_{3}+a_{4}+a_{5}+\cdots$ converges to a limit $L$ (i.e., the sequence of partial sums $\left.\left(s_{n}\right) \rightarrow L\right)$. Show that any regrouping of the terms
    $$
    \left(a_{1}+a_{2}+\cdots+a_{n_{1}}\right)+\left(a_{n_{1}+1}+\cdots+a_{n_{2}}\right)+\left(a_{n_{2}+1}+\cdots+a_{n_{3}}\right)+\cdots
    $$
    leads to a series that also converges to $L$.
  \item Compare this result to the example discussed at the end of Section $2.1$ where infinite addition was shown not to be associative. Why doesn't our proof in (a) apply to this example?
  }
\end{exercise}

\begin{solution}
  \enum{
  \item Let $s_n$ be the original partial sums, and let $s_m'$ be the regrouping.
    Since $s_m'$ is a subsequence of $s_n$, $(s_n) \to s$ implies $(s_m') \to s$.
  \item The subsequence $s_m' = (1 - 1) + \dots = 0$ converging does not imply the parent sequence $s_n$ converges.
    In fact BW tells us any bounded sequence of partial sums will have a convergent subsequence (regrouping in this case).
  }
\end{solution}

\begin{exercise}
  The Bolzano-Weierstrass Theorem is extremely important, and so is the strategy employed in the proof. To gain some more experience with this technique, assume the Nested Interval Property is true and use it to provide a proof of the Axiom of Completeness. To prevent the argument from being circular, assume also that $\left(1 / 2^{n}\right) \rightarrow 0$. (Why precisely is this last assumption needed to avoid circularity?)
\end{exercise}

\begin{solution}
  Let $A$ be a bounded set, we're basically going to binary search for $\sup A$ and then use NIP to prove the limit exists.

  Let $M$ be an upper bound on $A$, and pick any $L \in A$ as our starting lower bound for $\sup A$ and define $I_1 = [L, M]$. Doing binary search gives $I_{n+1} \subseteq I_n$ with length proportional to $(1/2)^n$. Applying the Nested Interval Property gives
  $$
  \bigcap_{n=1}^\infty I_n \ne \emptyset
  $$
  As the length $(1/2)^n$ goes to zero, there is a single $s \in \bigcap_{n=1}^\infty I_n$ which must be the least upper bound since $I_n = [L_n, M_n]$ gives $L_n \le x \le M_n$ for all $n$ meaning $s = \sup A$ since
  \enumr{
  \item $s \ge L_n$ implies $s$ is an upper bound
  \item $s \le M_n$ implies $s$ is the least upper bound
  }
\end{solution}

\begin{exercise}
  Assume $\left(a_{n}\right)$ is a bounded sequence with the property that every convergent subsequence of $\left(a_{n}\right)$ converges to the same limit $a \in \mathbf{R}$. Show that $\left(a_{n}\right)$ must converge to $a$.
\end{exercise}

\begin{solution}
  $(a_2, a_3, \dots)$ Is a convergent subsequence, so obviously if $(a_2, a_3, \dots) \to a$ then $(a_n) \to a$ also.
\end{solution}

\begin{exercise}
  Use a similar strategy to the one in Example 2.5.3 to show $\lim b^{1 / n}$ exists for all $b \geq 0$ and find the value of the limit. (The results in Exercise 2.3.1 may be assumed.)
\end{exercise}

\begin{solution}
  Intuitively $\lim b^{1/n} = 1$

  Two facts I'll take as granted (you can prove them if you wish)
  \enumr{
  \item If $b < 1$ then $b^{1/n}$ is increasing
  \item If $b > 1$ then $b^{1/n}$ is decreasing
  }
  Thus $b^{1/n}$ is monotonic, and bounded since
  \enumr{
  \item If $b > 1$ then $b^{1/n} > 1$ since $b > 1^n$
  \item If $b < 1$ then $b^{1/n} < 1$ since $b < 1^n$
  }
  Therefor $b^{1/n}$ converges by the monotone convergence theorem. to find the limit equate terms
  $$
  b^{1/{n+1}} = b^{1/n} \implies b^1 = b^{\frac{n+1}{n}} = b^2 \implies b=1
  $$
  \TODO Shorten this
\end{solution}

\begin{exercise}
  Extend the result proved in Example 2.5.3 to the case $|b|<1$; that is, show $\lim \left(b^{n}\right)=0$ if and only if $-1<b<1$.
\end{exercise}

\begin{solution}
  If $|b| \ge 1$ then $\lim (b^n) \ne 0$ (diverges for $b \ne 1$).

  Now for the other direction, if $|b| < 1$ we immediately get $|b^n| < 1$ thus $b^n$ is bounded.
  Since it is decreasing the monotone convergence theorem implies it converges.
  To find the limit equating terms $b^{n+1} = b^{n}$ gives $b = 0$ or $b = 1$, since $b$ is \emph{strictly} decreasing we have $b = 0$.
\end{solution}

\begin{exercise}
  Another way to prove the Bolzano-Weierstrass Theorem is to show that every sequence contains a monotone subsequence. A useful device in this endeavor is the notion of a peak term. Given a sequence $\left(x_{n}\right)$, a particular term $x_{m}$ is a peak term if no later term in the sequence exceeds it; i.e., if $x_{m} \geq x_{n}$ for all $n \geq m$.
  \enum{
  \item Find examples of sequences with zero, one, and two peak terms. Find an example of a sequence with infinitely many peak terms that is not monotone.
  \item Show that every sequence contains a monotone subsequence and explain how this furnishes a new proof of the Bolzano-Weierstrass Theorem.
  }
\end{exercise}

\begin{solution}
  \enum{
  \item $(1,2,\dots)$ has zero peak terms, $(1, 0, 1/2, 2/3, 3/4, \dots)$ has a single peak term, $(2, 1, 1/2, 2/3, \dots)$ has two peak terms (a similar argument works for $k$ peak terms) and $(1,1/2,1/3,\dots)$ has infinitely many peak terms.
    The sequence $(1, -1/2, 1/3, -1/4, \dots)$ has infinitely many peak terms, but is not monotone.
  \item The sequence of peak terms is monotonic decreasing, thus if the parent sequence is bounded we have found a subsequence which converges, hence proving BW.
    (If there aren't infinitely many peak terms, then take the sequence of valley terms)
  }
\end{solution}


\begin{exercise}
  Let $\left(a_{n}\right)$ be a bounded sequence, and define the set
  $$
  S=\left\{x \in \mathbf{R}: x<a_{n} \text { for infinitely many terms } a_{n}\right\}
  $$
  Show that there exists a subsequence $\left(a_{n_{k}}\right)$ converging to $s=\sup S$. (This is a direct proof of the Bolzano-Weierstrass Theorem using the Axiom of Completeness.)
\end{exercise}

\begin{solution}
  For every $\epsilon > 0$ there exists an $x \in S$ with $x > s - \epsilon$ implying $|s-x| < \epsilon$. Therefor we can get arbitrarily close to $s = \sup S$ so there is a subsequence converging to this value.

  To make this more rigorous, pick $x_n \in S$ such that $|x_n - s| < 1/n$ then pick $N > 1/\epsilon$ to get $|x_n - s| < \epsilon$ for all $n > N$.
\end{solution}
