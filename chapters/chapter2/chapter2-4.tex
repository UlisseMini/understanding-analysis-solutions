\section{The Monotone Convergence Theorem and a First Look at Infinite Series}

\begin{exercise}
  \enum{
  \item Prove that the sequence defined by $x_{1}=3$ and
    $$
    x_{n+1}=\frac{1}{4-x_{n}}
    $$
    converges.
  \item Now that we know $\lim x_{n}$ exists, explain why $\lim x_{n+1}$ must also exist and equal the same value.
  \item Take the limit of each side of the recursive equation in part (a) to explicitly compute $\lim x_{n}$.
  }
\end{exercise}

\begin{solution}
  \enum{
  \item $x_2 = 1$ makes me conjecture $x_n$ is monotonic. For induction suppose $x_n > x_{n+1}$ then we have
    $$
    4 - x_n < 4 - x_{n+1} \implies \frac{1}{4 - x_n} > \frac{1}{4 - x_{n+1}} \implies x_{n+1} > x_{n+2}
    $$
    Thus $x_n$ is decreasing, to show $x_n$ is bounded notice $x_n$ cannot be negative since $x_n < 3$ means $x_{n+1} = 1/(4-x_n) > 0$.
    therefore by the monotone convergence theorem $(x_n)$ converges.
  \item Clearly skipping a single term does not change what the series converges to.
  \item Since $x = \lim (x_n) = \lim (x_{n+1})$ we must have
    $$
    x = \frac{1}{4 - x} \iff x^2 - 4x + 1 = 0 \iff (x - 2)^2 = 3 \iff x = 2 \pm \sqrt 3
    $$
    $2 + \sqrt 3 > 3$ is impossible since $x_n < 3$ thus $x = 2 - \sqrt 3$.
  }
\end{solution}

\begin{exercise}
  \enum{
  \item Consider the recursively defined sequence $y_{1}=1$
  $$
  y_{n+1}=3-y_{n}
  $$
  and set $y=\lim y_{n} .$ Because $\left(y_{n}\right)$ and $\left(y_{n+1}\right)$ have the same limit, taking the limit across the recursive equation gives $y=3-y$. Solving for $y$, we conclude $\lim y_{n}=3 / 2$
  What is wrong with this argument?
  \item This time set $y_{1}=1$ and $y_{n+1}=3-\frac{1}{y_{n}}$. Can the strategy in (a) be applied to compute the limit of this sequence?
  }
\end{exercise}

\begin{solution}
  \enum{
  \item The sequence $y_n = (1, 2, 1, 2, \dots)$ does not converge.
  \item Yes, $y_n$ converges by the monotone convergence theorem since $0 < y_n < 3$ and $y_n$ is increasing.
  }
\end{solution}

\begin{exercise}
  \enum{
  \item Show that
    $$
    \sqrt{2}, \sqrt{2+\sqrt{2}}, \sqrt{2+\sqrt{2+\sqrt{2}}}, \ldots
    $$
    converges and find the limit.
  \item Does the sequence
    $$
    \sqrt{2}, \sqrt{2 \sqrt{2}}, \sqrt{2 \sqrt{2 \sqrt{2}}}, \ldots
    $$
    converge? If so, find the limit.
  }
\end{exercise}

\begin{solution}
  \enum{
  \item Let $x_1 = \sqrt 2$ and $x_{n+1} = \sqrt{2 + x_n}$ clearly $x_2 > x_1$. assuming $x_{n+1} > x_n$ gives
    $$
    2 + x_{n+1} > 2 + x_n \iff \sqrt{2 + x_{n+1}} > \sqrt{2 + x_n} \iff x_{n+2} > x_{n+1}
    $$
    Since $x_n$ is monotonically increasing and bounded the monotone convergence theorem tells us $(x_n) \to x$. Equating both sides like in 2.4.1 gives
    $$
    x = \sqrt{2 + x} \iff x^2 - x - 2 = 0 \iff x = \frac 12 \pm \frac{3}{2}
    $$
    Since $x > 0$ we must have $x = 2$.
  \item We have $x_1 = 2^{1/2}$ and $x_{n+1} = (2x_n)^{1/2}$. We have
    $$
    x_{n+1} = (2x_n)^{1/2} \ge x_n \iff 2x_n \ge x_n^2 \iff 2 \ge x_n
    $$
    Since $x_1 = 2^{1/2} \le 2$ induction implies $x_n$ is increasing. Now to show $x_n$ is bounded notice that $x_1 \le 2$ and if $x_n \le 2$ then
    $$
    2x_n \le 4 \implies (2x_n)^{1/2} \le 2
    $$
    Now the monotone convergence theorem tells us $(x_n)$ converges. To find the limit use $\lim x_n = \lim x_{n+1} = x$ to get
    $$
    x = (2x)^{1/2} \implies x^2 = 2x \implies x = \pm 2
    $$
    Since $x_n \ge 0$ we have $x = 2$.
  }
\end{solution}

\begin{exercise}
  \enum{
  \item In Section 1.4 we used the Axiom of Completeness (AoC) to prove the Archimedean Property of $\mathbf{R}$ (Theorem 1.4.2). Show that the Monotone Convergence Theorem can also be used to prove the Archimedean Property without making any use of AoC.
  \item Use the Monotone Convergence Theorem to supply a proof for the Nested Interval Property (Theorem 1.4.1) that doesn't make use of AoC.

  These two results suggest that we could have used the Monotone Convergence Theorem in place of $\mathrm{AoC}$ as our starting axiom for building a proper theory of the real numbers.
  }
\end{exercise}

\begin{solution}
  \enum{
  \item MCT tells us $(1/n)$ converges, obviously it must converge to zero therefore we have $|1/n - 0| = 1/n < \epsilon$ for any $\epsilon$, which is the Archimedean Property.
  \item We have $I_n = [a_n, b_n]$ with $a_n \le b_n$ since $I_n\ne \emptyset$. Since $I_{n+1} \subseteq I_n$ we must have $b_{n+1} \le b_n$ and $a_{n+1} \ge a_n$ the MCT tells us that $(a_n) \to a$ and $(b_n) \to b$. by the Order Limit Theorem we have $a \le b$ since $a_n \le b_n$, therefore $a \in I_n$ for all $n$ meaning $a \in \bigcap_{n=1}^\infty I_n$ and thus $\bigcap_{n=1}^\infty I_n \ne \emptyset$.
  }
\end{solution}

\begin{exercise}[Calculating Square Roots]
  Let $x_{1}=2$, and define
  $$
  x_{n+1}=\frac{1}{2}\left(x_{n}+\frac{2}{x_{n}}\right)
  $$
  \enum{
  \item Show that $x_{n}^{2}$ is always greater than or equal to 2 , and then use this to prove that $x_{n}-x_{n+1} \geq 0$. Conclude that $\lim x_{n}=\sqrt{2}$.
  \item Modify the sequence $\left(x_{n}\right)$ so that it converges to $\sqrt{c}$.
  }
\end{exercise}

\begin{solution}
  \enum{
  \item Clearly $x_1^2 \ge 2$, now procede by induction. if $x_{n}^2 \ge 2$ then we have
    $$
    x_{n+1}^2 = \frac 14\left(\frac{x_n^2 + 2}{x_n}\right)^2
    = \frac 14\left(\frac{(x_n^2 + 2)^2}{x_n^2}\right)
    \ge \frac 14\left(\frac{(x_n^2 + 2)^2}{2}\right)
    $$
    Now since $x_n^2 \ge 2$ we have $(x_n^2 + 2)^2 \ge 16$ meaning
    $$
    x_{n+1}^2 = \frac 14\left(\frac{(x_n^2 + 2)^2}{2}\right) \ge 2.
    $$
    Now to show $x_n - x_{n+1} \ge 0$ we use $x_n \ge 0$
    $$
    \begin{aligned}
      x_n - x_{n+1} &= x_n - \frac 12\left(x_n + \frac{2}{x_n}\right) \\
                    &= \frac 12 x_n + \frac{1}{x_n} \ge 0
    \end{aligned}
    $$
    Now we know $(x_n) \to x$ converges by MCT, to show $x^2 = 2$ we equate $x_n = x_{n+1}$ (true in the limit since $|x_n - x_{n+1}|$ becomes arbitrarily small)
    $$
    x = \frac 12 \left(x + \frac{2}{x}\right) \iff x^2 = \frac 12 x^2 + 1 \iff x^2 = 2
    $$
    therefore $x = \pm \sqrt 2$, and since every $x_n$ is positive $x = \sqrt 2$.
  \item Let
    $$
    x_{n+1} = \frac{1}{2}\left(x_n + \frac{c}{x_n}\right)
    $$
    I won't go through the convergence analysis again, but the only fixed point is
    $$
    x = \frac 12\left(x + \frac{c}{x}\right) \implies \frac 12 x^2 = \frac 12 c \implies x^2 = c
    $$
    So if $x_n$ converges, it must converge to $x^2 = c$.
  }
\end{solution}

\begin{exercise}[Arithmetic-Geometric Mean]
  \enum{
  \item Explain why $\sqrt{x y} \leq$ $(x+y) / 2$ for any two positive real numbers $x$ and $y$. (The geometric mean is always less than the arithmetic mean.)
  \item Now let $0 \leq x_{1} \leq y_{1}$ and define
    $$
    x_{n+1}=\sqrt{x_{n} y_{n}} \quad \text { and } \quad y_{n+1}=\frac{x_{n}+y_{n}}{2}
    $$
    Show $\lim x_{n}$ and $\lim y_{n}$ both exist and are equal.
  }

\end{exercise}

\begin{solution}
  \enum{
  \item We have
    $$
    \sqrt{xy} \le (x+y)/2 \iff 4xy \le x^2 + 2xy + y^2 \iff 0 \le (x - y)^2
    $$
  \item The only fixed point is $x_n = y_n$ so we only need to show both sequences converge.

    The inequality $x_1 \le y_1$ is always true since
    $$
    \sqrt{x_ny_n} \le \frac{x_n + y_n}{2} \implies x_{n+1} \le y_{n+1}
    $$
    Also $x_n \le y_n$ implies $(x_n+y_n)/2 = y_{n+1} \le y_n$, similarly $\sqrt{x_ny_n} = x_{n+1} \ge x_n$ meaning both sequences converge by the monotone convergence theorem.
  }
\end{solution}

\begin{exercise}[Limit Superior]
  \label{ex:lim_sup}
  Let $\left(a_{n}\right)$ be a bounded sequence.

  \enum{
  \item Prove that the sequence defined by $y_{n}=\sup \left\{a_{k}: k \geq n\right\}$ converges.
  \item The limit superior of $\left(a_{n}\right)$, or $\lim \sup a_{n}$, is defined by
    $$
    \limsup a_{n}=\lim y_{n}
    $$
    where $y_{n}$ is the sequence from part (a) of this exercise. Provide a reasonable definition for $\lim \inf a_{n}$ and briefly explain why it always exists for any bounded sequence.
  \item Prove that $\lim \inf a_{n} \leq \lim \sup a_{n}$ for every bounded sequence, and give an example of a sequence for which the inequality is strict.
  \item Show that $\lim \inf a_{n}=\lim \sup a_{n}$ if and only if $\lim a_{n}$ exists. In this case, all three share the same value.
  }
\end{exercise}

\begin{solution}
  \enum{
  \item $(y_n)$ is decreasing and converges by the monotone convergence theorem.
  \item Define $\lim \inf a_n = \lim z_n$ for $z_n = \inf\{a_n : k \ge n\}$. $z_n$ converges since it is increasing and bounded.
  \item Obviously $\inf\{a_k : k \ge n\} \le \sup\{a_n : k \ge n\}$ so by the Order Limit Theorem $\lim \inf a_n \le \lim \sup a_n$.
  \item If $\lim \inf a_n = \lim \sup a_n$ then the squeeze theorem (Exercise 2.3.3) implies $a_n$ converges to the same value, since $\inf\{a_{k\ge n}\} \le a_n \le \sup\{a_{k\ge n}\}$.
  }
\end{solution}

\begin{exercise}
  For each series, find an explicit formula for the sequence of partial sums and determine if the series converges.
  \enum{
  \item $\sum_{n=1}^{\infty} \frac{1}{2^{n}}$
  \item $\sum_{n=1}^{\infty} \frac{1}{n(n+1)}$
  \item $\sum_{n=1}^{\infty} \log \left(\frac{n+1}{n}\right)$
  }
  (In (c), $\log (x)$ refers to the natural logarithm function from calculus.)
\end{exercise}

\begin{solution}
  \enum{
  \item This is a geometric series, we can use the usual trick to derive $s_n$. Let $r = 1/2$ for convenience
    $$
    \begin{aligned}
      s_n  &= 1 + r + r^2 + \dots + r^n \\
      rs_n &= r + r^2 + \dots + r^{n+1} \\
      rs_n - s_n &= r^{n+1} - 1 \implies s_n = \frac{r^{n+1} - 1}{r - 1}
    \end{aligned}
    $$
    This is the formula when $n$ starts at zero, but the sum in question starts at one so we subtract the first term to correct this
    $$
    \sum_{n=1}^\infty \frac{1}{2^n}
    = -1 + \sum_{n=0}^\infty \frac{1}{2^n}
    = -1 + \lim_{n\to \infty} \frac{(1/2)^{n+1} - 1}{1/2 - 1}
    = -1 + \frac{-1}{-1/2} = 1
    $$
  \item We can use partial fractions to get
    $$
    \frac{1}{n(n+1)} = \frac{1}{n} -\frac{1}{n+1}
    $$
    Which gives us a telescoping series, most of the terms cancel and we get
    $$
    s_n = 1 - \frac{1}{n+1}
    $$
    Therefor
    $$
    \sum_{n=1}^\infty \frac{1}{n(n+1)} = \lim_{n \to \infty} \left(1 - \frac{1}{n+1}\right) = 1
    $$
  \item Another telescoping series, since
    $$
    \log\left(\frac{n+1}{n}\right) = \log(n+1) - \log(n)
    $$
    therefore most of the terms cancel and we get
    $$
    s_n = \log(n+1)
    $$
    Which doesn't converge.
  }
\end{solution}

\begin{exercise}
  Complete the proof of Theorem 2.4.6 by showing that if the series $\sum_{n=0}^{\infty} 2^{n} b_{2^{n}}$ diverges, then so does $\sum_{n=1}^{\infty} b_{n}$. Example $2.4 .5$ may be a useful reference.
\end{exercise}

\begin{solution}
  Let $s_n = b_1 + b_2 + \dots + b_n$ and $t_k = b_1 + 2b_2 + \dots + 2^kb_{2^k}$.

  We want to show $s_n$ is unbounded, first we find a series similar to $t_k$ that is less then $s_n$, then rewrite it in terms of $t_k$.

  Let $n = 2^k$ so things match up nicely. We get
  $$
  \begin{aligned}
  s_n
  &=   b_1 + b_2 + (b_3 + b_4) + \dots + (b_{2^{k-1}} + \dots + b_{2^k}) \\
  &\le b_1 + b_2 + (b_4 + b_4) + \dots + 2^{k-1}b_{2^k}
  \end{aligned}
  $$
  (Notice there are $2^k - 2^{k-1} = 2^{k-1}$ terms in the last term)

  Now define $t_k'$ to be our new series $b_1 + b_2 + 2b_4 + 4b_8 + \dots + 2^{k-1}b_{2^k}$.
  This looks a lot like $t_k$, and in fact some algebra gives
  $$
  t_k'
  = \frac 12 \left(b_1 + 2b_2 + 4b_4 + \dots + 2^kb_k\right) + \frac 12 b_1
  = \frac 12 t_k + \frac 12 b_1
  $$
  therefore we are justified in writing
  $$
  s_n \ge t_k' \ge \frac 12 t_k
  $$
  And since $t_k/2$ diverges and $s_n$ is bigger, $s_n$ must also diverge.

  \textbf{Summary}: $s_n$ converges iff $t_k$ conv since $t_k \ge s_n \ge t_k/2$ for $n = 2^k$.
\end{solution}

\begin{exercise}[Infinite Products]
  A close relative of infinite series is the infinite product
  $$
  \prod_{n=1}^{\infty} b_{n}=b_{1} b_{2} b_{3} \cdots
  $$
  which is understood in terms of its sequence of partial products
  $$
  p_{m}=\prod_{n=1}^{m} b_{n}=b_{1} b_{2} b_{3} \cdots b_{m}
  $$

  Consider the special class of infinite products of the form
  $$
  \prod_{n=1}^{\infty}\left(1+a_{n}\right)=\left(1+a_{1}\right)\left(1+a_{2}\right)\left(1+a_{3}\right) \cdots, \quad \text { where } a_{n} \geq 0
  $$
  \enum{
  \item Find an explicit formula for the sequence of partial products in the case where $a_{n}=1 / n$ and decide whether the sequence converges. Write out the first few terms in the sequence of partial products in the case where $a_{n}=1 / n^{2}$ and make a conjecture about the convergence of this sequence.
  \item Show, in general, that the sequence of partial products converges if and only if $\sum_{n=1}^{\infty} a_{n}$ converges. (The inequality $1+x \leq 3^{x}$ for positive $x$ will be useful in one direction.)
  }
\end{exercise}

\begin{solution}
  \enum{
  \item This is a telescoping product, most of the terms cancel
    $$
    p_m = \prod_{n=1}^m (1 + 1/n) = \prod_{n=1}^m \frac{n+1}{n} = \frac 21 \cdot \frac 32 \cdot \frac 42 \cdots \frac {m+1}{m} = m+1
    $$
    therefore $(p_m)$ diverges.

    In the cast $a_n = 1/n^2$ we get
    $$
    \prod_{n=1}^\infty (1+ 1/n^2) = \prod_{n=1}^\infty \frac{1 + n^2}{n^2} = \frac 21 \cdot \frac 54 \cdot \frac{10}{9} \cdots
    $$ 
    The growth seems slower, I conjecture it converges now.
  \item Using the inequality suggested we have $1 + a_n \le 3^{a_n}$ letting $s_m = a_1 + \dots + a_m$ we get
    $$
    p_m = (1+a_1)\cdots(1+a_m) \le 3^{a_1}3^{a_2}\cdots 3^{a_m} = 3^{s_m}
    $$
    Now if $s_m$ converges it is bounded by some $M$ meaning $p_m$ is bounded by $3^M$. and because $a_n \ge 0$ the partial products $p_m$ are increasing, so they converge by the MCT. This shows $s_m$ converging implies $p_m$ converges.

    For the other direction suppose $p_m \to p$. Distributing inside the products gives $p_2 = a_1 + a_2 + 1 + a_1a_2 > s_2$ and in general $p_m > s_m$ implying that if $p_m$ is bounded then $s_n$ is bounded aswell. This completes the proof.

    \textbf{Summary}: Convergence is if and only if because $s_m \le p_m \le 3^{s_m}$.

    (By the way the inequality $1 + x \le 3^x$ can be derived from $\log(1 + x) \le x$ implying $1 + x \le e^x$, I assume abbott rounded up to $3$.)
  }
\end{solution}
