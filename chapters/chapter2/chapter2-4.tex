\section{The Monotone Convergence Theorem and a First Look at Infinite Series}

\begin{exercise}
  \enum{
  \item Prove that the sequence defined by $x_{1}=3$ and
    $$
    x_{n+1}=\frac{1}{4-x_{n}}
    $$
    converges.
  \item Now that we know $\lim x_{n}$ exists, explain why $\lim x_{n+1}$ must also exist and equal the same value.
  \item Take the limit of each side of the recursive equation in part (a) to explicitly compute $\lim x_{n}$.
  }
\end{exercise}

\begin{solution}
  \enum{
  \item \TODO
  \item \TODO
  \item \TODO
  }
\end{solution}

\begin{exercise}
  \enum{
  \item Consider the recursively defined sequence $y_{1}=1$
  $$
  y_{n+1}=3-y_{n}
  $$
  and set $y=\lim y_{n} .$ Because $\left(y_{n}\right)$ and $\left(y_{n+1}\right)$ have the same limit, taking the limit across the recursive equation gives $y=3-y$. Solving for $y$, we conclude $\lim y_{n}=3 / 2$
  What is wrong with this argument?
  \item This time set $y_{1}=1$ and $y_{n+1}=3-\frac{1}{y_{n}} .$ Can the strategy in (a) be applied to compute the limit of this sequence?
  }
\end{exercise}

\begin{solution}
  \enum{
  \item \TODO
  \item \TODO
  }
\end{solution}

\begin{exercise}
  \enum{
  \item Show that
    $$
    \sqrt{2}, \sqrt{2+\sqrt{2}}, \sqrt{2+\sqrt{2+\sqrt{2}}}, \ldots
    $$
    converges and find the limit.
  \item Does the sequence
    $$
    \sqrt{2}, \sqrt{2 \sqrt{2}}, \sqrt{2 \sqrt{2 \sqrt{2}}}, \ldots
    $$
    converge? If so, find the limit.
  }
\end{exercise}

\begin{solution}
  \enum{
  \item \TODO
  \item \TODO
  }
\end{solution}

\begin{exercise}
  \enum{
  \item In Section 1.4 we used the Axiom of Completeness (AoC) to prove the Archimedean Property of $\mathbf{R}$ (Theorem 1.4.2). Show that the Monotone Convergence Theorem can also be used to prove the Archimedean Property without making any use of AoC.
  \item Use the Monotone Convergence Theorem to supply a proof for the Nested Interval Property (Theorem 1.4.1) that doesn't make use of AoC.

  These two results suggest that we could have used the Monotone Convergence Theorem in place of $\mathrm{AoC}$ as our starting axiom for building a proper theory of the real numbers.
  }
\end{exercise}

\begin{solution}
  \enum{
  \item \TODO
  \item \TODO
  }
\end{solution}

\begin{exercise}[Calculating Square Roots]
  Let $x_{1}=2$, and define
  $$
  x_{n+1}=\frac{1}{2}\left(x_{n}+\frac{2}{x_{n}}\right)
  $$
  \enum{
  \item Show that $x_{n}^{2}$ is always greater than or equal to 2 , and then use this to prove that $x_{n}-x_{n+1} \geq 0$. Conclude that $\lim x_{n}=\sqrt{2}$.
  \item Modify the sequence $\left(x_{n}\right)$ so that it converges to $\sqrt{c}$.
  }
\end{exercise}

\begin{solution}
  \enum{
  \item \TODO
  \item \TODO
  }
\end{solution}

\begin{exercise}[Arithmetic-Geometric Mean]
  \enum{
  \item Explain why $\sqrt{x y} \leq$ $(x+y) / 2$ for any two positive real numbers $x$ and $y$. (The geometric mean is always less than the arithmetic mean.)
  \item Now let $0 \leq x_{1} \leq y_{1}$ and define
    $$
    x_{n+1}=\sqrt{x_{n} y_{n}} \quad \text { and } \quad y_{n+1}=\frac{x_{n}+y_{n}}{2}
    $$
    Show $\lim x_{n}$ and $\lim y_{n}$ both exist and are equal.
  }

\end{exercise}

\begin{solution}
  \enum{
  \item \TODO
  \item \TODO
  }
\end{solution}

\begin{exercise}[Limit Superior]
  Let $\left(a_{n}\right)$ be a bounded sequence.

  \enum{
  \item Prove that the sequence defined by $y_{n}=\sup \left\{a_{k}: k \geq n\right\}$ converges.
  \item The limit superior of $\left(a_{n}\right)$, or $\lim \sup a_{n}$, is defined by
    $$
    \limsup a_{n}=\lim y_{n}
    $$
    where $y_{n}$ is the sequence from part (a) of this exercise. Provide a reasonable definition for $\lim \inf a_{n}$ and briefly explain why it always exists for any bounded sequence.
  \item Prove that $\lim \inf a_{n} \leq \lim \sup a_{n}$ for every bounded sequence, and give an example of a sequence for which the inequality is strict.
  \item Show that $\lim \inf a_{n}=\lim \sup a_{n}$ if and only if $\lim a_{n}$ exists. In this case, all three share the same value.
  }
\end{exercise}

\begin{solution}
  \enum{
  \item \TODO
  \item \TODO
  \item \TODO
  \item \TODO
  }
\end{solution}

\begin{exercise}
  For each series, find an explicit formula for the sequence of partial sums and determine if the series converges.
  \enum{
  \item $\sum_{n=1}^{\infty} \frac{1}{2^{n}}$
  \item $\sum_{n=1}^{\infty} \frac{1}{n(n+1)}$
  \item $\sum_{n=1}^{\infty} \log \left(\frac{n+1}{n}\right)$
  }
  (In (c), $\log (x)$ refers to the natural logarithm function from calculus.)
\end{exercise}

\begin{solution}
  \enum{
  \item \TODO
  \item \TODO
  \item \TODO
  }
\end{solution}

\begin{exercise}
  Complete the proof of Theorem 2.4.6 by showing that if the series $\sum_{n=0}^{\infty} 2^{n} b_{2^{n}}$ diverges, then so does $\sum_{n=1}^{\infty} b_{n}$. Example $2.4 .5$ may be a useful reference.
\end{exercise}

\begin{solution}
  \TODO
\end{solution}

\begin{exercise}[Infinite Products]
  A close relative of infinite series is the infinite product
  $$
  \prod_{n=1}^{\infty} b_{n}=b_{1} b_{2} b_{3} \cdots
  $$
  which is understood in terms of its sequence of partial products
  $$
  p_{m}=\prod_{n=1}^{m} b_{n}=b_{1} b_{2} b_{3} \cdots b_{m}
  $$

  Consider the special class of infinite products of the form
  $$
  \prod_{n=1}^{\infty}\left(1+a_{n}\right)=\left(1+a_{1}\right)\left(1+a_{2}\right)\left(1+a_{3}\right) \cdots, \quad \text { where } a_{n} \geq 0
  $$
  \enum{
  \item Find an explicit formula for the sequence of partial products in the case where $a_{n}=1 / n$ and decide whether the sequence converges. Write out the first few terms in the sequence of partial products in the case where $a_{n}=1 / n^{2}$ and make a conjecture about the convergence of this sequence.
  \item Show, in general, that the sequence of partial products converges if and only if $\sum_{n=1}^{\infty} a_{n}$ converges. (The inequality $1+x \leq 3^{x}$ for positive $x$ will be useful in one direction.)
  }
\end{exercise}

\begin{solution}
  \enum{
  \item \TODO
  \item \TODO
  }
\end{solution}
