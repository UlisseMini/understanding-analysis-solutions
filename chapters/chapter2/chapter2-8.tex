\section{Double Summations and Products of Infinite Series}

\begin{exercise}
    Using the particular array \(a_{ij}\) from Section 2.1, compute \(\lim_{n \to \infty} s_{nn}\). How does this value compare to the two iterated values for the
sum already computed?
\end{exercise}

\begin{solution}
By inspection \(s_{nn} = -2 + 1/2^{n - 1}\), so \(lim_{n \to \infty} s_{nn} = -2\). This is the same as the result when fixing \(j\) and summing down each column, since each column series has finitely many non-zero elements.
\end{solution}

\begin{exercise}
    Show that if the iterated series
    \[ \sum^\infty_{i = 1} \sum^\infty_{j=1} \left| a_{ij} \right| \]
    converges (meaning that for each fixed \(i \in \mathbf{N}\) the series \(\sum^\infty_{j=1}|a_{ij}|\) converges to some real number \(b_i\), and the series \(\sum^\infty_{i=1}b_i\) converges as well), then the iterated series
    \[\sum^\infty_{i=1} \sum^\infty_{j=1} a_{ij}\]
    converges.
\end{exercise}

\begin{solution}
    Since \(\sum^\infty_{j=1}|a_{ij}|\) converges, \(c_i = \sum^\infty_{j=1}a_{ij}\) converges as well; moreover \(|c_i| \leq |b_i|\), so \(\sum^\infty_{i=1}c_i\) converges by comparison with \(\sum^\infty_{i=1}|b_i|\).
\end{solution}

\begin{exercise}
Define
\[
    t_{mn}=  \sum^m_{i=1}\sum^n_{j=1}|a_{ij}|
\]
\enum{
    \item Prove that \((t_{nn})\) converges.
    \item Now, use the fact that \((t_{nn})\) is a Cauchy sequence to argue that \((s_{nn})\) converges.
}
\end{exercise}

\begin{solution}
\enum{
    \item Note that \(t_{nn}\) is monotone increasing; moreover
    \[
         \sum^\infty_{i=1}\sum^\infty_{j=1}|a_{ij}|
    \geq \sum^\infty_{i=1}\sum^n_{j=1}     |a_{ij}|
    \geq \sum^n_{i=1}     \sum^n_{j=1}     |a_{ij}| = t_{nn}
    \]
    and therefore \(t_{nn}\) is bounded; by the Monotone Convergence Theorem \(t_{nn}\) converges.

    \item Since \((t_{nn})\) is a Cauchy sequence, for any \(\epsilon > 0\) there exists \(N\) such that if \(p>q>N\),
     \[
\begin{aligned}
    \epsilon > \left| t_{pp} - t_{qq} \right| &= \left| \sum^p_{i=1} \sum^p_{j=1} |a_{ij}| - \sum^q_{i=1} \sum^q_{j=1} |a_{ij}| \right| \\
    &= \left| \sum^p_{i=q+1} \sum^p_{j=1} |a_{ij}| + \sum^q_{i=1} \sum^p_{j=q+1} |a_{ij}|  + \sum^q_{i=1} \sum^q_{j=1} |a_{ij}| - \sum^q_{i=1} \sum^q_{j=1} |a_{ij}| \right| \\
    &= \left| \sum^p_{i=q+1} \sum^p_{j=1} |a_{ij}| + \sum^q_{i=1} \sum^p_{j=q+1} |a_{ij}|  \right| \\
    &\geq \left|\sum^p_{i=q+1} \sum^p_{j=1} a_{ij}  + \sum^q_{i=1} \sum^p_{j=q+1} a_{ij} \right| \\
    &= \left| \sum^p_{i=q+1} \sum^p_{j=1} a_{ij} + \sum^q_{i=1} \sum^p_{j=q+1} a_{ij}  + \sum^q_{i=1} \sum^q_{j=1} a_{ij} - \sum^q_{i=1} \sum^q_{j=1} a_{ij} \right|\\
    &= \left|\sum^p_{i=1} \sum^p_{j=1} a_{ij} - \sum^q_{i=1} \sum^q_{j=1} a_{ij} \right| \\
    &= \left|s_{pp} - s_{qq}\right|
\end{aligned}
\]

and therefore \(s_{nn}\) is also a Cauchy sequence and thus converges.
}
\end{solution}

\begin{exercise}
\enum{
    \item Let \(\epsilon > 0\) be arbitrary and argue that there exists an \(N_1 \in \mathbf{N}\) such that \(m, n \geq N_1\) implies \(B - \frac{\epsilon}{2} < t_{mn} \leq B\).
    \item Now, show that there exists an \(N\) such that
    \[|s_{mn} - S| < \epsilon\]
    for all \(m, n \geq N\).
}
\end{exercise}

\begin{solution}
\enum{
    \item \(t_{mn} \leq B\) follows from the fact that \(B\) is an upper bound on \(\{t_{mn} : m, n \in \mathbf{N}\}\).
    Lemma 1.3.8 indicates that there exists some \(p, q\) such that \(t_{pq} > B - \epsilon/2\), and since \(p_1 \leq p_2 \text{ and } q_1 \leq q_2 \implies t_{p_1q_1} \leq t_{p_2q_2}\), we can choose \(N_1 = \max\{p, q\}\).

    \item By the triangle inequality, \(|s_{mn} - S| \leq |s_{mn} - s_{nn}| + |s_{nn} - S|\). Letting \(n' = \min\{n, m\}\) and \(m' = \max\{n, m\}\),
\[
    \left|s_{mn} - s_{nn}\right| = \left| \sum^{m'}_{i=n'} \sum^n_{j=1} a_{ij} \right| \leq \sum^{m'}_{i=n'} \sum^n_{j=1} |a_{ij}| = \left|t_{mn} - t_{nn}\right| \leq \frac{\epsilon}{2}
\]
from Exercise 2.8.4a), as long as \(m,n \geq N_1\). Since \(S = \lim_{n\to\infty} s_{nn}\) there exists \(N_2\) such that for \(n \geq N_2\), \(|s_{nn} - S| < \epsilon/2\); thus picking \(N = \max\{N_1, N_2\}\) ensures
    \[|s_{mn} - S| < \epsilon\]
    for all \(m, n \geq N\).

}
\end{solution}

\begin{exercise}
\enum{
    \item Show that for all \(m \geq N\)
    \[|(r_1 + r_2 + \cdots + r_m) - S| \leq \epsilon\]
    Conclude that the iterated sum \(\sum^\infty_{i=1}\sum^\infty_{j=1}a_{ij}\) converges to \(S\).

    \item Finish the proof by showing that the other iterated sum, \(\sum^\infty_{j=1}\sum^\infty_{i=1}a_{ij}\), converges to \(S\) as well. Notice that the same argument can be used once it is established that, for each fixed column \(j\), the sum \(\sum^\infty_{i=1}a_{ij}\) converges to some real number \(c_j\).

}
\end{exercise}

\begin{solution}
\enum{
\item For any given \(m\), there must be some \(N_3\) such that for \(n > N_3\), \(k \in \mathbf{N} \leq m\),
\[ \left| r_k - \sum^n_{j=1} a_{kj}\right| < \frac{\epsilon}{2m} \]

Then there must exist some \(N\) such that when \(m,n \geq N\),

\[
\begin{aligned}
    \left| \sum^m_{i=1} r_i - S \right|
    & \leq \left|\sum^m_{i=1}r_i - \sum^m_{i=1}\sum^n_{j=1}a_{ij}\right| + \left|\sum^m_{i=1}\sum^n_{j=1}a_{ij} - S \right|\\
    &= \left|\sum^m_{i=1} \left(r_i - \sum^n_{j=1}a_{ij}\right)\right| + \left|s_{mn} - S \right|\\
    & \leq \sum^m_{i=1}\left| r_k - \sum^n_{j=1} a_{kj}\right| + \left|s_{mn} - S \right| \\
    &< \sum^m_{i=1} \left(\frac{\epsilon}{2m}\right) + \frac{\epsilon}{2} = \epsilon
\end{aligned}
\]

and thus \(\sum^\infty_{i=1}\sum^\infty_{j=1}a_{ij}\) converges to \(S\).

\item \(\sum^\infty_{i=1}|a_{ij}|\) converges for any fixed \(j\) by comparison with \(\sum^\infty_{i=1}\sum^\infty_{k=1}|a_{ik}|\) which converges by the hypothesis, and thus \(\sum^\infty_{i=1}a_{ij}\) converges to some real number \(c_j\).
Then a similar argument to (a) can be used to show that there must be some \(N\)such that when \(n \geq N\),
    \[\left|\sum^n_{j=1}c_j - S\right| \leq \epsilon\]
and thus \(\sum^\infty_{j=1}\sum^\infty_{i=1}a_{ij}\) converges to \(S\).
}

\end{solution}

\begin{exercise}
\enum{
    \item Assuming the hypothesis - and hence the conclusion - of Theorem 2.8.1, show that \(\sum^\infty_{k=2}d_k\) converges absolutely.

    \item Imitate the strategy in the proof of Theorem 2.8.1 to show that \(\sum^\infty_{k=2}d_k\) converges to \(S = \lim_{n\to\infty}s_{nn}\).
}
\end{exercise}

\begin{solution}
\enum{
    \item Note that \(\sum^n_{i=1}\sum^n_{j=1}a_{ij}\) contains all of the terms of \(\sum^\infty_{k=2}d_k\), and thus by comparison to \(\sum^n_{i=1}\sum^n_{j=1}|a_{ij}|\), \(\sum^\infty_{k=2}|d_k|\) must converge.

    \item What we need to show is that for all \(\epsilon>0\) there exists \(N\) such that for all \(n > N\), \(|S - \sum^n_{k=2}d_k| < \epsilon\). Note first that \( \sum^n_{k=2}d_k \) contains all the elements of \(s_{pp}\) when \(p \leq n / 2\), and that \(s_{qq}\) contains all the elements of \(\sum^n_{k=2}d_k\) as long as \(q \geq n-1\).

    Since \((s_{nn}) \to S\), for arbitrary \(\epsilon_1 > 0\) we can choose \(n_1\) large enough such that \(|s_{n_1 n_1} - S| < \epsilon_1\). If we choose \(N = 2n\) then whenever \(n > N\), \(\sum^n_{k=2}d_k\) will contain all terms in \(s_{n_1}s_{n_1}\), and if we choose \(m = n - 1\) then \(s_{mm}\) will contain all terms in \(\sum^n_{k=2}d_k\). Thus
    \[ \sum^n_{k=2}|d_k| - t_{n_1n_1} \leq t_{mm} - t_{n_1n_1} \]
where \(t_{nn}\) was defined near the start of the proof of Theorem 2.8.1.
Moreover since \(t_{nn}\) converges (as proved in Exercise 2.8.3a) and is thus a Cauchy sequence, for arbitrary \(\epsilon_2 > 0\), we can also choose \(n_1\) large enough to ensure for any \(m_1 > n_1\), \(t_{m_1 m_1} - t_{n_1 n_1} < \epsilon_2\).

Putting it all together, choosing \(\epsilon_1 = \epsilon_2 = \epsilon/2\), \(n_1\) large enough to satisfy the two conditions discussed above, and \(N = 2n_1\):
\[ \begin{aligned}
\abs{S - \sum^n_{k=2}d_k} &\leq |S - s_{n_1n_1}| + \abs{\sum^n_{k=2}d_k - s_{n_1n_1}} \\
&< \epsilon_1 + \abs{\sum^n_{k=2}|d_k| - t_{n_1n_1}} \\
&\leq \epsilon_1 + t_{mm} - t_{n_1n_1} < \epsilon_1 + \epsilon_2 \\
&= \epsilon
\end{aligned}
\]
completing the proof.
}
\end{solution}

\begin{exercise}
    Assume that \(\sum^\infty_{i=1}a_i\) converges absolutely to \(A\), and \(\sum^\infty_{j=1}b_j\) converges absolutely to \(B\).

\enum{
    \item Show that the iterated sum \(\sum^\infty_{i=1}\sum^\infty_{j=1}|a_i b_j|\) converges so that we may apply Theorem 2.8.1.
    \item Let \(s_{nn} = \sum^n_{i=1}\sum^n_{j=1}a_i b_j\), and prove that \(\lim_{n \to \infty} s_{nn} = AB\). Conclude that
\[
    \sum^\infty_{i=1}\sum^\infty_{j=1}a_i b_j = \sum^\infty_{j=1}\sum^\infty_{i=1}a_ib_j = \sum^\infty_{k=2}d_k = AB,
\]
where, as before, \(d_k = a_1 b_{k-1} + a_2 b_{k-2} + \cdots + a_{k-1} b_1\).
}
\end{exercise}

\begin{solution}
    \TODO
\end{solution}
