\section{The Cauchy Criterion}

\begin{exercise}
  Prove every convergent sequence is a Cauchy sequence. (Theorem 2.6.2)
\end{exercise}

\begin{solution}
  \TODO
\end{solution}

\begin{exercise}
  Give an example of each of the following, or argue that such a request is impossible.
  \enum{
  \item A Cauchy sequence that is not monotone.
  \item A Cauchy sequence with an unbounded subsequence.
  \item A divergent monotone sequence with a Cauchy subsequence.
  \item An unbounded sequence containing a subsequence that is Cauchy.
  }
\end{exercise}

\begin{solution}
  \enum{
  \item \TODO
  \item \TODO
  \item \TODO
  \item \TODO
  }
\end{solution}

\begin{exercise}
  If $\left(x_{n}\right)$ and $\left(y_{n}\right)$ are Cauchy sequences, then one easy way to prove that $\left(x_{n}+y_{n}\right)$ is Cauchy is to use the Cauchy Criterion. By Theorem 2.6.4, $\left(x_{n}\right)$ and $\left(y_{n}\right)$ must be convergent, and the Algebraic Limit Theorem then implies $\left(x_{n}+y_{n}\right)$ is convergent and hence Cauchy.
  \enum{
  \item Give a direct argument that $\left(x_{n}+y_{n}\right)$ is a Cauchy sequence that does not use the Cauchy Criterion or the Algebraic Limit Theorem.
  \item Do the same for the product $\left(x_{n} y_{n}\right)$.
  }
\end{exercise}

\begin{solution}
  \enum{
  \item \TODO
  \item \TODO
  }
\end{solution}


\begin{exercise}
  Let $\left(a_{n}\right)$ and $\left(b_{n}\right)$ be Cauchy sequences. Decide whether each of the following sequences is a Cauchy sequence, justifying each conclusion.
  \enum{
  \item $c_{n}=\left|a_{n}-b_{n}\right|$
  \item $c_{n}=(-1)^{n} a_{n}$
  \item $c_{n}=\left[\left[a_{n}\right]\right]$, where $[[x]]$ refers to the greatest integer less than or equal to $x$.
  }
\end{exercise}

\begin{solution}
  \enum{
  \item \TODO
  \item \TODO
  \item \TODO
  }
\end{solution}

\begin{exercise}
  Consider the following (invented) definition: A sequence $\left(s_{n}\right)$ is pseudo-Cauchy if, for all $\epsilon>0$, there exists an $N$ such that if $n \geq N$, then $\left|s_{n+1}-s_{n}\right|<\epsilon$

  Decide which one of the following two propositions is actually true. Supply a proof for the valid statement and a counterexample for the other.
  \enumr{
  \item Pseudo-Cauchy sequences are bounded.
  \item  If $\left(x_{n}\right)$ and $\left(y_{n}\right)$ are pseudo-Cauchy, then $\left(x_{n}+y_{n}\right)$ is pseudo-Cauchy as well.
  }
\end{exercise}

\begin{solution}
  \TODO
\end{solution}

\begin{exercise}
  Let's call a sequence $\left(a_{n}\right)$ quasi-increasing if for all $\epsilon>0$ there exists an $N$ such that whenever $n>m \geq N$ it follows that $a_{n}>a_{m}-\epsilon$
  \enum{
  \item Give an example of a sequence that is quasi-increasing but not monotone or eventually monotone.
  \item Give an example of a quasi-increasing sequence that is divergent and not monotone or eventually monotone.
  \item Is there an analogue of the Monotone Convergence Theorem for quasiincreasing sequences? Give an example of a bounded, quasi-increasing sequence that doesn't converge, or prove that no such sequence exists.
  }
\end{exercise}

\begin{solution}
  \enum{
  \item \TODO
  \item \TODO
  \item \TODO
  }
\end{solution}

\begin{exercise}
  Exercises 2.4.4 and 2.5.4 establish the equivalence of the Axiom of Completeness and the Monotone Convergence Theorem. They also show the Nested Interval Property is equivalent to these other two in the presence of the Archimedean Property.
  \enum{
  \item Assume the Bolzano-Weierstrass Theorem is true and use it to construct a proof of the Monotone Convergence Theorem without making any appeal to the Archimedean Property. This shows that BW, AoC, and MCT are all equivalent.
  \item Use the Cauchy Criterion to prove the Bolzano-Weierstrass Theorem, and find the point in the argument where the Archimedean Property is implicitly required. This establishes the final link in the equivalence of the five characterizations of completeness discussed at the end of Section $2.6$.
  \item How do we know it is impossible to prove the Axiom of Completeness starting from the Archimedean Property?
  }
\end{exercise}

\begin{solution}
  \enum{
  \item \TODO
  \item \TODO
  \item \TODO
  }
\end{solution}
