\section{The Cauchy Criterion}

\begin{exercise}
  Prove every convergent sequence is a Cauchy sequence. (Theorem 2.6.2)
\end{exercise}

\begin{solution}
  Suppose $(x_n)$ is convergent, we must show that for $m,n > N$ we have $|x_n - x_m| < \epsilon$

  Set $|x_n - x| < \epsilon/2$ for $n > N$.

  We get $|x_n - x_m| \le |x_n - x| + |x - x_m| \le \epsilon/2 + \epsilon/2 = \epsilon$
\end{solution}

\begin{exercise}
  Give an example of each of the following, or argue that such a request is impossible.
  \enum{
  \item A Cauchy sequence that is not monotone.
  \item A Cauchy sequence with an unbounded subsequence.
  \item A divergent monotone sequence with a Cauchy subsequence.
  \item An unbounded sequence containing a subsequence that is Cauchy.
  }
\end{exercise}

\begin{solution}
  \enum{
  \item $x_n = (-1)^n/n$ is cauchy by Theorem 2.6.2.
  \item Impossible since all cauchy sequences converge.
  \item Impossible, If a subsequence was cauchy it would converge, implying the subsequence would be bounded and therefore the parent sequence would be bounded (because it is monotone) and thus would converge.
  \item $(2, 1/2, 3, 1/3, \dots)$ has subsequence $(1/2, 1/3, \dots)$ which is cauchy.
  }
\end{solution}

\begin{exercise}
  If $\left(x_{n}\right)$ and $\left(y_{n}\right)$ are Cauchy sequences, then one easy way to prove that $\left(x_{n}+y_{n}\right)$ is Cauchy is to use the Cauchy Criterion. By Theorem 2.6.4, $\left(x_{n}\right)$ and $\left(y_{n}\right)$ must be convergent, and the Algebraic Limit Theorem then implies $\left(x_{n}+y_{n}\right)$ is convergent and hence Cauchy.
  \enum{
  \item Give a direct argument that $\left(x_{n}+y_{n}\right)$ is a Cauchy sequence that does not use the Cauchy Criterion or the Algebraic Limit Theorem.
  \item Do the same for the product $\left(x_{n} y_{n}\right)$.
  }
\end{exercise}

\begin{solution}
  \enum{
  \item We have $|(x_n + y_n) - (x_m + y_m)| \le |x_n - x_m| + |y_n - y_m| < \epsilon/2 + \epsilon/2 = \epsilon$
  \item Bound $|x_n| \le M_1$, and $|y_n| \le M_2$ then
    $$
    \begin{aligned}
      |x_n y_n - x_m y_m|
      &= |(x_ny_n - x_ny_m) + (x_ny_m - x_my_m)| \\
      &\le |x_n(y_n - y_m)| + |y_m(x_n - x_m)| \\
      &\le M_1|y_n - y_m| + M_2|x_n - x_m| \\
      &< \epsilon/2 + \epsilon/2 = \epsilon
    \end{aligned}
    $$
    After setting $|y_n - y_m| < \epsilon/(2M_1)$ and $|x_n-x_m|<\epsilon/(2M_2)$.
  }
\end{solution}


\begin{exercise}
  Let $\left(a_{n}\right)$ and $\left(b_{n}\right)$ be Cauchy sequences. Decide whether each of the following sequences is a Cauchy sequence, justifying each conclusion.
  \enum{
  \item $c_{n}=\left|a_{n}-b_{n}\right|$
  \item $c_{n}=(-1)^{n} a_{n}$
  \item $c_{n}=\left[\left[a_{n}\right]\right]$, where $[[x]]$ refers to the greatest integer less than or equal to $x$.
  }
\end{exercise}

\begin{solution}
  \enum{
  \item Yes, since $|(a_n - b_n) - (a_m - b_m)| \le |a_n - a_m| + |b_m - b_n| < \epsilon/2 + \epsilon/2 = \epsilon$
  \item No, if $a_n = 1$ then $(-1)^na_n$ diverges, and thus is not cauchy.
  \item No, if $a_n = 1 - (-1)^n/n$ then $[[a_n]]$ fluctuates between $0$ and $1$ and so cannot be cauchy.
  }
\end{solution}

\begin{exercise}
  Consider the following (invented) definition: A sequence $\left(s_{n}\right)$ is pseudo-Cauchy if, for all $\epsilon>0$, there exists an $N$ such that if $n \geq N$, then $\left|s_{n+1}-s_{n}\right|<\epsilon$

  Decide which one of the following two propositions is actually true. Supply a proof for the valid statement and a counterexample for the other.
  \enumr{
  \item Pseudo-Cauchy sequences are bounded.
  \item  If $\left(x_{n}\right)$ and $\left(y_{n}\right)$ are pseudo-Cauchy, then $\left(x_{n}+y_{n}\right)$ is pseudo-Cauchy as well.
  }
\end{exercise}

\begin{solution}
  \enumr{
  \item False, consider $s_n = \log n$. clearly $|s_{n+1} - s_n|$ can be made arbitrarily small but $s_n$ is unbounded.
  \item True, as $|(x_{n+1} + y_{n+1}) - (x_n + y_n)| \le |x_{n+1} - x_n| + |y_{n+1} - y_n| < \epsilon/2 + \epsilon/2 = \epsilon$.
  }
\end{solution}

\begin{exercise}
  Let's call a sequence $\left(a_{n}\right)$ quasi-increasing if for all $\epsilon>0$ there exists an $N$ such that whenever $n>m \geq N$ it follows that $a_{n}>a_{m}-\epsilon$
  \enum{
  \item Give an example of a sequence that is quasi-increasing but not monotone or eventually monotone.
  \item Give an example of a quasi-increasing sequence that is divergent and not monotone or eventually monotone.
  \item Is there an analogue of the Monotone Convergence Theorem for quasiincreasing sequences? Give an example of a bounded, quasi-increasing sequence that doesn't converge, or prove that no such sequence exists.
  }
\end{exercise}

\begin{solution}
  \enum{
  \item $a_n = (-1)^n/n$  is quasi-increasing since we can get $(-1)^m/m - (-1)^n/n \le 1/m + 1/n < \epsilon$ for large enough $n > m \ge N$.
  \item $a_n = (2, 1/2, 3, 1/3, \dots)$ is quasi-increasing since $n > m - \epsilon$ is clearly true as $(n)$ is increasing. And $1/n > 1/m - \epsilon$ is true after picking $N$ large enough that for $m \ge N$ we have $1/m < \epsilon$ and thus $1/n > 1/m - \epsilon$.
  \item In (b) I gave such an example, so there is no Monotone Convergence Theorem for quasiincreasing sequences (without modifying the definition that is.)
  }
\end{solution}

\begin{exercise}
  Exercises 2.4.4 and 2.5.4 establish the equivalence of the Axiom of Completeness and the Monotone Convergence Theorem. They also show the Nested Interval Property is equivalent to these other two in the presence of the Archimedean Property.
  \enum{
  \item Assume the Bolzano-Weierstrass Theorem is true and use it to construct a proof of the Monotone Convergence Theorem without making any appeal to the Archimedean Property. This shows that BW, AoC, and MCT are all equivalent.
  \item Use the Cauchy Criterion to prove the Bolzano-Weierstrass Theorem, and find the point in the argument where the Archimedean Property is implicitly required. This establishes the final link in the equivalence of the five characterizations of completeness discussed at the end of Section $2.6$.
  \item How do we know it is impossible to prove the Axiom of Completeness starting from the Archimedean Property?
  }
\end{exercise}

\begin{solution}
  \enum{
  \item Suppose $(x_n)$ is increasing and bounded, BW tells us there exists a convergent subsequence $(x_{n_k}) \to x$, We will show $(x_n) \to x$. First note $x_k \le x_{n_k}$ implies $x_n \le x$ by the Order Limit Theorem.

    Pick $K$ such that for $k \ge K$ we have $|x_{n_k} - x| < \epsilon$.
    Since $(x_n)$ is increasing and $x_n \le x$ every $n \ge n_K$ satisfies $|x_n - x| < \epsilon$ as well.
    Thus $(x_n)$ converges, completing the proof.
  \item We're basically going to use the cauchy criterion as a replacement for NIP in the proof of BW.
    Recall we had $I_{n+1} \subseteq I_n$ with $a_{n_k} \in I_k$, we will show $a_{n_k}$ is cauchy.

    The length of $I_k$ is $M(1/2)^{k-1}$ by construction, so clearly $|a_{n_k} - a_{n_j}| < M(1/2)^{N-1}$ for $k,j \ge N$, implying $(a_{n_k})$ converges by the cauchy criterion.

    We needed the Archimedean Property to conclude $M(1/2)^{N-1} \in \mathbf{Q}$ can be made smaller then any $\epsilon \in \mathbf{R}^+$.
  \item The Archimedean Property is true for $\mathbf{Q}$ meaning it cannot prove AoC which is only true for $\mathbf{R}$. (If we did, then we would have proved AoC for $\mathbf{Q}$ which is obviously false.)
  }
\end{solution}
