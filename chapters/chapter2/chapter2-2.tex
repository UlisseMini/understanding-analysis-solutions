\section{The Limit of a Sequence}

\begin{exercise}
What happens if we reverse the order of the quantifiers in Definition 2.2.3?

Definition: A sequence $\left(x_{n}\right)$ \emph{verconges} to $x$ if \emph{there exists} an $\epsilon>0$ such that \emph{for all} $N \in \mathbf{N}$ it is true that $n \geq N$ implies $\left|x_{n}-x\right|<\epsilon$

Give an example of a vercongent sequence. Is there an example of a vercongent sequence that is divergent? Can a sequence verconge to two different values? What exactly is being described in this strange definition?
\end{exercise}

\begin{solution}
  Firstly, since we have \emph{for all} $N \in \mathbf{N}$ we can remove $N$ entirely and just say $n \in \mathbf{N}$. Our new definition is

  Definition: A sequence $(x_n)$ \emph{verconges} to $x$ if \emph{there exists} an $\epsilon > 0$ such that $\emph{for all}$ $n \in \mathbf{N}$ we have $|x_n - x| < \epsilon$.

  In other words, a series $(x_n)$ \emph{verconges} to $x$ if $|x_n - x|$ is bounded. This is a silly definition though since if $|x_n - x|$ is bounded, then $|x_n - x'|$ is bounded for all $x' \in \mathbf{R}$, meaning if a sequence is vercongent it verconges to every $x' \in \mathbf{R}$.

  Put another way, a sequence is vercongent \emph{if and only if} it is bounded.
\end{solution}

\begin{exercise}
  Verify, using the definition of convergence of a sequence, that the following sequences converge to the proposed limit.
  \enum{
  \item $\lim \frac{2 n+1}{5 n+4}=\frac{2}{5}$.
  \item $\lim \frac{2 n^{2}}{n^{3}+3}=0$.
  \item $\lim \frac{\sin \left(n^{2}\right)}{\sqrt[3]{n}}=0$.
  }
\end{exercise}

\begin{solution}
  \enum{
  \item We have
    $$
    \left|\frac{2n + 1}{5n + 4} - \frac{2}{5}\right|
    = \left|\frac{5(2n + 1) - 2(5n + 4)}{5(5n + 4)}\right|
    = \left|\frac{-3}{5(5n + 4)}\right|
    = \frac{3}{5(5n + 4)} < \epsilon
    $$
    We now find $n$ such that the distance is less then $\epsilon$
    $$
    \frac{3}{5(5n + 4)} < \frac{1}{n} < \epsilon \implies n > \frac{1}{\epsilon}
    $$
    You could also solve for the smallest $n$, which would give you
    $$
    \frac{3}{5(5n + 4)} < \epsilon \implies 5n + 4 > \frac{3}{5\epsilon} \implies n > \frac{3}{25\epsilon} - \frac{4}{5}
    $$
    I prefer the first approach, the second is better if you were doing numerical analysis and wanted a precise convergence rate.
  \item We have
    $$
    \left|\frac{2n^2}{n^3 + 3} - 0\right| = \frac{2n^2}{n^3 + 3} < \frac{2n^2}{n^3} = \frac{2}{n} < \epsilon \implies n > \frac{2}{\epsilon}
    $$
  \item We have
    $$
    \frac{\sin(n^2)}{n^{1/3}} \le \frac{1}{n^{1/3}} < \epsilon \implies n > \frac{1}{\epsilon^3}
    $$
    Really slow convergence! if $\epsilon = 10^{-2}$ we would require $n > 10^6$
  }
\end{solution}

\begin{exercise}
  Describe what we would have to demonstrate in order to disprove each of the following statements.
  \enum{
  \item At every college in the United States, there is a student who is at least seven feet tall.
  \item For all colleges in the United States, there exists a professor who gives every student a grade of either $\mathrm{A}$ or $\mathrm{B}$.
  \item There exists a college in the United States where every student is at least six feet tall.
  }
\end{exercise}

\begin{solution}
  \enum{
  \item Find a collage in the United States with no students over seven feet tall.
  \item Find a collage in the United States with a professor who has given a grade other then an A or B.
  \item Find a collage in the united States with at least one student under six feet tall.
  }
\end{solution}

\begin{exercise}
  Give an example of each or state that the request is impossible. For any that are impossible, give a compelling argument for why that is the case.
  \enum{
  \item A sequence with an infinite number of ones that does not converge to one.
  \item A sequence with an infinite number of ones that converges to a limit not equal to one.
  \item A divergent sequence such that for every $n \in \mathbf{N}$ it is possible to find $n$ consecutive ones somewhere in the sequence.
  }
\end{exercise}

\begin{solution}
  \enum{
  \item $a_n = (-1)^n$
  \item Impossible, if $\lim a_n = a \ne 1$ then for any $n \ge N$ we can find a $n$ with $a_n = 1$ meaning $\epsilon < |1 - a|$ is impossible.
  \item $a_n = (1, 2, 1, 1, 3, 1, 1, 1, \dots)$
  }
\end{solution}

\begin{exercise}
  Let $[[x]]$ be the greatest integer less than or equal to $x$. For example, $[[\pi]]=3$ and $[[3]]=3$. For each sequence, find $\lim a_{n}$ and verify it with the definition of convergence.
  \enum{
  \item $a_{n}=[[5 / n]]$,
  \item $a_{n}=[[(12+4 n) / 3 n]]$.
  }
  Reflecting on these examples, comment on the statement following Definition 2.2.3 that ``the smaller the $\epsilon$-neighborhood, the larger $N$ may have to be.''
\end{exercise}

\begin{solution}
  \enum{
  \item For all $n > 5$ we have $[[5/n]] = 0$ meaning $\lim a_n = 0$.
  \item The inside clearly converges to $4/3$ from above, so $\lim a_n = 1$.
  }
  Some sequences eventually reach their limit, meaning $N$ no longer has to increase.
\end{solution}

\begin{exercise}
  \textbf{Theorem 2.2.7 (Uniqueness of Limits).} \textit{The limit of a sequence, when it exists, must be unique.}

  Prove Theorem 2.2.7. To get started, assume $\left(a_{n}\right) \rightarrow a$ and also that $\left(a_{n}\right) \rightarrow b$. Now argue $a=b$
\end{exercise}

\begin{solution}
  If $a \ne b$ then we can set $\epsilon$ small enough that having both $|a_n - a| < \epsilon$ and $|a_n - b| < \epsilon$ is impossible. Therefor $a = b$.

  (Making this rigorous is trivial and left as an exercise to the reader)
\end{solution}

\begin{exercise}
  Here are two useful definitions:
  \enumr{
  \item A sequence $\left(a_{n}\right)$ is \emph{eventually} in a set $A \subseteq \mathbf{R}$ if there exists an $N \in \mathbf{N}$ such that $a_{n} \in A$ for all $n \geq N$.
  \item A sequence $\left(a_{n}\right)$ is \emph{frequently} in a set $A \subseteq \mathbf{R}$ if, for every $N \in \mathbf{N}$, there exists an $n \geq N$ such that $a_{n} \in A$.
    \enum{
    \item Is the sequence $(-1)^{n}$ eventually or frequently in the set $\{1\}$?
    \item Which definition is stronger? Does frequently imply eventually or does eventually imply frequently?
    \item Give an alternate rephrasing of Definition 2.2.3B using either frequently or eventually. Which is the term we want?
    \item Suppose an infinite number of terms of a sequence $\left(x_{n}\right)$ are equal to 2 . Is $\left(x_{n}\right)$ necessarily eventually in the interval $(1.9,2.1) ?$ Is it frequently in $(1.9,2.1) ?$
    }
  }
\end{exercise}

\begin{solution}
  \enum{
  \item Frequently, but not eventually.
  \item Eventually is stronger, it implies frequently.
  \item $(x_n) \to x$ \emph{if and only if} $x_n$ is eventually in any $\epsilon$-neighborhood around $x$.
  \item $(x_n)$ is frequently in $(1.9, 2.1)$ but not necessarily eventually (consider $x_n = 2(-1)^n$).
  }
\end{solution}

\begin{exercise}
  For some additional practice with nested quantifiers, consider the following invented definition:

  Let's call a sequence $\left(x_{n}\right)$ zero-heavy if there exists $M \in \mathbf{N}$ such that for all $N \in \mathbf{N}$ there exists $n$ satisfying $N \leq n \leq N+M$ where $x_{n}=0$
  \enum{
  \item Is the sequence $(0,1,0,1,0,1, \ldots)$ zero heavy?
  \item If a sequence is zero-heavy does it necessarily contain an infinite number of zeros? If not, provide a counterexample.
  \item If a sequence contains an infinite number of zeros, is it necessarily zeroheavy? If not, provide a counterexample.
  \item Form the logical negation of the above definition. That is, complete the sentence: A sequence is not zero-heavy if ....
  }
\end{exercise}

\begin{solution}
  \enum{
  \item No.
  \item Yes. as any finite number of zeros $K$ would lead to a contradiction when $M > K$.
  \item No, consider $(0,1,0,\dots)$ from (a).
  \item A sequence is not zero-heavy if there exists an $M \in \mathbf{N}$ such that for all $N \in \mathbf{N}$ there exists an $n \in \mathbf{N}$ such that $N \le n \le N + M$ but $x_n \ne 0$.
  }
\end{solution}

