\chapter{The Real Numbers}


\setcounter{section}{1} % skip the first section, it has no exercises
\section{Some Preliminaries}

\begin{exercise}
  \enum{
  \item Prove that $\sqrt 3$ is irrational. Does a similar similar argument work to show $\sqrt 6$ is irrational?
  \item Where does the proof break down if we try to prove $\sqrt 4$ is irrational?
  }
\end{exercise}

\begin{solution}
  \enum{
  \item
    Suppose for contradiction that $p/q$ is a fraction in lowest terms, and that $(p/q)^2 = 3$.
    Then $p^2 = 3q^2$ implying $p$ is a multiple of $3$ since $3$ is not a perfect square. Therefor we can write $p$ as $3r$ for some $r$, substituting we get $(3r)^2 = 3q^2$ and $3r^2 = q^2$ implying $q$ is also a multiple of $3$ contradicting the assumption that $p/q$ is in lowest terms. \\
    For $\sqrt 6$ the same argument applies, since $6$ is not a perfect square.

  \item $4$ is a perfect square, meaning $p^2 = 4q^2$ does not imply that $p$ is a multiple of four as $p$ could be $2$.
  }
\end{solution}

