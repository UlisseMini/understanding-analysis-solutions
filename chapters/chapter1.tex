\chapter{The Real Numbers}


\setcounter{section}{1} % skip the first section, it has no exercises
\section{Some Preliminaries}

\begin{exercise}
  \enum{
  \item Prove that $\sqrt 3$ is irrational. Does a similar similar argument work to show $\sqrt 6$ is irrational?
  \item Where does the proof break down if we try to prove $\sqrt 4$ is irrational?
  }
\end{exercise}

\begin{solution}
  \enum{
  \item
    Suppose for contradiction that $p/q$ is a fraction in lowest terms, and that $(p/q)^2 = 3$.
    Then $p^2 = 3q^2$ implying $p$ is a multiple of $3$ since $3$ is not a perfect square. Therefor we can write $p$ as $3r$ for some $r$, substituting we get $(3r)^2 = 3q^2$ and $3r^2 = q^2$ implying $q$ is also a multiple of $3$ contradicting the assumption that $p/q$ is in lowest terms. \\
    For $\sqrt 6$ the same argument applies, since $6$ is not a perfect square.

  \item $4$ is a perfect square, meaning $p^2 = 4q^2$ does not imply that $p$ is a multiple of four as $p$ could be $2$.
  }
\end{solution}

\begin{exercise}
  Show that there is no rational number satisfying $2^r = 3$
\end{exercise}

\begin{solution}
  Letting $r = p/q$ we have $2^{p/q} = 3$ implying $2^p = 3^q$ which is impossible since $2$ and $3$ are coprime.
\end{solution}


\begin{exercise}
  Decide which of the following represent true statements about the nature of sets. For any that are false, provide a specific example where the statement in question does not hold.
  \enum{
  \item If $A_{1} \supseteq A_{2} \supseteq A_{3} \supseteq A_{4} \cdots$ are all sets containing an infinite number of elements, then the intersection $\bigcap_{n=1}^{\infty} A_{n}$ is infinite as well.
  \item If $A_{1} \supseteq A_{2} \supseteq A_{3} \supseteq A_{4} \cdots$ are all finite, nonempty sets of real numbers, then the intersection $\bigcap_{n=1}^{\infty} A_{n}$ is finite and nonempty.
  \item $A \cap(B \cup C)=(A \cap B) \cup C$.
  \item $A \cap(B \cap C)=(A \cap B) \cap C$.
  \item $A \cap(B \cup C)=(A \cap B) \cup(A \cap C)$.
  }
\end{exercise}

\begin{solution}
  \enum{
  \item False, consider $A_1 = \{1, 2, \dots\}, A_2 = \{2, 3, \dots\}$, ... has $\bigcap_{n=1}^\infty A_n = \emptyset$.
  \item True.
  \item False, $A = \emptyset$ gives $\emptyset = C$.
  \item True, intersection is associative.
  \item True, draw a diagram.
  }
\end{solution}
