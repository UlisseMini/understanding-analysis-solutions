\chapter{The Real Numbers}


\setcounter{section}{1} % skip the first section, it has no exercises
\section{Some Preliminaries}

\begin{exercise}
  \enum{
  \item Prove that $\sqrt 3$ is irrational. Does a similar similar argument work to show $\sqrt 6$ is irrational?
  \item Where does the proof break down if we try to prove $\sqrt 4$ is irrational?
  }
\end{exercise}

\begin{solution}
  \enum{
  \item
    Suppose for contradiction that $p/q$ is a fraction in lowest terms, and that $(p/q)^2 = 3$.
    Then $p^2 = 3q^2$ implying $p$ is a multiple of $3$ since $3$ is not a perfect square. Therefor we can write $p$ as $3r$ for some $r$, substituting we get $(3r)^2 = 3q^2$ and $3r^2 = q^2$ implying $q$ is also a multiple of $3$ contradicting the assumption that $p/q$ is in lowest terms. \\
    For $\sqrt 6$ the same argument applies, since $6$ is not a perfect square.

  \item $4$ is a perfect square, meaning $p^2 = 4q^2$ does not imply that $p$ is a multiple of four as $p$ could be $2$.
  }
\end{solution}

\begin{exercise}
  Show that there is no rational number satisfying $2^r = 3$
\end{exercise}

\begin{solution}
  Letting $r = p/q$ we have $2^{p/q} = 3$ implying $2^p = 3^q$ which is impossible since $2$ and $3$ are coprime.
\end{solution}


\begin{exercise}
  Decide which of the following represent true statements about the nature of sets. For any that are false, provide a specific example where the statement in question does not hold.
  \enum{
  \item If $A_{1} \supseteq A_{2} \supseteq A_{3} \supseteq A_{4} \cdots$ are all sets containing an infinite number of elements, then the intersection $\bigcap_{n=1}^{\infty} A_{n}$ is infinite as well.
  \item If $A_{1} \supseteq A_{2} \supseteq A_{3} \supseteq A_{4} \cdots$ are all finite, nonempty sets of real numbers, then the intersection $\bigcap_{n=1}^{\infty} A_{n}$ is finite and nonempty.
  \item $A \cap(B \cup C)=(A \cap B) \cup C$.
  \item $A \cap(B \cap C)=(A \cap B) \cap C$.
  \item $A \cap(B \cup C)=(A \cap B) \cup(A \cap C)$.
  }
\end{exercise}

\begin{solution}
  \enum{
  \item False, consider $A_1 = \{1, 2, \dots\}, A_2 = \{2, 3, \dots\}$, ... has $\bigcap_{n=1}^\infty A_n = \emptyset$.
  \item True.
  \item False, $A = \emptyset$ gives $\emptyset = C$.
  \item True, intersection is associative.
  \item True, draw a diagram.
  }
\end{solution}

\begin{exercise}
  Produce an infinite collection of sets $A_{1}, A_{2}, A_{3}, \ldots$ with the property that every $A_{i}$ has an infinite number of elements, $A_{i} \cap A_{j}=\emptyset$ for all $i \neq j$, and $\bigcup_{i=1}^{\infty} A_{i}=\mathbf{N}$
\end{exercise}

\begin{solution}
  This question is asking us to partition $\mathbf N$ into an infinite collection of sets.
  This is equivalent to asking us to unroll $\mathbf N$ into a square, which we can do along the diagonal
  $$
  \begin{array}{lccccc}
    1 & 3 & 6 & 10 & 15 & \cdots \\
    2 & 5 & 9 & 14 & \cdots & \\
    4 & 8 & 13 & \cdots & & \\
    7 & 12 & \cdots & & & \\
    11 & \ldots & & & & \\
    \vdots & & & & &
  \end{array}
  $$
\end{solution}

\begin{exercise}[De Morgan's Laws]
  \label{demorgan}
  Let $A$ and $B$ be subsets of $\mathbf R$.
  \enum{
  \item If $x \in(A \cap B)^{c}$, explain why $x \in A^{c} \cup B^{c}$. This shows that $(A \cap B)^{c} \subseteq$ $A^{c} \cup B^{c}$

  \item Prove the reverse inclusion $(A \cap B)^{c} \supseteq A^{c} \cup B^{c}$, and conclude that $(A \cap B)^{c}=A^{c} \cup B^{c}$
  \item Show $(A \cup B)^{c}=A^{c} \cap B^{c}$ by demonstrating inclusion both ways.
  }
\end{exercise}

\begin{solution}
  \enum{
  \item \TODO
  \item \TODO
  \item \TODO
  }
\end{solution}


\begin{exercise}
  \enum{
  \item Verify the triangle inequality in the special case where $a$ and $b$ have the same sign.
  \item Find an efficient proof for all the cases at once by first demonstrating $(a+b)^{2} \leq(|a|+|b|)^{2}$
  \item Prove $|a-b| \leq|a-c|+|c-d|+|d-b|$ for all $a, b, c$, and $d$.
  \item Prove $\| a|-| b|| \leq|a-b|$. (The unremarkable identity $a=a-b+b$ may be useful.)
  }
\end{exercise}

\begin{solution}
  \enum{
  \item We have equality $|a + b| = |a| + |b|$ meaning $|a + b| \le |a| + |b|$ also holds.
  \item $(a+b)^2 \le (|a| + |b|)^2$ reduces to $2ab \le 2|a||b|$ which is obviously true.
    and since squaring preserves inequality this implies $|a + b| \le |a| + |b|$.
  \item I would like to do this using the triangle inequality, I notice that $(a-c)+(c-d)+(d-b) = a-b$. Meaning I can use the triangle inequality for multiple terms
    $$|a - b| = |(a-c)+(c-d)+(d-b)| \le |a-c| + |c-d| + |d-b|$$
    The general triangle inequality is proved by repeated application of the two variable inequality
    $$|(a + b) + c| \le |a + b| + |c| \le |a| + |b| + |c|$$
  \item I would like to cancel the subtraction inside $||a| - |b||$ since then the inside will be positive, and the outer absolute value will vanish. Using the suggestion let $a = (a - b) + b$
    $$||a| - |b|| = ||(a-b)+b| - |b|| \stackrel{!}{\le} ||a-b| + |b| - |b|| = |a - b|$$
    However this is incorrect by itself, since $|a| \le |c|$ does not imply $||a|-|b|| \le ||c|-|b||$ (draw a picture, or use the counterexample $a=0, c=1, b=2$). \par

    We can salvage this argument though, notice if $|a| \ge |b|$ then $|a| \le |c|$ does imply $||a|-|b|| \le ||c| - |b||$. And since we can swap $a$ and $b$ without changing anything, we can say without loss of generality assume $|a| \ge |b|$ and then apply the previous argument.
  }
\end{solution}

\begin{exercise}
  Given a function $f$ and a subset $A$ of its domain, let $f(A)$ represent the range of $f$ over the set $A$; that is, $f(A)=\{f(x): x \in A\}$.
  \enum{
  \item Let $f(x)=x^{2} .$ If $A=[0,2]$ (the closed interval $\left.\{x \in \mathbf{R}: 0 \leq x \leq 2\}\right)$ and $B=[1,4]$, find $f(A)$ and $f(B) .$ Does $f(A \cap B)=f(A) \cap f(B)$ in this case? Does $f(A \cup B)=f(A) \cup f(B) ?$
  \item Find two sets $A$ and $B$ for which $f(A \cap B) \neq f(A) \cap f(B)$.
  \item  Show that, for an arbitrary function $g: \mathbf{R} \rightarrow \mathbf{R}$, it is always true that $g(A \cap B) \subseteq g(A) \cap g(B)$ for all sets $A, B \subseteq \mathbf{R}$
  \item Form and prove a conjecture about the relationship between $g(A \cup B)$ and $g(A) \cup g(B)$ for an arbitrary function $g$
  }
\end{exercise}

\begin{solution}
  \enum{
  \item $f(A)=[0, 4]$, $f(B)=[1,16]$, $f(A\cap B) = [1,4] = f(A)\cap f(B)$ and $f(A\cup B) = [0, 16] = f(A) \cup f(B)$
  \item $A = \{-1\}$, $B = \{1\}$ thus $f(A \cap B) = \emptyset$ but $f(A) \cap f(B) = \{1\}$
  \item Suppose $y \in g(A \cap B)$, then $\exists x \in A \cap B$ such that $g(x) = y$. But if $x \in A \cap B$ then $x \in A$ and $x \in B$, meaning $y \in g(A)$ and $y \in g(B)$ implying $y \in g(A) \cap g(B)$ and thus $g(A \cap B) \subseteq g(A) \cap g(B)$. \par
    Notice why it is possible to have $x \in g(A) \cap g(B)$ but $x \notin g(A \cap B)$, this happens when something in $A \setminus B$ and something in $B \setminus A$ maps to the same thing. If $g$ is 1-1 this does not happen. \TODO draw a picture with tikz and prove in appendix
  \item I conjecture that $g(A \cup B) = g(A) \cup g(B)$. To prove this we show inclusion both ways,
    First suppose $y \in g(A\cup B)$. then either $y \in g(A)$ or $y \in g(B)$, implying $y \in g(A) \cup g(B)$. Now suppose $y \in g(A) \cup g(B)$ meaning either $y \in g(A)$ or $y \in g(B)$ which is the same as $y \in g(A \cup B)$ as above.
  }
\end{solution}

\begin{theorem}
  $g(A \cap B) = g(A) \cap g(B)$ if and only if $g$ is one-to-one
\end{theorem}

\begin{exercise}
  Here are two important definitions related to a function $f:$ $A \rightarrow B .$ The function $f$ is \emph{one-to-one} $(1-1)$ if $a_{1} \neq a_{2}$ in $A$ implies that $f\left(a_{1}\right) \neq$ $f\left(a_{2}\right)$ in $B$. The function $f$ is \emph{onto} if, given any $b \in B$, it is possible to find an element $a \in A$ for which $f(a)=b$
  Give an example of each or state that the request is impossible:
  \enum{
  \item $f: \mathbf{N} \rightarrow \mathbf{N}$ that is $1-1$ but not onto.
  \item $f: \mathbf{N} \rightarrow \mathbf{N}$ that is onto but not $1-1$.
  \item $f: \mathbf{N} \rightarrow \mathbf{Z}$ that is $1-1$ and onto.
  }
\end{exercise}

\begin{solution}
  \enum{
  \item Let $f(n) = n + 1$ does not have a solution to $f(a) = 1$
  \item Let $f(1) = 1$ and $f(n) = n - 1$ for $n > 1$
  \item Let $f(n) = n/2$ for even $n$, and $f(n) = -(n+1)/2$ for odd $n$.
  }
\end{solution}


\begin{exercise}
  Given a function $f: D \rightarrow \mathbf{R}$ and a subset $B \subseteq \mathbf{R}$, let $f^{-1}(B)$ be the set of all points from the domain $D$ that get mapped into $B ;$ that is, $f^{-1}(B)=\{x \in D: f(x) \in B\} .$ This set is called the \emph{preimage} of $B$.
  \enum{
  \item Let $f(x)=x^{2} .$ If $A$ is the closed interval $[0,4]$ and $B$ is the closed interval $[-1,1]$, find $f^{-1}(A)$ and $f^{-1}(B)$. Does $f^{-1}(A \cap B)=f^{-1}(A) \cap f^{-1}(B)$ in this case? Does $f^{-1}(A \cup B)=f^{-1}(A) \cup f^{-1}(B) ?$
  \item The good behavior of preimages demonstrated in (a) is completely general. Show that for an arbitrary function $g: \mathbf{R} \rightarrow \mathbf{R}$, it is always true that $g^{-1}(A \cap B)=g^{-1}(A) \cap g^{-1}(B)$ and $g^{-1}(A \cup B)=g^{-1}(A) \cup g^{-1}(B)$ for all sets $A, B \subseteq \mathbf{R}$
  }
\end{exercise}

\begin{solution}
  \enum{
  \item $f^{-1}(A) = [-2, 2]$, $f^{-1}(B) = [-1, 1]$, $f^{-1}(A\cap B) = [-1, 1] = f^{-1}(A) \cap f^{-1}(B)$, $f^{-1}(A \cup B) = [-2, 2] = f^{-1}(A) \cup f^{-1}(B)$
  \item The problem we had in 1.2.7 does not apply to preimages, before \TODO
  }
\end{solution}


\begin{exercise}
  Decide which of the following are true statements. Provide a short justification for those that are valid and a counterexample for those that are not:
  \enum{
  \item Two real numbers satisfy $a<b$ if and only if $a<b+\epsilon$ for every $\epsilon>0$.
  \item Two real numbers satisfy $a<b$ if $a<b+\epsilon$ for every $\epsilon>0$.
  \item Two real numbers satisfy $a \leq b$ if and only if $a<b+\epsilon$ for every $\epsilon>0$.
  }
\end{exercise}


\begin{solution}
  \enum{
  \item False, if $a=b$ then $a < b + \epsilon$ forall $\epsilon > 0$ but $a \not < b$
  \item False, see above
  \item True, suppose $a < b + \epsilon$ forall $\epsilon > 0$, We want to show this implies $a \le b$. We either have $a \le b$ or $a > b$, but $a > b$ is impossible since the gap implies there exists an $\epsilon$ small enough such that $a > b + \epsilon$. Now suppose $a \le b$, obviously $a < b + \epsilon$ forall $\epsilon > 0$.
  }
\end{solution}

\begin{exercise}
  Form the logical negation of each claim. One trivial way to do this is to simply add ``It is not the case that...'' in front of each assertion. To make this interesting, fashion the negation into a positive statement that avoids using the word ``not'' altogether. In each case, make an intuitive guess as to whether the claim or its negation is the true statement.
  \enum{
  \item For all real numbers satisfying $a<b$, there exists an $n \in \mathbf{N}$ such that $a+1 / n<b$
  \item There exists a real number $x>0$ such that $x<1 / n$ for all $n \in \mathbf{N}$.
  \item Between every two distinct real numbers there is a rational number.
  }
\end{exercise}

\begin{solution}
  \enum{
  \item There exist real numbers satisfying $a < b$ where $a + 1/n \ge b$ forall $n \in \mathbf N$ 
  \item \TODO
  \item \TODO
  }
\end{solution}


\begin{exercise}
  Let $y_{1}=6$, and for each $n \in \mathbf{N}$ define $y_{n+1}=\left(2 y_{n}-6\right) / 3$
  \enum{
  \item Use induction to prove that the sequence satisfies $y_{n}>-6$ for all $n \in \mathbf{N}$.
  \item Use another induction argument to show the sequence $\left(y_{1}, y_{2}, y_{3}, \ldots\right)$ is decreasing.
  }
\end{exercise}

\begin{solution}
  \enum{
  \item \TODO
  \item \TODO
  }
\end{solution}

\begin{exercise}
  For this exercise, assume Exercise \ref{demorgan} has been successfully
  \enum{
  \item Show how induction can be used to conclude that
    $$
    \left(A_{1} \cup A_{2} \cup \cdots \cup A_{n}\right)^{c}=A_{1}^{c} \cap A_{2}^{c} \cap \cdots \cap A_{n}^{c}
    $$
    for any finite $n \in \mathbf{N}$
  \item It is tempting to appeal to induction to conclude
    $$
    \left(\bigcup_{i=1}^{\infty} A_{i}\right)^{c}=\bigcap_{i=1}^{\infty} A_{i}^{c}
    $$
    but induction does not apply here. Induction is used to prove that a particular statement holds for every value of $n \in \mathbf{N}$, but this does not imply the validity of the infinite case. To illustrate this point, find an example of a collection of sets $B_{1}, B_{2}, B_{3}, \ldots$ where $\bigcap_{i=1}^{n} B_{i} \neq \emptyset$ is true for every $n \in \mathbf{N}$, but $\bigcap_{i=1}^{\infty} B_{i} \neq \emptyset$ fails.
  \item Nevertheless, the infinite version of De Morgan's Law stated in (b) is a valid statement. Provide a proof that does not use induction.
  }

\end{exercise}

\begin{solution}
  \enum{
  \item \TODO
  \item \TODO
  \item \TODO
  }
\end{solution}
