\chapter{The Real Numbers}


\setcounter{section}{1} % skip the first section, it has no exercises
\section{Some Preliminaries}

\begin{exercise}
  \enum{
  \item Prove that $\sqrt 3$ is irrational. Does a similar similar argument work to show $\sqrt 6$ is irrational?
  \item Where does the proof break down if we try to prove $\sqrt 4$ is irrational?
  }
\end{exercise}

\begin{solution}
  \enum{
  \item
    Suppose for contradiction that $p/q$ is a fraction in lowest terms, and that $(p/q)^2 = 3$.
    Then $p^2 = 3q^2$ implying $p$ is a multiple of $3$ since $3$ is not a perfect square. Therefor we can write $p$ as $3r$ for some $r$, substituting we get $(3r)^2 = 3q^2$ and $3r^2 = q^2$ implying $q$ is also a multiple of $3$ contradicting the assumption that $p/q$ is in lowest terms. \\
    For $\sqrt 6$ the same argument applies, since $6$ is not a perfect square.

  \item $4$ is a perfect square, meaning $p^2 = 4q^2$ does not imply that $p$ is a multiple of four as $p$ could be $2$.
  }
\end{solution}

\begin{exercise}
  Show that there is no rational number satisfying $2^r = 3$
\end{exercise}

\begin{solution}
  Letting $r = p/q$ we have $2^{p/q} = 3$ implying $2^p = 3^q$ which is impossible since $2$ and $3$ are coprime.
\end{solution}


\begin{exercise}
  Decide which of the following represent true statements about the nature of sets. For any that are false, provide a specific example where the statement in question does not hold.
  \enum{
  \item If $A_{1} \supseteq A_{2} \supseteq A_{3} \supseteq A_{4} \cdots$ are all sets containing an infinite number of elements, then the intersection $\bigcap_{n=1}^{\infty} A_{n}$ is infinite as well.
  \item If $A_{1} \supseteq A_{2} \supseteq A_{3} \supseteq A_{4} \cdots$ are all finite, nonempty sets of real numbers, then the intersection $\bigcap_{n=1}^{\infty} A_{n}$ is finite and nonempty.
  \item $A \cap(B \cup C)=(A \cap B) \cup C$.
  \item $A \cap(B \cap C)=(A \cap B) \cap C$.
  \item $A \cap(B \cup C)=(A \cap B) \cup(A \cap C)$.
  }
\end{exercise}

\begin{solution}
  \enum{
  \item False, consider $A_1 = \{1, 2, \dots\}, A_2 = \{2, 3, \dots\}$, ... has $\bigcap_{n=1}^\infty A_n = \emptyset$.
  \item True.
  \item False, $A = \emptyset$ gives $\emptyset = C$.
  \item True, intersection is associative.
  \item True, draw a diagram.
  }
\end{solution}

\begin{exercise}
  Produce an infinite collection of sets $A_{1}, A_{2}, A_{3}, \ldots$ with the property that every $A_{i}$ has an infinite number of elements, $A_{i} \cap A_{j}=\emptyset$ for all $i \neq j$, and $\bigcup_{i=1}^{\infty} A_{i}=\mathbf{N}$
\end{exercise}

\begin{solution}
  This question is asking us to partition $\mathbf N$ into an infinite collection of sets.
  This is equivalent to asking us to unroll $\mathbf N$ into a square, which we can do along the diagonal
  $$
  \begin{array}{lccccc}
    1 & 3 & 6 & 10 & 15 & \cdots \\
    2 & 5 & 9 & 14 & \cdots & \\
    4 & 8 & 13 & \cdots & & \\
    7 & 12 & \cdots & & & \\
    11 & \ldots & & & & \\
    \vdots & & & & &
  \end{array}
  $$
\end{solution}

\begin{exercise}[De Morgan's Laws]
  Let $A$ and $B$ be subsets of $\mathbf R$.
  \enum{
  \item If $x \in(A \cap B)^{c}$, explain why $x \in A^{c} \cup B^{c}$. This shows that $(A \cap B)^{c} \subseteq$ $A^{c} \cup B^{c}$

  \item Prove the reverse inclusion $(A \cap B)^{c} \supseteq A^{c} \cup B^{c}$, and conclude that $(A \cap B)^{c}=A^{c} \cup B^{c}$
  \item Show $(A \cup B)^{c}=A^{c} \cap B^{c}$ by demonstrating inclusion both ways.
  }
\end{exercise}

\begin{solution}
  \enum{
  \item \TODO
  \item \TODO
  \item \TODO
  }
\end{solution}


\begin{exercise}
  \enum{
  \item Verify the triangle inequality in the special case where $a$ and $b$ have the same sign.
  \item Find an efficient proof for all the cases at once by first demonstrating $(a+b)^{2} \leq(|a|+|b|)^{2}$
  \item Prove $|a-b| \leq|a-c|+|c-d|+|d-b|$ for all $a, b, c$, and $d$.
  \item Prove $\| a|-| b|| \leq|a-b|$. (The unremarkable identity $a=a-b+b$ may be useful.)
  }
\end{exercise}

\begin{solution}
  \enum{
  \item We have equality $|a + b| = |a| + |b|$ meaning $|a + b| \le |a| + |b|$ also holds.
  \item $(a+b)^2 \le (|a| + |b|)^2$ reduces to $2ab \le 2|a||b|$ which is obviously true.
    and since squaring preserves inequality this implies $|a + b| \le |a| + |b|$. Showing that squaring preserves inequality is left as an exercise to the reader.
  \item I would like to do this using the triangle inequality, I notice that $(a-c)+(c-d)+(d-b) = a-b$. Meaning I can use the triangle inequality for multiple terms
    $$|a - b| = |(a-c)+(c-d)+(d-b)| \le |a-c| + |c-d| + |d-b|$$
    The general triangle inequality is proved by repeated application of the two variable inequality
    $$|(a + b) + c| \le |a + b| + |c| \le |a| + |b| + |c|$$
  \item I would like to cancel the subtraction inside $||a| - |b||$ since then the inside will be positive, and the outer absolute value will vanish. \TODO
  }
\end{solution}
